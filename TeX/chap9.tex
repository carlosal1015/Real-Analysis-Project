\documentclass{amsart}
\newtheorem{definition}{Definición}
\begin{document}
\section{Definiciones básicas y modelos}
En esta sección presentamos a nuestros lectores las nociones básicas subyacentes de las relaciones de recurrencia, así como varios ejemplos de tales relaciones.
\subsection{Primeras definiciones}
Una relación de recurrencia es una familia numerable de ecuaciones que defina sucesiones en modo recursivo. Las sucesiones que así surgen se llaman \emph{soluciones de la recurrencia}, dependiendo de uno o más de los datos iniciales: cada término que sigue el dato inicial en tales sucesiones es definida como una función de los términos anteriores.

\begin{definition}
	Una \textbf{relación de recurrencia} en las incógnitas $x_{i}$, $i\in\mathbb{N}$, es una familia de ecuaciones \[ x_{n}=f_{n}\left(x_{0},\ldots,x_{n-1}\right),\quad n\geq r, \] donde $r\in\mathbb{N}_{\geq1}$, y ${\left(f_{n}\right)}_{n\geq r}$ son funciones \[ f_{n}\colon D_{n}\rightarrow\mathbb{R},\quad D_{n}\subseteq\mathbb{R}^{n},\text{ o }f_{n}\colon D_{n}\rightarrow\mathbb{C},\quad D_{n}\subseteq\mathbb{C}^{n}. \] Dependiendo en el caso encontrado, hablaremos de \textbf{recurrencias reales} o \textbf{recurrencias complejas}. Las incógnitas $x_{0},\ldots,x_{r-1}$ son llamadas \textbf{libres}. Su número $r$ es el \textbf{orden} de la relación.
	
	Al reemplazar $x$ con $y$, la relación de recurrencia de orden $r$ \[ x_{n}=f_{n}\left(x_{0},\ldots,x_{n-1}\right),\quad n\geq r, \] puede también escribirse como \[ x_{n+r}=f_{n+r}\left(x_{0},\ldots,x_{n+r-1}\right),\quad n\geq0. \]
\end{definition}

\begin{definition}
	Una sucesión ${\left(a_{n}\right)}_{n}$ es una \textbf{solución} de la relación de recurrencia de orden $r$
	\begin{equation}
	x_{n}=f_{n}\left(x_{0},\ldots,x_{n-1}\right),\quad n\geq r,
	\end{equation}
	con $f_{n}\colon D_{n}\rightarrow\mathbb{R}$, $D_{n}\in\mathbb{R}^{n}$, si \[ \left(a_{0},\ldots,a_{n-1}\right)\in D_{n},\quad a_{n}=f_{n}\left(a_{0},a_{1},\ldots,a_{n-1}\right)\quad\forall\,n\geq r. \]
\end{definition}

La sucesión $\left(a_{0},\ldots,a_{n-1}\right)$ de valores asignados para las $r$ incógnitas libres es llamado la $r$--sucesión de \textbf{valor inicial} o de las \textbf{condiciones iniciales} de la solución. Definimos la \textbf{solución general real} (\textbf{respectivamente compleja}) de la sucesión como la familia de todas las soluciones con elementos que están en $\mathds{R}$ (respectivamente, en $\mathds{C}$).

\begin{example}
	Considere la relación de recurrencia de primer orden definida por \[ x_{n}=\frac{1}{x_{n-1}-1},\quad n\geq1. \]
\end{example}
La $1$--sucesión (2) no es una sucesión de valor inicial de una solución, en efecto, $2$ pertenece al dominio de $f_{0}\left(x\right)=\frac{1}{x-1}$, pero $\left(2,f_{0}(2)\right)=\left(2,1\right)$ no pertenece al dominio de $f_{1}\left(x_{0},x_{1}\right)=\frac{1}{x-1}$. Por otra parte, la $1$--sucesión (3) es en efecto la sucesión de valor inicial de la solución (sucesión) \[ \left(a_{n}\right)_{n}\coloneqq\left(3,1/2,-2,-1/3,-3/4,-4/7,-7/11,\ldots\right). \]
Note que para $n\geq2$ uno tiene $a_{n}<0$ y así $a_{n+1}=\frac{1}{a_{n}-1}<0$ es distinto de $1$.

\begin{example}
	En muchas ocasiones una relación de recurrencia de orden $r$ involucra solo los últimos $r$ términos y es de la forma \[ x_{n}=g_{n}\left(x_{n-r},\ldots,x_{n-1}\right),\quad n\geq r, \] donde ${\left(g_{n}\right)}_{n\geq r}$ son las funciones definidas en un subconjunto $E_{n}$ de $\mathds{R}^{r}$ o $\mathds{C}^{r}$. Este último es de hecho una relación de recurrencia: es suficiente para establecer $f_{n}\left(x_{0},\ldots,x_{n-1}\right)$$\coloneqq$ g_{n}$\left(x_{n-r},\ldots,x_{n-1}\right)$ para $\left(x_{0},\ldots,x_{n-1}\right)\in D_{n}\coloneqq\mathds{R}^{n-r}\times E_{n}$ (o $\mathds{C}^{n-r}\times E_{n}$) a fin de cumplir los requerimientos de la definición %9.2
\end{example}
\subsection{Algunos modelos de recurrencias lineales}
Ahora damos una serie de ejemplos que ilustran cómo reducir la solución de un problema en el que la búsqueda de las soluciones de una relación de recurrencia apropiada.
\begin{example}
	Un niño decide escalar una escalera con $n\geq 1$ de tal manera que cada paso que él despeja uno o dos de los pasos de la escalera %(vea)
	Encuentre la relación de recurrencia que sirva para calcular el número de diferentes maneras posibles de escalar la escalera.
\end{example}
\begin{proof}[Solución]
	Usamos la variable desconocida $x_{n}$ para denotar el número de maneras en las cuales el niño puede escalar la escalera de $n\geq1$ pasos. Es fácil de observar que $x_{1}=1$ y $x_{2}=2$ (dos pasos cada uno de longitud uno, o un paso de longitud dos escalones). Ahora sea $n\geq3$: si con el primer paso el niño mueve solo el primer escalón; existen claramente $x_{n-1}$ posibles maneras de escalar los que quedan. Si en cambio con el primer lugar, se  suben dos peldaños de escalera.
\end{proof}
\end{document}

https://rajsain.files.wordpress.com/2013/11/randomized-algorithms-motwani-and-raghavan.pdf

https://www.csie.ntu.edu.tw/~r97002/temp/Concrete%20Mathematics%202e.pdf

https://link.springer.com/chapter/10.1007/978-3-642-61544-3_9

https://link.springer.com/chapter/10.1007/978-94-011-1814-9_9

https://link.springer.com/chapter/10.1007/978-3-319-15579-1_39

https://link.springer.com/chapter/10.1007/978-94-011-2058-6_14

https://link.springer.com/chapter/10.1007/BFb0120904

https://link.springer.com/chapter/10.1007%2FBFb0120904

https://link.springer.com/article/10.1007/BF00874886

https://link.springer.com/search?date-facet-mode=between&showAll=true&query=recurrence+AND+relation&facet-discipline=%22Mathematics%22