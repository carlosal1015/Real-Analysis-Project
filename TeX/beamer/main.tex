% arara: clean: {
% arara: --> extensions:
% arara: --> ['log','idx','ilg','ind','out','bbl','blg','thm','toc','aux','synctex.gz','bcf','run.xml','contb',
% arara: --> 'contg','contn','diffb','diffg','diffn','funb','fung','funn','genb','geng','genn','intb','intg','pytxcode',
% arara: --> 'intn','limb','limg','limn','logb','logg','logn','realb','realg','realn','seqb','seqg','seqn','ist',
% arara: --> 'serb','serg','sern','setb','setg','setn','ssfunb','ssfung','ssfunn','topb','topg','topn','pdf']
% arara: --> }
% arara: pdflatex: {
% arara: --> shell: yes,
% arara: --> draft: yes,
% arara: --> }
% arara: biber
% arara: pdflatex: {
% arara: --> shell: yes,
% arara: --> draft: yes,
% arara: --> }
% arara: pdflatex: {
% arara: --> shell: yes,
% arara: --> synctex: yes,
% arara: --> interaction: batchmode
% arara: --> }
% arara: clean: {
% arara: --> extensions:
% arara: --> ['log','idx','ilg','ind','out','bbl','blg','thm','toc','aux','synctex.gz','bcf','run.xml','contb',
% arara: --> 'contg','contn','diffb','diffg','diffn','funb','fung','funn','genb','geng','genn','intb','intg','pytxcode',
% arara: --> 'intn','limb','limg','limn','logb','logg','logn','realb','realg','realn','seqb','seqg','seqn','ist',
% arara: --> 'serb','serg','sern','setb','setg','setn','ssfunb','ssfung','ssfunn','topb','topg','topn']
% arara: --> }
\documentclass[spanish,10pt,utf8,handout,xcolor=table,aspectratio=1610]{beamer} %
\usepackage[T1]{fontenc}
\usepackage{mathpazo}
\usepackage[spanish]{babel}
\usepackage{amsmath,mathrsfs,amsfonts,amsthm}
\usepackage{minted}
\usepackage{enumitem}
\usepackage{booktabs}

\usepackage[backend=biber,style=numeric, defernumbers=true, sorting=ynt,maxbibnames=4,maxcitenames=4]{biblatex}
\addbibresource{bibliography/reference.bib}
% https://tex.stackexchange.com/questions/410249/conflict-between-marvosym-and-mathabx
\newcommand{\MVAt}{{\usefont{U}{mvs}{m}{n}\symbol{`@}}}
\renewcommand{\spanishfigurename}{Figura}
\renewcommand{\spanishcontentsname}{Índice analítico}
\renewcommand{\listingscaption}{Programa}

% https://tex.stackexchange.com/questions/68080/beamer-bibliography-icon
\setbeamertemplate{bibliography item}{%
	\ifboolexpr{ test {\ifentrytype{book}} or test {\ifentrytype{mvbook}}
		or test {\ifentrytype{collection}} or test {\ifentrytype{mvcollection}}
		or test {\ifentrytype{reference}} or test {\ifentrytype{mvreference}} }
	{\setbeamertemplate{bibliography item}[book]}
	{\ifentrytype{online}
		{\setbeamertemplate{bibliography item}[online]}
		{\setbeamertemplate{bibliography item}[article]}}%
	\usebeamertemplate{bibliography item}}

\defbibenvironment{bibliography}
{\list{}
	{\settowidth{\labelwidth}{\usebeamertemplate{bibliography item}}%
		\setlength{\leftmargin}{\labelwidth}%
		\setlength{\labelsep}{\biblabelsep}%
		\addtolength{\leftmargin}{\labelsep}%
		\setlength{\itemsep}{\bibitemsep}%
		\setlength{\parsep}{\bibparsep}}}
{\endlist}
{\item}
%\usetheme[numbering=fullbar]{focus}%progressbar
%\definecolor{main}{RGB}{92, 138, 168}
%\definecolor{background}{RGB}{240, 247, 255}
\usepackage[
	audience=english
%	audience=spanish,
%	audience=german
]{beameraudience}

%\newcommand{\progressbar}{%
%	\pgfmathsetmacro{\theta}{360/\inserttotalframenumber*\insertframenumber}
%	\begin{tikzpicture}[scale=0.035]
%	\fill[yellow] (0,0) circle (9);
%	\fill[red] (0,0) -- (9,0) arc (0:-\theta:9);
%	\fill[white] (0,0) circle (5);
%	\node at (0,0) {\insertframenumber};
%	\end{tikzpicture}
%}

%\setbeamertemplate{footline}{\hfill \progressbar}
\usepackage{textpos}
\usepackage{mathtools}
\usepackage{dsfont}
%\useoutertheme{splitwithminiframes}
%\useoutertheme{sidebarwithminiframes}
%\usecolortheme[cautious]{owl}
\makeatletter
\def\th@claim{%
	\normalfont % body font
	\setbeamercolor{block title}{bg=white,fg=orange}
	\setbeamercolor{block body}{bg=white,fg=black}
}
\makeatother
\theoremstyle{claim}
\newtheorem{claim}[theorem]{Afirmación}

\makeatletter
\def\th@remark{%
	\normalfont % body font
	\setbeamercolor{block title}{bg=white,fg=red}
	\setbeamercolor{block body}{bg=white,fg=black}
}
\makeatother
\theoremstyle{remark}
\newtheorem{remark}[theorem]{Observación}

\theoremstyle{definition}

\usetheme{CambridgeUS}
\usecolortheme{dolphin}
\useinnertheme{rectangles}
%\useoutertheme[hooks]{tree}

\usefonttheme[onlymath]{serif}

%\setbeamercovered{transparent}
\setbeamercovered{dynamic}

\setbeamertemplate{background canvas}{\includegraphics[height=\paperheight]{background}}
\setbeamertemplate{footline}[frame number]{}
\setbeamertemplate{navigation symbols}{}
\setbeamertemplate{footline}{}
\setbeamertemplate{headline}{}
\setbeamertemplate{blocks}[rounded][shadow=false]

\addtobeamertemplate{block begin}{\pgfsetfillopacity{0.6}}{\pgfsetfillopacity{1}}
\addtobeamertemplate{block alerted begin}{\pgfsetfillopacity{0.6}}{\pgfsetfillopacity{1}}
\addtobeamertemplate{block example begin}{\pgfsetfillopacity{0.6}}{\pgfsetfillopacity{1}}

\title[Teorema de los cuatro colores]{\Huge\sffamily Relaciones de recurrencias}
\subtitle{Ecuaciones en diferencias y análisis en escalas de tiempo}

\author[Grupo N$^\circ6$]{%
	\texorpdfstring{%
		\begin{columns}
			\column{.3\linewidth}
			\centering
			C. Aznarán Laos %\inst{1,2}
			\column{.3\linewidth}
			\centering
			F. Cruz Ordoñez %\inst{1,2}
		\end{columns}
		\vspace{12pt}
		\begin{columns}
			\column{.3\linewidth}
			\centering
			G. Quiroz Gómez %\inst{1,2}
			\column{.3\linewidth}
			\centering
			J. Micha Velasque %\inst{1,2}
		\end{columns}
		\vspace{12pt}
		\begin{columns}
			\column{.3\linewidth}
			\centering
			D. García Fernández %\inst{1,2}
%			\column{.3\linewidth}
			\centering
		\end{columns}
	}
	{Author 1, Author 2, Author 3}
}

\institute[FC -- UNI]{\large%\inst{1}
	Facultad de Ciencias \and%\inst{2}
	Universidad Nacional de Ingeniería
}
\date{25, 27 de junio del 2019}

\titlegraphic{
	\begin{picture}(0,0)
	\put(190,100){\makebox(0,0)[rt]{\includegraphics[width=2.5cm]{logouni}}}
	\end{picture}
}

\graphicspath{{images/}}

\AtBeginSubsection[]
{
	\begin{frame}<beamer>
		\frametitle{\contentsname}
		\tableofcontents[
		currentsection,
		sectionstyle=show/show,
		subsectionstyle=show/shaded/hide%-show/shaded/hide
		]
	\end{frame}
}
%\includeonlyframes{%
%	toc,%
%}
\begin{document}

\frame{\titlepage} %\begin{frame}[plain]\maketitle\end{frame}
\include{./contents/toc}
%%%%%%%%%%%%%%%%%%%%%%%%%%%%%%%%%%%%%%%%%%%%%%%%%%%%%%%%%%%%%%%%%
\section{Introducción}
\subsection{Relación de recurrencia}

\begin{frame}
\frametitle{\secname\hspace{0pt plus 1 filll}\subsecname}%\beamergotobutton{right}

\begin{definition}
	Una \textbf{relación de recurrencia} en las incógnitas $x_{i}$, $i\in\mathds{N}$, es una familia de ecuaciones
		\begin{equation}\label{eq:recurrence}
			x_{n}=f_{n}\left(x_{0},\ldots,x_{n-1}\right),\quad \forall n\geq r,
		\end{equation}
	donde $r\in\mathds{N}$, y ${\left(f_{n}\right)}_{n\geq r}$ son funciones \[ f_{n}\colon D_{n}\rightarrow\mathds{R},\quad D_{n}\subseteq\mathds{R}^{n},\quad\text{o}\quad f_{n}\colon D_{n}\rightarrow\mathds{C},\quad D_{n}\subseteq\mathds{C}^{n}. \]

	Dependiendo del caso encontrado, las llamaremos \textbf{recurrencias reales}\index{Relación de recurrencia!real} o \textbf{recurrencias complejas}\index{Relación de recurrencia!compleja}. Las incógnitas $x_{0},\ldots,x_{r-1}$ son llamadas \textbf{libres}. Su número $r$ es el \textbf{orden} de la relación\index{Relación de recurrencia!orden}.
\end{definition}

\begin{definition}
	Una sucesión ${\left(a_{n}\right)}_{n}$ es una \textbf{solución} de~\eqref{eq:recurrence}, sii
		\begin{equation*}
		\left(a_{0},\ldots,a_{n-1}\right)\in D_{n},\quad a_{n}=f_{n}\left(a_{0},a_{1},\ldots,a_{n-1}\right)\quad\forall\,n\geq r.
		\end{equation*}
\end{definition}
\end{frame}

\begin{frame}
\frametitle{\secname}
\framesubtitle{\subsecname}

\begin{example}
	La sucesión real definida por \[ x_{0}=2, x_{1}=1, x_{2}=2^{1/2}, x_{3}=1, \ldots, x_{2m-1}=1 ,x_{2m}=2^{1/2^{m}}, \ldots \] es la \alert{solución} de la \emph{relación de recurrencia} con coeficientes reales \[ x_{n}=\sqrt{x_{n-2}},\quad \forall n\geq2, \] y los \emph{valores iniciales} $x_{0}=2$ y $x_{1}=1$.
\end{example}

\begin{example}
	Considere la relación de recurrencia de primer orden definida por \[ x_{n}=\frac{1}{x_{n-1}-1},\quad\forall n\geq1.\]
	\begin{itemize}[topsep=0pt]
		\item La $1$--tupla $\left(2\right)\in D_{0}$ de valor inicial \alert{no es una solución}.
		\item La $1$--tupla $\left(3\right)\in D_{0}$ de valor inicial \alert{es una solución}.
	\end{itemize}
\end{example}
\end{frame}

\begin{frame}
\frametitle{\secname}
\framesubtitle{\subsecname}

\begin{remark}
	En muchas ocasiones una relación de recurrencia de orden $r$ involucra solo los últimos $r$ términos y es de la forma \[ x_{n}=g_{n}\left(x_{n-r},\ldots,x_{n-1}\right),\quad n\geq r, \] donde ${\left(g_{n}\right)}_{n\geq r}$ son las funciones definidas en un subconjunto $E_{n}$ de $\mathds{R}^{r}$ o $\mathds{C}^{r}$.

	\

	Este último es de hecho una relación de recurrencia: es suficiente establecer \[ f_{n}\left(x_{0},\ldots,x_{n-1}\right)\coloneqq g_{n}\left(x_{n-r},\ldots,x_{n-1}\right) \] para $\left(x_{0},\ldots,x_{n-1}\right)\in D_{n}\coloneqq\mathds{R}^{n-r}\times E_{n}$ (o $\mathds{C}^{n-r}\times E_{n}$) a fin de cumplir los requerimientos de la definición~\eqref{eq:recurrence}.
\end{remark}
\end{frame}
\begin{frame}
	\frametitle{\subsecname}
	Aquí es conveniente representar cualquier sucesión de números reales $(a_{n})_{n} $ como la función $f\colon\mathds{N}\rightarrow\mathds{R}$ definido por: \[ f(n)=a_{n},\quad\forall n\in\mathds{N}. \]
	\begin{definition}
		Una \textbf{ecuación en diferencias} es una expresión de la forma:
		\begin{equation}\label{eq:diffeq}
		G\left(n,f(n),f\left(n+1\right),\ldots,f\left(n+k\right)\right)=0,\quad\forall n\in\mathds{N}.
		\end{equation}
	\end{definition}
	El \textbf{orden} de una ecuación en diferencias se halla mediante la diferencia entre los ``términos mayor'' y ``menor'' respectivamente. En~(2), es \alert{$n+k-n=k$}.%\eqref{eq:diffeq}
	\begin{example}
		\begin{itemize}
			\item El \alert{orden} de $f\left(n+3\right)-f\left(n+1\right)-5f(n)=n$ es \alert{$3$}.
			\item El \alert{orden} de $f\left(n+3\right)-f\left(n+1\right)=n^{2}-3$ es \alert{$2$}.
		\end{itemize}
	\end{example}
\end{frame}

\begin{frame}
	\begin{definition}
		La \textbf{solución} de~(2) a toda sucesión $\{f\left(0\right),f\left(1\right),\ldots,f(n),\ldots\}$ que la satisfaga, ahora se le llama \emph{solución general} de una E.D al conjunto de todas las soluciones que tendrán tanto parámetros como orden tenga la ecuación. La determinación de estos parámetros, a partir de unas condiciones iniciales, nos proporcionará las distintas soluciones particulares.
	\end{definition}

	\begin{example}
		Sea \[ f\left(n+1\right)-f\left(n\right)=3 \] una ecuación en diferencias de orden uno cuya solución general es $f\left(n\right)=3n+c$.

	\

		Si consideramos las condiciones iniciales, por ejemplo, $f(0)=2$, entonces $f(0)=3\times0+c=c$, por tanto $c=2$ y la solución particular es $f_{p}(n)=3n+2$.
		Es decir, la solución es la sucesión $f_{p}(n)=\left\{2,5,8,11,\ldots\right\}$.
	\end{example}
\end{frame}

\begin{frame}
	\begin{definition}
		Llamamos ecuación en diferencias lineal de orden $k$ a toda expresión de la forma:
		\[ f\left(n+k\right)+a_{1}(n)f\left(n+k-1\right)+\cdots+a_{k-1}(n)f\left(n+1\right)+a_{k}(n)f\left(n\right)=b\left(n\right), \]
		donde $a_{k}(n)\neq0$.
	\end{definition}

	\begin{block}{Clasificación de las ecuaciones de diferencias lineal}
		\begin{itemize}
			\item Homogéneas si $b(n)=0$.
			\item Completas si $b(n)\neq0$.
			\item De coeficientes constantes si $a_{i}(n)=a_{i}$, $\forall i$.
			\item De coeficientes no constantes si $a_{i}(n)\neq a_{i}$ para algún $i$.
		\end{itemize}
	\end{block}
\end{frame}

\begin{frame}
	\begin{theorem}[De la existencia y la unicidad]
		Dada la ecuación: \[ f\left(n+k\right)+a_{1}(n)f\left(n+k-1\right)+\cdots+a_{n-1}(n)f\left(n+1\right)+a_{n}(n)f\left(n\right)=0, \] y dados $n$ números reales $k_{0}$, $k_{1}$, \ldots, $k_{n-1}$ existe una única solución que verifica \[ f\left(0\right)=k_{0},f\left(1\right)=k_{1},\ldots,f\left(n-1\right)=k_{n-1}. \]
	\end{theorem}

	\begin{theorem}
		Toda combinación lineal de soluciones de una ecuación en diferencias lineal homogénea de orden $n$ es también una solución.
	\end{theorem}

	\begin{corollary}
		Las soluciones de una ecuación en diferencia lineal de orden $n$ forman un espacio vectorial.
	\end{corollary}

	\begin{theorem}
		La dimensión del espacio de soluciones de una ecuación en diferencias lineal de orden $n$ es $n$.
	\end{theorem}
\end{frame}

%http://personal.us.es/pnadal/Informacion/leccion5ecdiferencias.pdf
%\begin{frame}
%	\begin{block}{Ecuaciones en diferencias de primer orden}
%	\end{block}
%	
%	\begin{block}{Ecuaciones en diferencias de segundo orden}
%	\end{block}
%\end{frame}

\subsubsection{Número de Catalan}

\begin{frame}{Número de Catalan}
	\begin{block}{Triangulación}
		\begin{figure}
			\centering
			\includegraphics[height=0.3\paperheight]{ca1}
		\end{figure}
	\end{block}
	
	\begin{block}{Caminos monótonos}
		\begin{figure}
			\centering
			\includegraphics[height=0.3\paperheight]{ca2}
		\end{figure}
	\end{block}
\end{frame}
\section{Ecuaciones de recurrencia}

\begin{frame}
\frametitle{\secname}

\textit{Resolver una ecuación de recurrencia} significa encontrar una sucesión que satisfaga las ecuaciones de recurrencia.

\

Encontrar una ``solución general'' significa encontrar una fórmula que describe todas las soluciones posibles (todas las sucesiones posibles que satisfacen la ecuación).

\

Veamos el siguiente ejemplo:

\begin{claim}
	Considere que $T_{n}$ satisface la siguiente ecuación
	\begin{equation}
	T_{n}=2T_{n-1}+1\quad\forall n\in\mathds{N}\setminus\left\{1\right\}.
	\end{equation}
\end{claim}

\end{frame}

\subsection{Ejemplos definidos por ecuaciones de recurrencia}

\begin{frame}
\frametitle{\subsecname}

\begin{example}[Desajustes]
	Imagine una fiesta donde las parejas llegan juntas, pero al final de la noche, cada persona se va con una nueva pareja. Para cada $n\in\mathds{N}$, digamos que $D_{n}$ es el número de diferentes formas en que las parejas pueden ser ``trastornadas'', es decir, reorganizadas en parejas, por lo que ni uno está emparejado con la persona con la que llegaron.
\end{example}

\begin{claim}[$D_{n}$ para cualquier valor de $n$]
	Para todo $n\geq4$ tendremos: \[ D_{n}=(n-1)\left\{D_{n-2}+D_{n-1}\right\}. \] Y la sucesión definida en $\mathds{N}$ es \[ S_{n} = A \times n!. \]
\end{claim}

\end{frame}

\begin{frame}
\frametitle{\subsecname}

\begin{theorem}[Desigualdad para la acotación de $D_{n}$]
Para todo $n\geq2$ tenemos: \[ \left(\frac{1}{3}\right)n!\leq D_{n}\leq\left(\frac{1}{2}\right)n!. \]
\end{theorem}

La mejor fórmula para $D_{n}$ que sabemos utiliza la función de ``entero más cercano''.

\begin{definition}
	Para cualquier número real $x$, se define \textbf{el entero más cercano a} $x$, $\lceil x\rfloor$:

	Si $x$ es escrito como $n+f$ donde $n$ es el entero $\lfloor x\rfloor$, y $f$ es una fracción donde $0\leq f<1$:

	\begin{itemize}
		\item Si $0\leq f<\frac{1}{2}$, entonces $\lceil x\rfloor=n$.
		\item Si $\frac{1}{2}\leq f<1$, entonces $\lceil x\rfloor=n+1$.
	\end{itemize}
\end{definition}

\begin{remark}
	Entonces $D_{n}=\lceil(n!)/e\rfloor$ cuando $e=2.71828182844\ldots$ es la base del logaritmo natural.

	$(n!)/e$  nunca es igual a $\lceil(n!)/e\rfloor+\frac{1}{2}$.
\end{remark}
\end{frame}

\subsubsection{Función de Ackermann}

\begin{frame}
\frametitle{\subsubsecname}
A finales de 1920's, el lógico y matemático alemán, discípulo de David Hilbert, Wilhelm Ackermann (1896–1962), dio un ejemplo de una función computable total que no es recursiva primitiva.

\begin{minipage}{0.45\paperwidth}
	\begin{definition}[Ackermann]
		Sea $A\colon\mathds{N}\times\mathds{N}\rightarrow\mathds{N}$ una función. Se define recursivamente usando tres reglas:
		\begin{enumerate}
			\item $A(1,n)=2$, $\forall n\geq 1$.
			\item $A(m,1)=2m$, $\forall m\geq 2$.
			\item Cuando $m>1$ y $n>1$ se tiene: $A\left(m,n\right)=A\left(A(m-1,n),n-1\right)$.
		\end{enumerate}
	\end{definition}
	\begin{remark}
		Entonces $A\left(2,n\right)=4,\quad\forall n\geq1$. Además $A\left(m,2\right)=2^{m},\quad\forall m\geq 1$. Seguidamente se puede continuar a calcular $A\left(m,3\right)=2\uparrow m$ con la función torre definida por $2\uparrow\left[k+1\right]=2^{2\uparrow k}$ con valor inicial $2\uparrow 1=2$, por el P.I.M.
	\end{remark}
\end{minipage}
\hfill
\begin{minipage}{0.45\paperwidth}
	\begin{listing}[H]
		\inputminted{python}{./code/ackermann.py}
		\caption{Programa \texttt{ackermann.py}}
	\end{listing}
\end{minipage}

\end{frame}
\begin{frame}
	Hola
	\frametitle{Soluciones}
	%------------------------------------------------------------ 1
	\only<1>{
		\framesubtitle{The first frame subtitle}
		\begin{itemize}
			\item some text on slide 1
	\end{itemize}}
	%------------------------------------------------------------ 2
	\only<2>{
		\framesubtitle{The second frame subtitle}
		\begin{itemize}
			\item some text on slide 2
	\end{itemize}}
\end{frame}
%%%%%%%%%%%%%%%%%%%%%%%%%%%%%%%%%%%%%%%%%%%%%%%%%%%%%%%%%%%%%%%%%
\section{Realización numérica}
\subsection{Discretización}
\subsubsection{Método de Euler}
\subsubsection{Método de Runge-Kutta}
\subsection{Escalas de tiempo}
\subsubsection{Derivada fraccionaria}
\subsection{Módulo \texttt{timescale}}

\frame[t]{
	Una escala de tiempo es un conjunto cerrado $\mathds{T}$ bajo la topología estándar sobre $\mathds{R}$.

	Se define el operador salto posterior $\sigma\colon\mathds{t}\rightarrow\mathds{T}$ por \[ \sigma\left(T\right)\coloneqq\inf\left\{z\in\mathds{T}:z>t\right\} \]

	la granicidad $\mu\colon\mathds{T}\rightarrow\mathds{R}$ por \[ \mu\left(t\right)=\sigma\left(t\right)-t \] y función granicidad minimal $\mu_{\ast}\colon\mathds{T}\rightarrow\mathds{R}$ por \[ \mu_{\ast}\left(s\right)=\inf_{\tau\in\left[s,\infty\right)\cap\mathds{T}}\mu\left(t\right). \]

	Una función $f\colon\mathds{T}\rightarrow\mathds{C}$ es llamada $\Delta$--diferenciable si para cualquier $\varepsilon>0$ existe $\delta>0$ tal que para todo $s\in\left(t-\delta,t+\delta\right)\cap\mathds{T}$ y existe un número $f^{\Delta}\left(t\right)$ tal que \[ |\left[f\left(\sigma\left(s\right)\right)-f\left(s\right)\right]f^{\Delta}\left(s\right)\left[\sigma\left(t\right)-s\right]|\leq\varepsilon|\sigma\left(t\right)-s|. \]

	\begin{figure}[H]
		\centering
		\includegraphics[width=0.4\paperwidth]{operators}
	\end{figure}

}

\frame{
	La integración se define de modo que $\int_{t}^{s}f^{\Delta}\left(\tau\right)\Delta\tau=f\left(t\right)-f\left(s\right)$.

	Si $\mathds{T}$ consiste únicamente de puntos aislados, entonces \[ f^{\Delta}\left(t\right)=\frac{f\left(\sigma\left(t\right)\right)-f\left(t\right)}{\mu\left(t\right)} \]

	y

\[
\int_{a}^{b}f\left(t\right)\Delta t=
\begin{cases}
	\sum_{t\in\left[a,b\right)\cap\mathds{T}}\mu\left(t\right)f\left(t\right),&\text{si }a<b.\\
	0, & \text{si } a = b.\\
	-\sum_{t\in\left[b,a\right)\cap\mathds{T}}\mu\left(t\right)f\left(t\right), & \text{si }a>b.
	\end{cases}
\]
}
%%%%%%%%%%%%%%%%%%%%%%%%%%%%%%%%%%%%%%%%%%%%%%%%%%%%%%%%%%%%%%%%%
\begin{frame}\transblindsvertical
\frametitle{Referencias}
	%------------------------------------------------------------ 1
	\only<1>{
		\framesubtitle{The first frame subtitle}
		\begin{itemize}
			\item Libros
			\nocite{*}
			\printbibliography[heading=none,keyword=book]
		\end{itemize}
	}
	%------------------------------------------------------------ 2
	\only<2>{
		\framesubtitle{The second frame subtitle}
		\begin{itemize}
			\item Artículos matemáticos
			\printbibliography[heading=none,keyword=paper]
		\end{itemize}
	}
	%------------------------------------------------------------ 3
	\only<2>{
		\framesubtitle{The second frame subtitle}
		\begin{itemize}
			\item Sitios web
			\printbibliography[heading=none,keyword=online]
		\end{itemize}
	}
\end{frame}
%\begin{frame}
\frametitle{Agradecimientos}
\begin{center}\Large
	¡Muchas gracias!
\end{center}

Colaboradores:

\begin{itemize}
	\item Creación del módulo \href{https://github.com/tomcuchta/timescalecalculus}{\texttt{timescalecalculus}}: Dr. Tom Cuchta y Matthias Baur.
	\item Tipografía en \LaTeX{}: Todo el grupo.
	\item Explicación del contenido matemático: Todo el grupo.
	\item Esquema de la exposición: Todo el grupo.
\end{itemize}
\vfill
\begin{minipage}{0.45\paperwidth}
	\textcolor{DarkBlue}{Presentación disponible en:}
	\begin{center}
		\href{https://github.com/carlosal1015/Real-Analysis-Project}{\includegraphics[width=2.5cm]{Octocat.png}}
	\end{center}
\end{minipage}
\hfill
\begin{minipage}{0.45\paperwidth}
	\begin{flushright}
		Dudas, sugerencias o preguntas a:

		\

		\href{mailto:caznaranl@uni.pe}{caznaranl\MVAt uni.pe}
		
		\href{mailto:fransscruz18@gmail.com}{fransscruz18\MVAt gmail.com}
		
		\href{mailto:yums123@hotmail.com}{yums123\MVAt hotmail.com}
		
		\href{mailto:junior_mv_194@hotmail.com}{junior\_mv\_194 \MVAt hotmail.com}
		
		\href{mailto:ge_qg_25@hotmail.com}{ge\_qg\_25\MVAt hotmail.com}
	\end{flushright}
\end{minipage}

\end{frame}
%\includeslide{}
%\appendix
%\frame[noframenumbering,plain]{text}

\setbeamertemplate{footline}
{%
	\hbox{%
		\begin{beamercolorbox}[wd=.3\paperwidth,ht=7.8pt,dp=3pt,center]{author in head/foot}\insertshortauthor
		\end{beamercolorbox}%
		\begin{beamercolorbox}[wd=.4\paperwidth,ht=7.8pt,dp=3pt,center]{title in head/foot}\insertshorttitle
		\end{beamercolorbox}%
		\begin{beamercolorbox}[wd=.3\paperwidth,ht=7.8pt,dp=3pt,center]{date in head/foot}\insertshortdate\hspace*{5mm}%
		\end{beamercolorbox}
	}%
}

\addtobeamertemplate{footline}{}{%
	\begin{textblock*}{100mm}(0.94\textwidth,-7mm)
		\progressbar
	\end{textblock*}
}
%\begin{frame}{Title}
% all version
%\end{frame}

%\justfor{english}{
%    \begin{frame}
%    English Slide
%    \pause
%    tba
%    \end{frame}
%}
%
%\justfor{spanish}{
%    \begin{frame}
%    Spanish Slide
%    \pause
%    tba
%    \end{frame}
%}
%
%\justfor{german}{
%    \begin{frame}
%    Deutcher Text
%    \pause
%    wird noch angekündigt
%    \end{frame}
%}
\end{document}
https://www.i-ciencias.com/pregunta/35252/termino-comun-para-las-ecuaciones-diferenciales-y-de-relaciones-de-recurrencia

@book{book:{2135209},
	title =     {Intelligent computations : abstract fractional calculus, inequalities, approximations},
	author =    {Anastassiou, George A},
	publisher = {Springer},
	isbn =      { 978-3-319-66936-6,3319669362,978-3-319-66935-9 },
	year =      {2018},
	series =    {Studies in computational intelligence 734},
	edition =   {},
	volume =    {},
	url =       {http://gen.lib.rus.ec/book/index.php?md5=edcebea5cf073b0368a47fba1cb9b635}}

Dada la ecuación en diferencias lineal de coeficientes constantes y de orden $k$: $a_{0}f(n + k)+a_{1}f(n+k¡1)+\cdots+a_{k}f(n)=g(n)$, el problema de hallar una función $f$ definida en $\mathds{Z}$; que verifique la ecuación, y tal que en los $k$ enteros consecutivos $n_{0},n_{0}+1,\ldots,n_{0}+k-1$ tome los valores dados $c_{0},c_{1},\ldots,c_{k-1}$; tiene solución única.