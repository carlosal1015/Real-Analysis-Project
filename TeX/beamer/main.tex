\documentclass{beamer}
%\usetheme[numbering=fullbar]{focus}%progressbar
%\definecolor{main}{RGB}{92, 138, 168}
%\definecolor{background}{RGB}{240, 247, 255}
\usepackage[
	audience=english
%	audience=spanish,
%	audience=german
]{beameraudience}

\usepackage{tikz}
\newcommand{\progressbar}{%
	\pgfmathsetmacro{\theta}{360/\inserttotalframenumber*\insertframenumber}
	\begin{tikzpicture}[scale=0.035]
	\fill[yellow] (0,0) circle (9);
	\fill[red] (0,0) -- (9,0) arc (0:-\theta:9);
	\fill[white] (0,0) circle (5);
	\node at (0,0) {\insertframenumber};
	\end{tikzpicture}
}

\setbeamertemplate{footline}{\hfill \progressbar}

\usepackage{textpos}
%\useoutertheme{splitwithminiframes}
%\useoutertheme{sidebarwithminiframes}
%\usecolortheme[cautious]{owl}
\begin{document}

\begin{frame}
	\frametitle{Common title, appearing on all slides in one frame}
	%------------------------------------------------------------ 1
	\only<1>{
		\framesubtitle{The first frame subtitle}
		\begin{itemize}
			\item some text on slide 1
	\end{itemize}}
	%------------------------------------------------------------ 2
	\only<2>{
		\framesubtitle{The second frame subtitle}
		\begin{itemize}
			\item some text on slide 2
	\end{itemize}}
\end{frame}
\frame{\titlepage}

\section{Introduction}
\frame{
	\frametitle{Equation}
	
	$$y=x^2$$
} 

\frame{text}

\frame{text}

\section{Experimental}
\frame{
	Content...
}

\frame{text}

\frame{text} \frame{text} \frame{text} 

\appendix  
\frame[noframenumbering,plain]{text} 

\setbeamertemplate{footline}
{%
	\hbox{%
		\begin{beamercolorbox}[wd=.3\paperwidth,ht=7.8pt,dp=3pt,center]{author in head/foot}\insertshortauthor
		\end{beamercolorbox}%
		\begin{beamercolorbox}[wd=.4\paperwidth,ht=7.8pt,dp=3pt,center]{title in head/foot}\insertshorttitle
		\end{beamercolorbox}%
		\begin{beamercolorbox}[wd=.3\paperwidth,ht=7.8pt,dp=3pt,center]{date in    head/foot}\insertshortdate\hspace*{5mm}%	
		\end{beamercolorbox}  
	}% 
} 

\addtobeamertemplate{footline}{}{%
	\begin{textblock*}{100mm}(0.94\textwidth,-7mm)
		\progressbar
	\end{textblock*}
}


\begin{frame}{Title}
 all version
\end{frame}

\justfor{english}{
    \begin{frame}
    English Slide
    \pause
    tba
    \end{frame}
}

\justfor{spanish}{
    \begin{frame}
    Spanish Slide
    \pause
    tba
    \end{frame}
}

\justfor{german}{
    \begin{frame}
    Deutcher Text
    \pause
    wird noch angekündigt
    \end{frame}
}

\end{document}