\section{Ecuaciones de recurrencia}
\subsection{Ejemplos definidos por ecuaciones de recurrencia}

\begin{frame}
\frametitle{\secname}
\framesubtitle{\subsecname}

\textit{Resolver una ecuación de recurrencia} significa encontrar una sucesión que satisfaga las ecuaciones de recurrencia.

\

Encontrar una ``solución general'' significa encontrar una fórmula que describe todas las soluciones posibles (todas las sucesiones posibles que satisfacen la ecuación).

\

Veamos el siguiente ejemplo:

\begin{claim}
	Considere que $T_{n}$ satisface la siguiente ecuación
	\begin{equation}
	T_{n}=2T_{n-1}+1\quad\forall n\in\mathds{N}\setminus\left\{1\right\}.
	\end{equation}
\end{claim}

\end{frame}

\begin{frame}
%\frametitle{\secname}
%\framesubtitle{\subsecname}

\begin{example}[Desajustes]
	Imagine una fiesta donde las parejas llegan juntas, pero al final de la noche, cada persona se va con una nueva pareja. Para cada $n\in\mathds{N}$, digamos que $D_{n}$ es el número de diferentes formas en que las parejas pueden ser ``trastornadas'', es decir, reorganizadas en parejas, por lo que ni uno está emparejado con la persona con la que llegaron.
\end{example}

\begin{claim}[$D_{n}$ para cualquier valor de $n$]
	Para todo $n\geq4$ tendremos: \[ D_{n}=(n-1)\left\{D_{n-2}+D_{n-1}\right\}. \] Y la sucesión definida en $\mathds{N}$ es \[ S_{n} = A \times n!. \]
\end{claim}

\end{frame}

\begin{frame}
%\frametitle{\secname}
%\framesubtitle{\subsecname}

\begin{theorem}[Desigualdad para la acotación de $D_{n}$]
Para todo $n\geq2$ tenemos: \[ \left(\frac{1}{3}\right)n!\leq D_{n}\leq\left(\frac{1}{2}\right)n!. \]
\end{theorem}

\

La mejor fórmula para $D_{n}$ que sabemos utiliza la función de ``entero más cercano''.

\

\begin{definition}
	Para cualquier número real $x$, se define \textbf{el entero más cercano a} $x$, $\lceil x\rfloor$:

	Si $x$ es escrito como $n+f$ donde $n$ es el entero $\lfloor x\rfloor$, y $f$ es una fracción donde $0\leq f<1$:

	\begin{itemize}
		\item Si $0\leq f<\frac{1}{2}$, entonces $\lceil x\rfloor=n$.
		\item Si $\frac{1}{2}\leq f<1$, entonces $\lceil x\rfloor=n+1$.
	\end{itemize}
\end{definition}

\

\begin{remark}
	Entonces $D_{n}=\lceil(n!)/e\rfloor$ cuando $e=2.71828182844\ldots$ es la base del logaritmo natural.

	$(n!)/e$  nunca es igual a $\lceil(n!)/e\rfloor+\frac{1}{2}$.
\end{remark}
\end{frame}

\subsubsection{Número de Ackermann}

\begin{frame}
%\frametitle{\secname}
%\framesubtitle{\subsecname}
Por los 1920's, el lógico y matemático alemán, discípulo de David Hilbert, Wilhelm Ackermann (1896–1962), inventó una función muy curiosa.

\begin{minipage}{0.45\paperwidth}
	\begin{definition}[Ackermann]
		Sea $A\colon\mathds{N}\times\mathds{N}\rightarrow\mathds{N}$ una función. Se define recursivamente usando tres reglas:
		\begin{enumerate}
			\item $A(1,n)=2$, $\forall n\geq 1$.
			\item $A(m,1)=2m$, $\forall m\geq 2$.
			\item Cuando $m>1$ y $n>1$ se tiene: $A\left(m,n\right)=A\left(A(m-1,n),n-1\right)$.
		\end{enumerate}
	\end{definition}
	\begin{remark}
		Entonces $A\left(2,n\right)=4,\quad\forall n\geq1$. Además $A\left(m,2\right)=2^{m},\quad\forall m\geq 1$. Seguidamente se puede continuar a calcular $A\left(m,3\right)=2\uparrow m$ con la función torre definida por $2\uparrow\left[k+1\right]=2^{2\uparrow k}$ con valor inicial $2\uparrow 1=2$, por P.I.M.
	\end{remark}
\end{minipage}
\hfill
\begin{minipage}{0.45\paperwidth}
	\begin{listing}[H]
		\inputminted{python}{./code/ackermann.py}
		\caption{Programa \texttt{ackermann.py}}
	\end{listing}
\end{minipage}

\end{frame}