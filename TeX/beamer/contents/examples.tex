\begin{frame}
\frametitle{Ejemplos Definidos Por Ecuaciones de Recurrencia.}
\begin{columns}
\begin{column}{\textwidth}
\textit{Resolver una ecuación de recurrencia} significa encontrar una secuencia que satisfaga las ecuación de recurrencia. Encontrar una ``solución general'' significa encontrar una fórmula que describe todas las soluciones posibles (todas las secuencias posibles que satisfacen la ecuación).\\
Veamos el siguiente ejemplo:
\begin{block}{Solución al Ejemplo.}
\begin{itemize}
	\item Considere $T_{n}$ que satisface la siguiente ecuación para todo $n \in\mathds{N}\setminus\left\{1\right\}$: \[ T_{n}=2T_{n-1}+1. \]
\end{itemize}
\end{block}
\end{column}
\end{columns}
\end{frame}

\begin{frame}
\frametitle{Ejemplo 1. [Desajustes]}
Imagina una fiesta donde las parejas llegan juntas, pero al final de la noche, cada persona se va con una nueva pareja. Para cada $n\in P$, digamos que $D_{n}$ es el número de diferentes formas en que las parejas pueden ser ``trastornadas'', es decir, reorganizadas en parejas, por lo que ni uno está emparejado con la persona con la que llegaron.

\begin{columns}
\begin{column}{\textwidth}
\begin{block}{$D_{n}$ para cualquier valor de $n$}
\begin{itemize}
\item Para todo $n \geq 4$ tendremos: \[ D_{n}=(n-1)\left\{D_{n-2}+D_{n-1}\right\}. \] Y la sucesión definida en $P$ es \[ S_{n} = A \times n!. \]
\end{itemize}
\end{block}
\end{column}
\end{columns}
\end{frame}

\begin{frame}
\frametitle{Teorema Acotación para $D_{n}$}
\begin{columns}
 \begin{column}{\textwidth}
\begin{block}{Desigualdad para la acotación de $D_{n}$}
\begin{itemize}
\item Para todo $n \geq 2$ tenemos: \[ \frac{1}{3})n! \leq D_{n} \leq (\frac{1}{2})n!. \]
\end{itemize}
\end{block}
 \end{column}
\end{columns}
La mejor fórmula para $D_{n}$ que sabemos utiliza la función de ``entero más cercano''. Para cualquier número real $x$, sea $\lceil x\rfloor$ que denote \textbf{el entero más cercano a} $x$, definido:

Si $x$ es escrito como $n+f$ donde $n$ es el entero $\lfloor x\rfloor$, y $f$ es una fracción donde $0\leq f< 1$:

Si $0 \leq f < \frac{1}{2}$, entonces $\lceil x \rfloor = n$.

Si $\frac{1}{2}\leq f<1$, entonces $\lceil x \rfloor = n+1$.

Entonces $D_{n}=\lceil(n!)/e\rfloor$ cuando $e=2.71828182844$ es la base del logaritmo natural. $(n!)/e$  nunca es igual a $\lceil(n!)/e\rfloor+\frac{1}{2}.$
\end{frame}

\begin{frame}
\frametitle{Ejemplo 2.[Números de Ackermann.]}
Por los 1920s, un lógico y matemático alemán, Wilhelm Ackermann (1896–1962), inventó una función muy curiosa.
\begin{columns}
\begin{column}{\textwidth}
\begin{block}{Akckerman}
\begin{itemize}
\item Sea $A\colon P\times P\rightarrow P$, se define recursivamente usando tres reglas:
\begin{enumerate}
	\item $A(1,n)=2$ para $n=1,2,\ldots$,
	\item $A(m,1)=2m$ para $m=2,3,\ldots$,
	\item Cuando $m>1$ y $n>1$ se tiene: $A\left(m,n\right)=A\left(A(m-1,n),n-1\right)$.
\end{enumerate}
\end{itemize}
\end{block}
 \end{column}
\end{columns}
Entonces $A(2,n)=4,\forall n\geq 1$. Además $A(m,2)=2^{m},\forall m\geq 1$. Seguidamente se puede continuar a calcular $A\left(m,3\right)=2\uparrow m$ con la función torre definida por $2\uparrow[k+1]=2^{2\uparrow k}$ con valor inicial $2\uparrow 1=2$, por PIM.
\end{frame}