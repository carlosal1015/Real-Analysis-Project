\begin{frame}
	\frametitle{\subsecname}
	Aquí es conveniente representar cualquier sucesión de números reales $(a_{n})_{n} $ como la función $f\colon\mathds{N}\rightarrow\mathds{R}$ definido por: \[ f(n)=a_{n},\quad\forall n\in\mathds{N}. \]
	\begin{definition}
		Una \textbf{ecuación en diferencias} es una expresión de la forma:
		\begin{equation}\label{eq:diffeq}
		G\left(n,f(n),f\left(n+1\right),\ldots,f\left(n+k\right)\right)=0,\quad\forall n\in\mathds{N}.
		\end{equation}
	\end{definition}
	El \textbf{orden} de una ecuación en diferencias se halla mediante la diferencia entre los ``términos mayor'' y ``menor'' respectivamente. En~(2), es \alert{$n+k-n=k$}.%\eqref{eq:diffeq}
	\begin{example}
		\begin{itemize}
			\item El \alert{orden} de $f\left(n+3\right)-f\left(n+1\right)-5f(n)=n$ es \alert{$3$}.
			\item El \alert{orden} de $f\left(n+3\right)-f\left(n+1\right)=n^{2}-3$ es \alert{$2$}.
		\end{itemize}
	\end{example}
\end{frame}


\begin{frame}
	\begin{definition}
		Una \textbf{ecuación en diferencias} se le llama \textbf{lineal}	  si puede expresarse de la siguiente forma:
		\begin{equation}\label{edl} f\left(n+k\right)+a_{1}(n)f\left(n+k-1\right)+\cdots+a_{k-1}(n)f\left(n+1\right)+a_{k}(n)f\left(n\right)=b\left(n\right), 
		\end{equation}
		donde $a_{k}(n)\neq0$.
	\end{definition}
  \begin{block}{Clasificación}
	\begin{table}[H]
		\centering
	\begin{tabular}{|c|c|}
				\hline
		   \alert{NOMBRE} & \alert{CONDICIÓN}\\
		   \hline
			Homogéneas(E.D.L.H) &si $b(n)=0$.\\
			\hline
			Completas (E.D.L.C) &si $b(n)\neq0$.\\
			\hline
			De coeficientes constantes & $a_{i}(n)=a_{i}$, $\forall i$.\\
			\hline
			De coeficientes no constantes &si $a_{i}(n)\neq a_{i}$ para algún $i$.\\
			\hline
		\end{tabular}
		\end{table}
	\end{block}
\end{frame}
\begin{frame}
%\begin{definition}
	%La \textbf{solución} de~(2) a toda sucesión $\{f\left(0\right),f\left(1\right),\ldots,f(n),\ldots\}$ que la satisfaga, ahora se le llama \emph{solución general} de una E.D al conjunto de todas las soluciones que tendrán tanto parámetros como orden tenga la ecuación. La determinación de estos parámetros, a partir de unas condiciones iniciales, nos proporcionará las distintas soluciones particulares.
%\end{definition}
\begin{definition}
	Sea una \textbf{ecuación en diferencias lineal homogénea} de coeficientes constantes y de orden $ k $,buscaremos soluciones del tipo  $f(n)=r^{n}$,haciendo este cambio y simplificando  tenemos:
	$$
	r^{n}(a_{0}r^{k}+a_{1}r^{k-1}+\cdots+a_{k})=0
	$$
	$$
	\rightarrow (a_{0}r^{k}+a_{1}r^{k-1}+\cdots+a_{k})=0,
	$$
	a la expresión anterior llamaremos \textbf{ecuación característica.}
\end{definition}
\begin{definition}
	Llamamos \textbf{solución} de una E.D. a toda sucesión $\{ f(1), \ldots, f(k) \}$ que la satisfaga.
\end{definition}

\end{frame}
\begin{frame}
\begin{example}
	Sea: \[ f\left(n+1\right)-f\left(n\right)=3 \] una ecuación en diferencias,halle su solución general
	y particular.
\end{example}
\begin{example}
	Hallar la solución de:
	$$
	f(n+2)-f(n+1)+f(n)=0, \; \forall n \in \mathbb{N},
	$$
	$$
	f(0)=0, f(1)=1.
	$$
\end{example}
\begin{example}[E.D.L.C]
	Hallar la solución de:
	$$f(n + 1) - 2f(n) = 2^{n} , \; \forall n \in \mathbb{N},$$
	$$f(0) = -1.$$
\end{example}
%es $f\left(n\right)=3n+c$.


%Si consideramos las condiciones iniciales, por ejemplo, $f(0)=2$, entonces $f(0)=3\times0+c=c$, por tanto $c=2$ y la solución particular es $f_{p}(n)=3n+2$.
%Es decir, la solución es la sucesión $f_{p}(n)=\left\{2,5,8,11,\ldots\right\}$.
%\end{example}
\end{frame}

\begin{frame}
	\begin{theorem}[De la existencia y la unicidad]
		Dada la ecuación (\ref{edl}) y dados $n$ números reales $k_{0}$, $k_{1}$, \ldots, $k_{n-1}$ existe una única solución que verifica \[ f\left(0\right)=k_{0},f\left(1\right)=k_{1},\ldots,f\left(n-1\right)=k_{n-1}. \]
	\end{theorem}

	\begin{theorem}
		Toda combinación lineal de soluciones de una ecuación en diferencias lineal homogénea de orden $n$ es también una solución.
	\end{theorem}
	\begin{theorem}
		Las soluciones de una ecuación en diferencia lineal de orden $n$ forman un espacio vectorial,cuya dimensión del espacio de soluciones de una ecuación en diferencias lineal de orden $k$ es $k$.
	\end{theorem}
\end{frame}

%http://personal.us.es/pnadal/Informacion/leccion5ecdiferencias.pdf
%\begin{frame}
%	\begin{block}{Ecuaciones en diferencias de primer orden}
%	\end{block}
%	
%	\begin{block}{Ecuaciones en diferencias de segundo orden}
%	\end{block}
%\end{frame}


\subsubsection{Número de Catalan}

\begin{frame}{Número de Catalan}
\begin{minipage}{6cm}
	\begin{block}{Triangulación}
	Una triangulación de un polígono es una partición del mismo en triángulos disjuntos cuyos vértices coinciden con los vértices del polígono.
	\end{block}
\end{minipage}
\begin{minipage}{5cm}
\begin{figure}
	\centering
	\includegraphics[scale=0.25]{ca1}	
\end{figure}


\end{minipage}
\begin{minipage}{6cm}
		\begin{figure}
			\centering
			\includegraphics[height=0.3\paperheight]{ca(1)}
		\end{figure}
\end{minipage}
\begin{minipage}{5.5cm}
	\begin{block}{Caminos en rejillas}
		 Un camino monótono es aquél que empieza en la esquina inferior izquierda y termina en la esquina superior derecha, y consiste únicamente en tramos que apuntan hacia arriba o hacia la derecha,sin que pase de diagonal.
	\end{block}
\end{minipage}
\end{frame}