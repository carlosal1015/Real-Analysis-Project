\documentclass[a4paper,12pt]{article}
\usepackage[total={15cm,21cm}]{geometry}
\usepackage[spanish]{babel}

\begin{document}
\section{Introducción}
Uno de los temas más importantes dentro del Análisis Matemático son las sucesiones,
funciones cuyo dominio y contradominio son los números Naturales(N)
y los números Reales(R) respectivamente,en el presente trabajo nos enfocaremos
en nada menos que las ``sucesiones de recurrencia'',donde todo término se
determina en función de los anteriores,un ejemplo famoso es la sucesión de Fibonacci.
Esta sucesión fue descrita por Leonardo de Pisa(Fibonacci) como la
solución a un problema de cría de conejos: ``Cierto hombre tiene una pareja de
conejos juntos en un lugar cerrado y desea saber cuántos son creados a partir
de este par en un año cuando, de acuerdo a su naturaleza, cada pareja necesita
un mes para envejecer y cada mes posterior procrea otra pareja''.\\
Viendo esto,hemos pretendido conocer un poco más sobre este tipo de sucesiones
desde un punto de visto matemático,dando su definición,algunos otros
ejemplos,su relación con las ecuaciones en diferencias y otras aplicaciones como
resolver sistemas de ecuaciones lineales empleando nuestros conocimientos
adqueridos en el curso de Análisis Real de la carrera de Matemática en la Universidad
Nacional de Ingeniería.
\end{document}