\preface
Uno de los temas más importantes dentro del \emph{Análisis Matemático} son las sucesiones, es decir, funciones cuyo dominio y contradominio es el conjunto de los números naturales $\mathds{N}$ y el de los números reales $\mathds{R}$, respectivamente. En el presente trabajo nos enfocaremos en nada menos que las ``relaciones de recurrencia'', donde cualquier  término se determina en función de al menos uno de los términos precedentes, en el célebre libro \cite[ver pág. 404]{sigler2003fibonacci} de \emph{Leonardo de Pisa}\footnote{Fibonacci} se da la solución al siguiente problema de cría de conejos:
\begin{quote}
	``Cierta persona cría una pareja de conejos juntos en un lugar cerrado y desea saber cuántos nacimientos durante un año han acontecido a partir del par inicial, de acuerdo a su naturaleza, cada pareja necesita un mes para envejecer y cada mes posterior procrea otra pareja''.
\end{quote}
Este ejemplo famoso es conocido como la \emph{sucesión de Fibonacci}. Viendo esto, hemos concebido un modelo matemático basado en sucesiones recursivas, dando su definición, algunos otros ejemplos, su relación con las ecuaciones en diferencias y otras aplicaciones como resolver sistemas de ecuaciones lineales empleando nuestros conocimientos adquiridos en el curso de Análisis Real de la carrera de Matemática en la Universidad
Nacional de Ingeniería.
\vspace{\baselineskip}
\begin{flushright}\noindent
Rímac,\hfill {\it Carlos Aznarán Laos}\\
junio 2019\hfill {\it Franss Cruz Ordoñez}\\
\end{flushright}