\documentclass[12pt]{beamer}
\usetheme{Warsaw}
\usepackage[utf8]{inputenc}
\usepackage{amsmath}
\usepackage{amsfonts}
\usepackage{amssymb}
\usepackage[utf8]{inputenc}
\usepackage[spanish]{babel}
\usepackage{multicol}
\usepackage{graphicx}
%\usepackage[pdftex]{color}
\newcommand{\ds}{\displaystyle}
\newtheorem{teo}{Teorema}{}
\author{Micha Velasque Junior.}
\title{Sucesiones y Series de Recurrencia.}
%\setbeamercovered{transparent} 
%\setbeamertemplate{navigation symbols}{} 
%\logo{} 
\institute{Universidad Nacional de Ingeniería.} 
\date{21 de junior del 2019} 
%\subject{}
\begin{document}

\begin{frame}
\titlepage
\end{frame}

\begin{frame}
    \frametitle{Ejemplos Definidos Por Ecuaciones de Recurrencia.}
\begin{columns}
 \begin{column}{\textwidth}
\textit{Resolver una ecuación de recurrencia} significa encontrar una secuencia que satisfaga las ecuación de recurrencia. Encontrar una "solución general" significa encontrar una fórmula que describe todas las soluciones posibles (todas las secuencias posibles que satisfacen la ecuación).\\
Veamos el siguiente ejemplo:\\
\begin{block}{Solución al Ejemplo.}
\begin{itemize}[<+->]
\item Considere $T_{n}$ que satisface la siguiente ecuación para todo $n \in \mathbb{N}$; $n>1$ : 

$$T_{n} = 2T_{n-1} + 1 $$ 

\end{itemize}
\end{block}
 \end{column} \ \
\end{columns}
\end{frame}

\begin{frame}
\frametitle{Ejemplo 1. [Desajustes]}
Imagina una fiesta donde las parejas llegan juntas, pero al final de la noche, cada persona se va con una nueva pareja. Para cada $n \in P$, digamos que $D_{n}$ es el número de diferentes formas en que las parejas pueden ser "trastornadas", es decir, reorganizadas en parejas, por lo que ni uno está emparejado con la persona con la que llegaron.\\
\begin{columns}
 \begin{column}{\textwidth}
\begin{block}{$D_{n}$ para cualquier valor de $n$}
\begin{itemize}[<+->]
\item Para todo $n \geq 4$ tendremos:\\
$$D_{n} =(n-1)\{D_{n-2} + D_{n-1}\}.$$
y la suceción definida en $P$ es 
$$S_{n} = A \times n!.$$

\end{itemize}
\end{block}
 \end{column} \ \
\end{columns}
\end{frame}

\begin{frame}
\frametitle{Teorema[Acotación para $D_{n}$]}
\begin{columns}
 \begin{column}{\textwidth}
\begin{block}{Desigualdad para la acotación de $D_{n}$}
\begin{itemize}[<+->]
\item Para todo $n \geq 2$ tenemos:\\
$$\ds(\frac{1}{3})n! \leq D_{n} \leq (\frac{1}{2})n!.$$

\end{itemize}
\end{block}
 \end{column} \ \
\end{columns}
La mejor fórmula para $\bf{D_{n}}$ que sabemos utiliza la función de `` entero más cercano". Para cualquier número real $x$, sea $\lceil x \rfloor$ que denote \textit{\textbf{el entero más cercano a}} $x$, definido:\\
Si $x$ es escrito como \textit{\textbf{n+f}} donde \textit{\textbf{n}} es el entero $\lfloor x \rfloor$, y \textit{\textbf{f}} es una fracción donde $0 \leq f < 1 :$\\

si $0 \leq f < \frac{1}{2}$ \hspace{1cm} entonces $\lceil x \rfloor = n;$\\

si $\frac{1}{2} \leq f < 1$ \hspace{1cm} entonces $\lceil x \rfloor = n+1.$\\
Entonces $\bf{D_{n}}$ = $\lceil(n!)/e\rfloor$ cuando $e = 2.71828182844...$ es la base del logaritmo natural.  // $(n!)/e$  nunca es igual a $~\lceil(n!)/e\rfloor + \frac{1}{2}.$
\end{frame}

\begin{frame}
\frametitle{Ejemplo 2.[Números de Ackermann.]}
Por los 1920s, un lógico y matemático alemán, Wilhelm Ackermann
(1896–1962), inventó una función muy curiosa.
\begin{columns}
 \begin{column}{\textwidth}
\begin{block}{Akckerman}
\begin{itemize}[<+->]
\item Sea $A: P \times P \rightarrow P,$ se define recursivamente usando tres reglas:\\
Regla 1.\hspace*{0.2cm} $\ds A(1,n)=2$ \hspace*{0.3cm} para $n = 1,2,...,$\\
Regla 2.\hspace*{0.2cm} $\ds A(m,1)=2m$ \hspace*{0.3cm} para $m =2,3,...,$\\
Regla 3.\hspace*{0.2cm} cuando $m>1$ y $n>1$ se tiene: $\ds A(m,n) = A(A(m-1,n),n-1).$

\end{itemize}
\end{block}
 \end{column} \ \
\end{columns} 
Entonces $A(2,n) = 4, ~~\forall n \geq 1.$ Ademas $A(m,2) = 2 ^{m},~~ \forall m \geq 1.$ Seguidamente se puede continuar a calcular $A(m,3) = 2 \uparrow m$ ; con la función torre definida por $2\uparrow[k+1] = 2^{2 \uparrow k}. $ con valor inicial $2 \uparrow 1 = 2$, por PIM.
\end{frame}

\begin{frame}
\frametitle{Teoremas Previos a la Solución de E.R.}
\begin{columns}
 \begin{column}{\textwidth}
\begin{block}{3 Teoremas para la Solución de Ecuaciónes de Recurrencia de 1er y 2do Orden.}
\begin{itemize}
\item Si $S$ es suceción aritmética condiferencia común b, entonces $\forall n \in \mathbb{N},~S_{n} = I + nb$, donde $I = S_{0}.$
\item Si $S$ es la sucesión geométrica con razón común $r$, entonces $\forall n \in \mathbb{N},~ S_{n} = r^{n} \times I$ cuando $I = S_{0}.$
\item Si $r \neq 1,$ entonces $\forall n \in \mathbb{N},~ I + rI + rI^{2}+...+r^{n}I = \ds\frac{r^{n+1}-1}{r-1} \times I.$ 
\end{itemize}
\end{block}
 \end{column} \ \
\end{columns} 
\end{frame}




\end{document}