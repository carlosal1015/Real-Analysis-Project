\documentclass[a4,paper]{article}
\usepackage[total={14cm,22cm}]{geometry}
\usepackage[spanish]{babel}
\usepackage{graphicx}
\usepackage[utf8]{inputenc}
\usepackage{amsmath}
\usepackage{here}
\usepackage{amsthm}
\usepackage{amssymb}
\usepackage{enumerate}
\newtheorem{teor}{Teorema}[section]
\newtheorem{lema}{Lema}[section]
\newtheorem{defi}{Definición}[section]
\newtheorem{ejem}{Ejemplo}[section]
\newtheorem{obs}{Observación}[section]
\newtheorem{prop}{Proposición}[section]
\newtheorem{cs}{caso}
\renewcommand\qedsymbol{$ \blacksquare $}
\begin{document}
	\section{Operadores de cambio y diferencia}
	Aquí es conveniente representar cualquier secuencia de números reales $(a_{k}) _{k \in \mathbb{N}} $ como la función	$f: \mathbb{N} \to  \mathbb{R}$ definido por:
	$$
	f(k)=a_{k}, \; \forall k \in \mathbb{N}.
	$$
	Dadas dos funciones $ f,g: \mathbb{N} \to \mathbb{R} $ y $ r \in \mathbb{R} $ consideremos las funciones:
	$$
	(f+g)(k)=f(k)+g(k), \; \text{and} \; (rf)(k)=rf(k) \; \forall k \in \mathbb{N}.
	$$
	
	Dotado de estas operaciones, el conjunto de funciones de $ \mathbb{N} \to \mathbb{R} $ es un $ \mathbb{R}- $ espacio.También consideraremos la función:
	$$
	(fg)(k)=f(k)g(k) \; \forall k \in \mathbb{N}.
	$$
	
	Un mapa lineal del espacio de funciones de $ \mathbb{N} $ a $ \mathbb{R} $ en sí mismo es un operador.
	\begin{defi}
		Consideramos el espacio de funciones de $ \mathbb{N} \to \mathbb{R} $. Para cada función $ f:\mathbb{N} \to \mathbb{R} $ el operador identidad y el operador de cambio $ \theta $
 están definidos: 
 $$
	\mathbb{I}(f)=f \; \text{y} \; \theta(f)(k)=f(k+1) \; \forall k \in \mathbb{N}
 $$
 
 Uno verifica inmediatamente que la identidad y el operador de cambio son de hecho lineales.
	\end{defi}
\begin{prop}
	(Linealidad de la identidad y el cambio de operador).\\
	Sea $ f,g:\mathbb{N} \to \mathbb{N} \; \text{y} \; c \in \mathbb{R}$.Luego tenemos:
	\begin{enumerate}
		\item $ \mathbb{I}(f+g)(k) = \mathbb{I}(f)(k)+ \mathbb{I}(g)(k) \text{y} $
		\item $ \mathbb{I}(cf)(k)=c\mathbb{I}(f)(k) \; \text{y} \; \theta(cf)(k)=c\theta(f)(k) $
	\end{enumerate}
	\begin{proof}
		Sea $ k \in \mathbb{N},$ luego:
		$$
		\theta(f+g)(k)=(f+g)(k+1)=f(k+1)+g(k+1)=\theta(f)(k)+\theta(g)(k).
		$$
		$$
		\theta(cf)(k)=(cf)(k+1)=cf(k+1)=c\theta(f)(k).
		$$
		Se verifica la linealidad de $ \mathbb{I} $ inmediatamente.
	\end{proof}
\end{prop}
Para cualquier operador $ T $, será conveniente un ligero abuso de notación,
para escribir $ T f (k)$ en lugar de $T (f) (k)$. Además en algunos casos, por ejemplo cuando $f$
depende de otros parámetros, uno escribe $T_{k} f (k)$ en lugar de $T f (k)$ para evitar la ambigüedad.\\
Así, por ejemplo, denotada por $\mathbb{I}_{\mathbb{N}}: N \to N$ la función definida por $\mathbb{I}_{\mathbb{N}} (k) = k$ para cada $ k \in \mathbb{N} $ escribiremos $\theta k = k + 1 $en lugar de $\theta (\mathbb{I}_{\mathbb{N}}) (k) = k + 1.$ Análogamente $\theta k^{2} = (k + 1)^{2}$, $\theta_{k} \; k^{a} = (k + 1)^{a} \; \text{y} \; \theta_{k} \; a^{k} = a^{k+1}$ para cada $ a \in \mathbb{R}. $\\
Para las funciones de valor real de una variable de número natural ahora introducimos el
análogo de la derivada habitual para funciones de valor real de una variable real:

	\begin{defi}
		El operador diferencia es el operador $ \bigtriangleup $ que a cada función $ f:\mathbb{N} \to \mathbb{R} $ asigna la función $\bigtriangleup f:\mathbb{N} \to \mathbb{R},$ definido de la siguiente manera:
		$$
		\bigtriangleup f(k)=f(k+1)-f(k), \; \forall k \in \mathbb{N}.
		$$ 
	\end{defi}
	\begin{obs}
		Usando el operador de cambio, uno tiene $ \bigtriangleup=\theta - \mathbb{I};$ es decir:
		$$
		\bigtriangleup f= \theta f-f, \; \forall f:\mathbb{N} \to \mathbb{R}.
		$$
	\end{obs}
	\begin{defi}
		Claramente, para cada función $ f: \mathbb{N} \to \mathbb{R},$ uno tiene:
		$$
		\bigtriangleup f(k)=\frac{f(k+1)-f(k)}{1},
		$$
	\end{defi}
	\noindent y entonces $ \bigtriangleup f:\mathbb{N} \to \mathbb{R} $ es una función que mide el cociente de diferencia de $ f $ sobre
	el intervalo más pequeño posible de números naturales, es decir, un intervalo de longitud uno.
	En este sentido, el operador diferencia constituye el análogo discreto de la noción de derivada para funciones de una variable real. En lo que sigue, el lector tendrá
	ocasión para anotar analogías y contrastes entre estas dos nociones.\\
	Al igual que la derivada, el operador de diferencia es lineal: de hecho, es una diferencia
	de dos operadores lineales.
	\begin{prop}
		(Linealidad de la diferencia).\\Sea $ f,g:\mathbb{N} \to \mathbb{R} \; \text{y} \; c \in \mathbb{R}.$Luego uno tiene:
		\begin{enumerate}[1.]
			\item $\bigtriangleup (f+g)= \bigtriangleup f + \bigtriangleup g$
			\item $ \bigtriangleup(cf)=c \bigtriangleup f $
		\end{enumerate}
	\end{prop}
			\begin{proof}
				 Como $ \bigtriangleup=\theta - \mathbb{I}$ se obtiene:
			\begin{enumerate}[1.]
				\item
					$\bigtriangleup(f+g)=(\theta-\mathbb{I} )(f+g)=\theta (f+g)-\mathbb{I}(f+g)=\theta(f)-f+\theta(g)-g=\bigtriangleup(f)-\bigtriangleup(g).$
				\item $\bigtriangleup(cf)=(\theta-\mathbb{I})(cf)=\theta(cf)-\mathbb{I}(cf)=c\;\theta(f)-cf=c(\theta -\mathbb{I})(f)=c \bigtriangleup(f).$
			\end{enumerate}
			\end{proof}
			Ahora vemos cómo el operador de diferencia actúa en algunas funciones simples condominio  $ \mathbb{N} $.
			\begin{ejem}
				\begin{enumerate}\\
				\vspace{2mm}
				\item Funciones constantes: Al igual que en el caso de la derivada de una  constante.
				Funciona con dominios en $ \mathbb{R} $, aquí también tenemos que la diferencia de
				una función constante (con dominio $ \mathbb{N} $) es igual a la función cero: de hecho, si
				$f(k) = c \in R $por cada $ k \in \mathbb{N}$, entonces
				$$
				\bigtriangleup f(k)=f(k+1)-f(k)=c-c=0 
				$$
				\item 
				Función de identidad en los números naturales: al igual que en el caso continuo, el diferencia de la función de identidad $I_{\mathbb{N}}: \mathbb{N} \to \mathbb{N} es la función constante k \to  1$ para todo $ k\in \mathbb{N}:$de hecho
				$$
				\bigtriangleup I_{\mathbb{N}}(k)=I_{\mathbb{N}}(k+1)-1= k+1-k=1.
				$$
				\end{enumerate}
			\newpage
			\end{ejem}
			\begin{ejem}
			Los operadores de cambio y diferencia conmutan. Más explícitamente, uno tiene
			$$
			\bigtriangleup \circ \theta = \theta \circ \bigtriangleup
			$$
			\begin{proof}
			De hecho, para cada $ k \in \mathbb{N} $ y cada función $f: \mathbb{N} \to \mathbb{R} $ uno tiene
			$$
			\bigtriangleup (\theta f) (k) = \theta f (k + 1) - \theta f (k) = f (k + 2) - f (k + 1),
			$$
			mientras
			$$
			\theta ( \bigtriangleup f)(k)=\bigtriangleup f(k+1)=f(k+2)-f(k+1).
			$$
			Por lo tanto uno tiene $ \bigtriangleup(\theta f)=\theta(\bigtriangleup f)(k)  $
			\end{proof}
			La fórmula para la diferencia de un producto se parece a la del derivado de un producto, excepto la introducción del operador de turno:
			\end{ejem}
		\begin{prop}
		(Diferencia de un producto) If $ f,g:\mathbb{N} \to \mathbb{R},$ luego 
		$$
		\bigtriangleup (fg)=\bigtriangleup f \; \theta \; g + f \bigtriangleup g.
		$$
		\end{prop}
	\begin{obs}
		Cabe destacar el hecho evidente de que a pesar de la aparente falta de simetría, uno tiene $\bigtriangleup (fg) = \bigtriangleup (gf).$
	\end{obs}
		\noindent \emph{Demostración}
		\begin{table}[h]
		\begin{tabular}{ccl}
		$\bigtriangleup (f(k)g(k))$& $=$& $f(k+1)g(k+1)-f(k)g(k)$\\
		 & $=$&$ f(k+1)g(k+1)-f(k)g(k+1)+f(k)(k+1)-f(k)g(k)$\\
		 & $=$&$ (f(k+1)-f(k))g(k+1)+f(k)g(k+1)-g(k) $\\
		 & $=$&$ \bigtriangleup f(k) \theta g(k)+f(k)\bigtriangleup g(k)$\\
		\end{tabular}\\
	
	\hfill{ $ \blacksquare $}
		\end{table}
	
	
	
	
\end{document}