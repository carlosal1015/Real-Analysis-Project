\textit{Resolver una ecuación de recurrencia} significa encontrar una sucesión que satisfaga las ecuaciones de recurrencias. Encontrar una ``solución general'' significa hallar una fórmula que describa todas las soluciones posibles (todas las sucesiones posibles que satisfacen la ecuación). Veamos el siguiente ejemplo:

\begin{example}
	Considere que $T_{n}$ satisface la siguiente ecuación para todo $n\in\mathds{N}$, $n>1$:
	\begin{equation*}
	T_{n}=2T_{n-1}+1.
	\end{equation*}
	La ecuación de recurrencia $T_{n}$ indica cómo continúa la sucesión pero no nos dice como empieza tal. % TODO: Crear una tabla de valores.
	\begin{itemize}
		\item Si $T_{1}=1$, se tiene $T=\left(1,7,3,15,31,\ldots\right)$.
		\item Si $T_{2}=1$, se tiene $T=\left(2,5,11,23,47,\ldots\right)$.
		\item Si $T_{4}=1$, se tiene $T=\left(4,9,19,39,79,\ldots\right)$.
		\item Si $T_{-1}=1$, se tiene $T=\left(-1,-1,-1,1-1,-1,\ldots\right)$.
	\end{itemize}
	?`Existe alguna fórmula para cada una de estas sucesiones? ?`Existe una fórmula en términos de $n$ y $T_{1}$ que describa todos los términos de la sucesión? ?`Existe una posible solución para $T_{n}$? Para poder responder este tipo de problemas, veamos un poco más de ecuaciones con recurrencia.
\end{example}

\section{Ejemplos definidos por ecuaciones de recurrencia}

\begin{example}[Parejas desordenadas]
	Imagina una fiesta donde las parejas llegan juntas, pero al final de la noche, cada persona se va con una nueva pareja. Para cada $n\in P$, digamos que $D_{n}$ es el número de diferentes formas en que las parejas pueden ser ``desordenadas'', es decir, reorganizadas en parejas, por lo que ni uno está emparejado con la persona con la que llegaron.

	Para los valores:
	\begin{itemize}
		\item $D_{1}=0$, una pareja no puede ser desordenada.
		\item $D_{2} = 1$, existe una y solo una manera de ``desordenar'' una pareja.
		\item $D_{3} = 2$, si las parejas llegan como $Aa$, $Bb$, $Cc$, entonces $A$ estaría emparejado con $b$ o $c$. Si $A$ esta emparejado con $b$, $C$ debe estar emparejado con $a$(y no $c$) y $B$ con $c$. Si $A$ esta emparejado con $c$, $B$ no debe estar emparejado con $a$(y no $b$) y $C$ con $b$.
	\end{itemize}
	?`Qué tan grandes son $D_{4}$, $D_{5}$ y $D_{10}$? ?`Cómo podemos calcularlos? ?`Existe alguna expresión cerrada para obtener todos los términos de la? % TODO: Sucesión.

	Vamos a desarrollar una estrategia para contar los desajustes cuando $n\leq4$. Supongamos que hay $n$ mujeres $A_{1},A_{2},A_{3},\ldots,A_{n}$, y cada $A_{j}$ llega con el hombre $a_{j}$.

	La mujer $A_{1}$ puede ser ``re-emparejada'' con cualquiera de los $n-1$ hombres restantes $a_{2}$ o $a_{3}$ o \ldots o $a_{n}$. Digamos que está emparejada con $a_{k}$, donde $2\leq k\leq n$ y ahora consideremos $a_{k}^{\prime}$ pareja original de la mujer $A_{k}$: ella podría tomar $a_{1}$ o ella podría rechazar $a_{1}$ y tomar a alguien más.

	Si $A_{1}$ es pareja con $a_{k}$ y $A_{k}$ es pareja con $a_{1}$, entonces $n-2$ parejas dejaron para desordenar, y eso puede hacerse exactamente de $D_{n-2}$ maneras diferentes.
	% TODO: Cambiar hacerse.

	Ahora para cada uno de los $n-1$ hombres que $A_{1}$ podría elegir, hay $\{D_{n-2}+ D_{n-1}\}$ diferentes formas de completar el trastorno. Por lo tanto, cuando $n\geq 4$ tenemos:
\begin{equation}\label{eq:1_1}
D_{n}=\left(n-1\right)\left\{D_{n-2}+D_{n-1}\right\}
\end{equation}
Usando la ecuación \eqref{eq:1_1} las evaluaciones para $n=1$ y $n=2$ verifican la igualdad, ahora evaluemos $D_{n}$ para cualquier valor de $n$, con $n\in\mathds{N}$
\begin{align*}
D_{3}&=\left(3-1\right)\left\{D_{2}+D_{1}\right\}=2\left(1+8\right)=2.\\
D_{4}&=\left(4-1\right)\left\{D_{3}+D_{2}\right\}=3\left(2+1\right)=9.\\
D_{5}&=\left(5-1\right)\left\{D_{4}+D_{3}\right\}=4\left(9+2\right)=44.\\
D_{6}&=(6-1)\left\{D_{5}+D_{4}\right\}=5\left(44 + 9\right)=265.\\
D_{7}&=\left(7-1\right)\left\{D_{6}+D_{5}\right\}=6\left(265+44\right)=1854.\\
D_{8}&=\left(8-1\right)\left\{D_{7}+D_{6}\right\}=7\left(1854+265\right)=14833.\\
D_{9}&=\left(9-1\right)\left\{D_{8}+D_{7}\right\}=8\left(14833+1854\right)=133496.\\
D_{10}&=\left(10-1\right)\left\{D_{9}+D_{8}\right\}=9\left(133496+14833\right)=1334961.
\end{align*}
La sucesión en $P$ definido por $S_{n}=A\times n!$ donde $A$ es un número real satisface la ecuación de recurrencia \eqref{eq:1_1}. Si $n\geq3$ se tiene:
\begin{align*}
	\left(n-1\right)\left\{S_{n-2}+S_{n-1}\right\}
	&=(n-1)\{A(n-2)!+A(n-1)!\} \\
	&=(n-1)A(n-2)!\{1+(n-1)\} \\
	&=A(n-1)(n-2)!\{n\}\\
	&=A\times n!\\
	&=S_{n}.
\end{align*}
?`Es válida la fórmula para $n=1$ o $n=2$? ?`Existe algún número real tal que $D_{n}=A(n!)$ cuando $n=1$ o $n=2$? No, porque si $0=D_{1}=A(1!)$, entonces $A$ debe ser igual a $0$, y si $1=D_{2}=A(2!)$, se tiene que $A$ debería tomar el valor de $\frac{1}{2}$. Sin embargo, podemos usar esta fórmula para probar que $D_{n}$ es acotado.
\end{example}

\begin{theorem}{}
Para todo $n\geq 2$, $\left(\frac{1}{3}\right)n!\leq D_{n}\leq\left(\frac{1}{2}\right)n!$.
\end{theorem}

\begin{proof}
	Primero considere la tabla de valores:
	\begin{table}[ht!]
		\centering
		\begin{tabular}{ >{$}c<{$} >{$}c<{$} >{$}c<{$} >{$}c<{$}}
			n & \left(\frac{1}{3}\right)n! & D_{n} & \left(\frac{1}{2}\right)n! \\
			\hline
			1 & \frac{1}{3} & 0 & \frac{1}{2}\\
			2 & \frac{2}{3} & 1 &  1=\frac{2}{2}\\
			3 & \frac{6}{3}=2 & 2 & 3=\frac{6}{2}\\
			4 & \frac{24}{3}=8 & 9 & 12=\frac{24}{2}\\
			5 & \frac{120}{3}=40 & 44 & 60=\frac{120}{2}\\
			6 & \frac{720}{3}=240 & 265 & 360=\frac{720}{2}\\
		\end{tabular}
	\end{table}
Por inducción fuerte en matemática sobre $n$.
\begin{enumerate}[label={Paso~\arabic*}]
	\item Si $n=2$, se tiene $\left(\frac{1}{3}\right)n!=\frac{2}{3}<1=D_{n}= \left(\frac{1}{2}\right)n!$ y $n=3$, se tiene $\left(\frac{1}{3}\right)n!=\frac{6}{3}=2=D_{n}<3=\left(\frac{1}{2}\right)n!$.
	\item Supongamos que $\exists k\geq3$ tal que si $2\leq n\leq k$, se tiene $\left(\frac{1}{3}\right)n!\leq D_{n}\leq\left(\frac{1}{2}\right)n!$.
	\item Si $n=k+1$, se tiene $n\geq 4$ y $D_{n}=(n-1)[D_{n-2}+D_{n-1}]$ cuando $2\leq n-2<n-1\leq k$. Así, $D_{n}\leq(n-1)\{(1/3)[n-2]!+(1/3)[n-1]!\}=(1/3)n!$, y $D_{n}\leq(n-1)\{(1/2)[n-2]!+(1/2)[n-1]!\}=(1/2)n!$.
\end{enumerate}
La mejor fórmula para $\bf{D_{n}}$ que sabemos utiliza la función de ``entero más cercano''. Para cualquier número real $x$, sea $\lceil x\rfloor$ que denote \emph{el entero más cercano a} $x$ se define de la siguiente manera:

\begin{itemize}
	\item Si $x$ es escrito como $n+f$ donde $n$ es el entero $\lfloor x\rfloor$, y $f$ es una fracción donde $0\leq f< 1$.
	\item Si $0 \leq f < \frac{1}{2}$, entonces $\lceil x\rfloor=n$.
	\item Si $\frac{1}{2}\leq f<1$, entonces $\lceil x\rfloor=n+1$.
\end{itemize}
?`Es $\lceil x\rfloor=\lfloor x+\frac{1}{2}\rfloor$? Así que $\lceil 3.29\rfloor=3$, $\lceil-3.78\rfloor=-4$. Entonces $D_{n}=\lceil(n!)/e\rfloor$ cuando $e=2.71828182844\ldots$ es la base del logaritmo natural. Note que $(n!)/e$ nunca es igual a $\lceil(n!)/e\rfloor+\frac{1}{2}$.

\begin{table}[ht!]
	\centering
	\begin{tabular}{>{$}c<{$} >{$}c<{$} >{$}c<{$}}
		n & D_{n} & (n!)/e \\
		\hline
		1 &  0 & 0.367879441\\
		2 &  1 &  0.735758882\\
		3 &  2 & 2.207276647\\
		4 &  9 & 8.829106588\\
		5 & 44 & 44.14553294\\
		6 & 265 & 264.8731976\\
		7 & 1854 & 1854.112384\\
		8 & 14833 & 14832.89907 \\
		9 & 133496 & 133496.0916\\
		10 &  1334961 & 1334960.916\\
	\end{tabular}
\end{table}

Hay otra fórmula (mucho menos compacta) para $D_{n}$ dado en los ejercicios, junto con un resumen de la prueba de que $D_{n}=\lceil(n!)/e\rfloor$.
\end{proof}

\begin{example}[Números de Ackermann]\index{Ackermann!número}
	En la década de 1920's, el lógico y matemático alemán, Wilhelm Ackermann (1896–1962), inventó una función muy curiosa, $A\colon P\times P\rightarrow P$ que define recursivamente usando ``tres reglas'':
	\begin{enumerate}[label={Regla~\arabic*}]
		\item $A(1,n)=2$ para $n=1,2,\ldots$.
		\item $A(m,1)=2m$ para $m=2,3,\ldots$.
		\item Cuando $m>1$ y $n>1$ se tiene $A(m,n)=A(A(m-1,n),n-1)$.
	\end{enumerate}
\end{example}

Entonces,
\begin{align*}
A(2,2)
&=A(A(2-1,2),2-1)&\text{regla }3\\
&= A(A(1,2),1)&\\
&= A(2,1)&\text{regla }1\\
&= 2(2)&\text{regla }2\\
&= 4.
\end{align*}
Además
\begin{align*}
A(2,3)
&= A(A(2-1,3),3-1)&\text{regla }3\\
&= A(A(1,3),2)\\
&= A(2,2)&\text{regla }1\\
&= 4.
\end{align*}
De hecho, si $A(2,k)= 4$, para algún $k\geq2$, entonces
\begin{align*}
A(2,k+1) &= A(A(2-1,k+1), (k+1)-1)&\text{regla }3\\
&=A(A(1,k+1),k)\\
&=A(2,k)&\text{regla }1\\
&=4.
\end{align*}
Así, tenemos por inducción matemática $A(2,n)=4$, $\forall n\geq1$.

Hasta ahora la tabla de los números de Ackermann se ve así:

\begin{table}[ht!]
	\centering
	\begin{tabular}{>{$}c<{$}| >{$}c<{$} >{$}c<{$} >{$}c<{$} >{$}c<{$} >{$}c<{$} >{$}c<{$} >{$}c<{$} >{$}c<{$} >{$}c<{$}}
		A & n=1 & n=2 & n=3 & n=4 & n=5 & n=6 & n=7 & n=8 & n=9 \\
		\hline
		m=1 & 2 & 2 & 2 & 2 & 2 & 2 & 2 & 2 & 2 \\
		m=2 & 4 & 4 & 4 & 4 & 4 & 4 & 4 & 4 & 4 \\
		m=3 & 6 &  &  &  &  &  &  &  &  \\
		m=4 & 8 &  &  &  &  &  &  &  &  \\
		m=5 & 10 &  &  &  &  &  &  &  &  \\
	\end{tabular}
\end{table}

Observamos que la segunda fila es de puro 4s. ?`Pero cómo es la segunda columna?
Si $A(k,2)=2^{k}$ para algunos $k\geq2$ se tiene
\begin{align*}
	A(k,2)=2^{k}
	&= A(A([k+1],2-1)&\text{regla }3\\
	&= A(A(k,2),1) &\\
	&= A(2^{k},1)&\\
	&= 2(2^{k})&\text{regla }2\\
	&= 2^{k+1}.
\end{align*}
Además,
\begin{align*}
	A(2,3)
	&=A(A(2-1,3),3-1)&\text{regla }3\\
	&=A(A(1,3),2)&\\
	&=A(2,2)&\text{regla }1\\
	&=4.
\end{align*}
Así, se tiene $\forall m\geq1:A(m,2)=2^{m}$. Ahora, ?`como son los otros valores?
\begin{align*}
	A(3,3)
	&= A(A(3-1,3),3-1)&\text{regla }3\\
	&= A(A(2,3), 2)  \\
	&=A(4,2)&\text{segunda fila}\\
	&=2^{4}&\text{segunda columna}\\
	&=16.&\\
\end{align*}

\begin{align*}
	A(4,3)
	&=A(A(4-1,3),3-1)&\text{regla }3\\
	&= A(A(3,3), 2) &\\
	&=A(16,2)&\text{encima}\\
	&=2^{16}&\text{segunda columna}\\
	&=65536.&\\
\end{align*}

\begin{align*}
	A(3,4)
	&= A(A(3-1,3),4-1)&\text{regla }3\\
	&= A(A(2,4), 3) &\\
	&=A(4,3)&\text{segunda fila}\\
	&=65536.&\\
\end{align*}
?`Cual es el valor de $A(4,4)$? ?`Podría ejecutar un programa recursivo simple para evaluar $A(4,4)$?
\begin{align*}
	A(5,3)
	&= A(A(5-1,3),3-1)&\text{regla }3\\
	&= A(A(4,3),2) &\\
	&=A(65536,2) &\\
	&=2^{65536}.&\text{segunda columna}\\
	&=n&\text{grande aproximadamente }20000\text{ dígitos en base }10.\\
\end{align*}
Hasta ahora tenemos:

\begin{table}[ht!]
	\centering
	\begin{tabular}{>{$}c<{$}| >{$}c<{$} >{$}c<{$} >{$}c<{$} >{$}c<{$} >{$}c<{$} >{$}c<{$} >{$}c<{$} >{$}c<{$} >{$}c<{$}}
		A & n=1 & n=2 & n=3 & n=4 & n=5 & n=6 & n=7 & n=8 & n=9 \\
		\hline
		m=1 & 2 & 2 & 2 & 2 & 2 & 2 & 2 & 2 & 2 \\
		m=2 & 4 & 4 & 4 & 4 & 4 & 4 & 4 & 4 & 4 \\
		m=3 & 6 & 8 & 16 & 65536 & \text{?} &  &  &  &  \\
		m=4 & 8 & 16 & 65536  & \text{?} &  &  &  &  &  \\
		m=5 & 10 & 2^{65536} &  &  &  &  &  &  &  \\
	\end{tabular}
\end{table}
?`Cómo continúa la tercera columna? Sea $2\uparrow$ denota el valor de ``Torre'' de k 2's, definida recursivamente por \[ 2\uparrow 1=2,\quad\text{y para}\quad k\geq1,\quad2\uparrow\left[k+1\right]=2^{2\uparrow k}. \]
Pero este es un número tan grande que nunca podría escribirse en dígitos decimales, incluso utilizando todo el papel del mundo, Su valor nunca podría ser calculado. Ahora nos preguntamos ?`Los números Ackermann son ``computables''? Por otro lado, supongamos que las sucesiones que encontramos, incluso aquellas definidas por ecuaciones de recurrencia, serán fáciles para entender y tratar.