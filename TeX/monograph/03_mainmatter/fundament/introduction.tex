\chapter{Introducción}\label{ch:intro}
\abstract{En este capítulo introducimos las relaciones de recurrencia. Estas son \emph{ecuaciones} que definen de \emph{manera recursiva}, a través de funciones adecuadas, los términos aparecen en una sucesión real o compleja. La primera sección trata algunos ejemplos bien conocidos que muestran cómo estas relaciones pueden surgir en la vida real, por ejemplo, el problema de la Torre de Hanói o el problema de Flavio Josefo. Luego, dedicamos una gran parte del capítulo a las ecuaciones en diferencias, es decir, $\Delta x_{n}$, donde $f$ es una función de valor real: en este contexto, los métodos más estudiados son el de Euler y Runge-Kutta. Estudiamos a fondo el caso que pueden ser usados para resolver ecuaciones diferenciales ordinarias. La última parte del capítulo está dedicada al célebre teorema del Polinomio minimal, que afirma que la raíz de un implica la solución: así le damos al estudiante el sabor de un sistema dinámico, esa noción no se desarrolla explícitamente en la monografía.}
%https://rajsain.files.wordpress.com/2013/11/randomized-algorithms-motwani-and-raghavan.pdf
%
%https://www.csie.ntu.edu.tw/~r97002/temp/Concrete%20Mathematics%202e.pdf
%
%https://link.springer.com/chapter/10.1007/978-3-642-61544-3_9
%
%https://link.springer.com/chapter/10.1007/978-94-011-1814-9_9
%
%https://link.springer.com/chapter/10.1007/978-3-319-15579-1_39
%
%https://link.springer.com/chapter/10.1007/978-94-011-2058-6_14
%
%https://link.springer.com/chapter/10.1007/BFb0120904
%
%https://link.springer.com/chapter/10.1007%2FBFb0120904
%
%https://link.springer.com/article/10.1007/BF00874886
%
%https://link.springer.com/search?date-facet-mode=between&showAll=true&query=recurrence+AND+relation&facet-discipline=%22Mathematics%22

%\motto{hola}
%\runinhead{xd}
%\subruninhead{xd}
%\begin{petit}
%A
%\end{petit}

%\begin{claim}
%Afirmo que el Lema de Zorn es cierto.
%\end{claim}
%
%\begin{proof}
%$\smartqed$
%
%$\qed$
%\end{proof}
%
%\begin{case}
%	
%\end{case}
%
%\begin{conjecture}
%	
%\end{conjecture}
%
%\begin{corollary}
%	
%\end{corollary}
%
%\begin{definition}
%	
%\end{definition}
%
%\begin{example}[Quispe]
%	
%\end{example}
%
%\begin{lemma}
%	
%\end{lemma}
%
%\begin{note}
%	
%\end{note}
%
%\begin{problem}
%	
%\end{problem}
%
%\begin{property}
%	
%\end{property}
%
%\begin{proposition}
%	
%\end{proposition}
%
%\begin{question}{Bryan}
%	
%\end{question}
%
%\begin{remark}
%
%\end{remark}
%
%\begin{theorem}
%
%\end{theorem}
%
%\begin{trailer}{Enfatizar párrafos}
%\end{trailer}
%
%\begin{question}{?`Qué hora es?}
%\end{question}
%
%\begin{important}{Importante}
%	A
%\end{important}
%
%\begin{warning}{Atención}
%	
%\end{warning}
%
%\begin{tips}{Consejos}
%	
%\end{tips}
%
%\begin{overview}{Enfatizar párrafos completos}
%	
%\end{overview}
%
%\begin{backgroundinformation}{Información de fondo}
%	
%\end{backgroundinformation}
%
%\begin{legaltext}{Texto legal}
%	
%\end{legaltext}