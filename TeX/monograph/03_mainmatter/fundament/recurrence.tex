\section{Relación de recurrencia}\label{sec:recurrence}
En esta sección presentamos a nuestros lectores las nociones básicas subyacentes de las relaciones de recurrencia, así como varios ejemplos de tales relaciones. Una relación de recurrencia es una familia numerable de ecuaciones que definen sucesiones en modo recursivo. Aquellas sucesiones que así surgen se llaman \emph{soluciones de la recurrencia}, dependiendo de uno o más \emph{valores iniciales}: cada término que sigue al valor inicial en tales sucesiones es definida como una función de los términos anteriores.

\begin{example}[Relaciones de recurrencias real]\leavevmode
	\begin{enumerate}
		\item El sistema de ecuaciones con coeficientes reales en la colección infinita de incógnitas $x_{0}$, $x_{1}$, \ldots, $x_{n},$\ldots \[\ccases{x_{1}&=3x_{0}\\x_{2}&=3x_{1}\\\mathrel{\makebox[\widthof{x}]{\vdots}}&=\mathrel{\makebox[\widthof{3x}]{\vdots}}\\x_{n+1}&=3x_{n}\\\mathrel{\makebox[\widthof{x}]{\vdots}}&=\mathrel{\makebox[\widthof{3x}]{\vdots}}}\] podría indicarse más concisamente por $x_{n+1}=3x_{n},n\geq0$, es una \emph{relación de recurrencia}. La sucesión ${\left(3^{n}\right)}_{n\geq0}$ es una solución de la recurrencia dada con \emph{valor inicial} $x_{0}=1$. Es fácil de convencerse a uno mismo que en general, para cualquier número real $c\in\mathds{R}$, la sucesión ${\left(c3^{n}\right)}_{n\geq0}$ es la única solución de la \emph{recurrencia} con el valor inicial $x_{0}=c$.
		\item Con el cuidado adecuado es fácil verificar que la sucesión real \[ x_{0}=1, x_{1}=1, x_{2}=2, x_{3}=2, x_{4}=4, \ldots, x_{7}=4, \ldots, x_{2^{m}}=2^{m}, \ldots \] es la solución de la \emph{relación de recurrencia} con coeficientes reales \[ x_{n}=\ccases{2x_{n/2},&\text{si }n\geq2\text{ es par},\\x_{n-1},&\text{si }n\text{ es impar},} \] con el \emph{valor inicial} $x_{0}=1$.
		\item La sucesión real \[ x_{0}=2, x_{1}=1, x_{2}=2^{1/2}, x_{3}=1, \ldots, x_{2m-1}=1 ,x_{2m}=2^{1/2^{m}}, \ldots \] es la solución de la \emph{relación de recurrencia} con coeficientes reales \[ x_{n}=\sqrt{x_{n}-2},\quad n\geq2, \] y los \emph{valores iniciales} $x_{0}=2$ y $x_{1}=1$.
	\end{enumerate}
\end{example}

La pregunta que ahora surge naturalmente es la de definir relaciones generales de recurrencia. Buscamos exponer de manera rigurosa lo que acabamos de inferior de los ejemplos anteriores.

\begin{definition}[Relación de recurrencia]\index{Relación de recurrencia!definición}\label{def:recurrence}
	Una \textbf{relación de recurrencia} en las incógnitas $x_{i}$, $i\in\mathds{N}$, es una familia de ecuaciones
	\begin{equation*}
	x_{n}=f_{n}\left(x_{0},\ldots,x_{n-1}\right),\quad n\geq r,
	\end{equation*}
	donde $r\in\mathds{N}_{\geq1}$, y ${\left(f_{n}\right)}_{n\geq r}$ son funciones
	\begin{equation*}
	f_{n}\colon D_{n}\rightarrow\mathds{R},\quad D_{n}\subseteq\mathds{R}^{n},\quad\text{o}\quad f_{n}\colon D_{n}\rightarrow\mathds{C},\quad D_{n}\subseteq\mathds{C}^{n}.
	\end{equation*}
	Dependiendo del caso encontrado, las llamaremos \textbf{recurrencias reales}\index{Relación de recurrencia!real} o \textbf{recurrencias complejas}\index{Relación de recurrencia!compleja}. Las incógnitas $x_{0},\ldots,x_{r-1}$ son llamadas \textbf{libres}. Su número $r$ es el \textbf{orden} de la relación\index{Relación de recurrencia!orden}.
\end{definition}

Al reemplazar $n$ por $n+r$, la relación de recurrencia de orden $r$
\begin{align*}
x_{n}&=f_{n}\left(x_{0},\ldots,x_{n-1}\right),\quad n\geq r,
\shortintertext{puede también escribirse como}
x_{n+r}&=f_{n+r}\left(x_{0},\ldots,x_{n+r-1}\right),\quad n\geq0.
\end{align*}

\begin{definition}[Solución de una recurrencia]\index{Relación de recurrencia!solución}
	Una sucesión ${\left(a_{n}\right)}_{n}$ es una \textbf{solución} de la relación de recurrencia de orden $r$
	\begin{equation}
	x_{n}=f_{n}\left(x_{0},\ldots,x_{n-1}\right),\quad n\geq r,
	\end{equation}
	con $f_{n}\colon D_{n}\rightarrow\mathds{R}$, $D_{n}\in\mathds{R}^{n}$, sii
	\begin{equation*}
	\left(a_{0},\ldots,a_{n-1}\right)\in D_{n},\quad a_{n}=f_{n}\left(a_{0},a_{1},\ldots,a_{n-1}\right)\quad\forall\,n\geq r.
	\end{equation*}
\end{definition}

La sucesión $\left(a_{0},\ldots,a_{r-1}\right)$ de valores asignados para las $r$ incógnitas libres es llamada la $r$--sucesión de \textbf{valor inicial} o de las \textbf{condiciones iniciales} de la solución. Definimos la \textbf{solución general real} (respectivamente \textbf{compleja}) de la sucesión como la familia de todas las soluciones con elementos que pertenece a $\mathds{R}$ (respectivamente en $\mathds{C}$).

\begin{example}
	Considere la relación de recurrencia de primer orden definida por \[x_{n}=\frac{1}{x_{n-1}-1},\quad n\geq1.\]

	La $1$--sucesión $\left(2\right)\in D_{0}$ no es una sucesión de valor inicial de una solución, en efecto, $2$ pertenece al dominio de $f_{0}\left(x\right)=\frac{1}{x-1}$, pero $\left(2,f_{0}(x=2)\right)=\left(2,1\right)$ no pertenece al dominio de $f_{1}\left(x_{0},x_{1}\right)=\frac{1}{x-1}$. En cambio, la $1$--sucesión $\left(3\right)$ es una sucesión de valor inicial de la solución (sucesión) \[ {\left(a_{n}\right)}_{n}\coloneqq\left\{3,\frac{1}{2},-2,-\frac{1}{3},-\frac{3}{4},-\frac{4}{7},-\frac{7}{1},\ldots\right\}. \] Note que para $n\geq2$ uno tiene $a_{n}<0$ y así $a_{n+1}=\frac{1}{a_{n}-1}<0$ es distinto de $1$.
\end{example}

\begin{example}[Forma alternativa de la relación de recurrencia]
	En muchas ocasiones una relación de recurrencia de orden $r$ involucra solo los últimos $r$ términos y es de la forma	\[ x_{n}=g_{n}\left(x_{n-r},\ldots,x_{n-1}\right),\quad n\geq r, \] donde ${\left(g_{n}\right)}_{n\geq r}$ son las funciones definidas en un subconjunto $E_{n}$ de $\mathds{R}^{r}$ o $\mathds{C}^{r}$. Este último es de hecho una relación de recurrencia: es suficiente para establecer $f_{n}\left(x_{0},\ldots,x_{n-1}\right)\coloneqq g_{n}\left(x_{n-r},\ldots,x_{n-1}\right)$ para $\left(x_{0},\ldots,x_{n-1}\right)\in D_{n}\coloneqq\mathds{R}^{n-r}\times E_{n}$ (o $\mathds{C}^{n-r}\times E_{n}$) a fin de cumplir los requerimientos de la definición~\eqref{def:recurrence}.
\end{example}

%Una relación de recurrencia es una ecuación que expresa cada término de una sucesión en función de los términos precedentes. Una relación de recurrencia presenta la siguiente forma:
%\begin{align*}
%u_{n}&=\varphi\left(n,u_{n-1}\right),\forall n>0,\\
%\intertext{donde}
%\varphi&\colon\mathds{N}\times X\rightarrow x
%\end{align*}
%es una función donde $X$ es un conjunto al que deben pertenecer los elementos de una sucesión. Para cualquier $u_{0}\in X$, esto define una sucesión única con $u_{0}$ como su primer elemento, llamado el valor inicial.
%
%Es fácil modificar la definición para obtener sucesiones a partir del término del índice $1$ o superior. Esto define la relación de recurrencia de primer orden. Una relación de recurrencia de orden $k$ tiene la forma:
%\begin{align*}
%u_{n}&=\varphi\left(n,u_{n-1},u_{n-2},\ldots,u_{n-k}\right),\forall n\geq k,\\
%\intertext{donde}
%\varphi&\colon\mathds{N}\times X^{k}\rightarrow X
%\end{align*}
%Es una función que involucra $k$ elementos consecutivos de la sucesión. En este caso, se necesitan $k$ valores iniciales para definir una sucesión.