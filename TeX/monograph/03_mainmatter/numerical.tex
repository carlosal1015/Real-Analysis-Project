\begin{partbacktext}
\part{Realización numérica}
La segunda parte de la monografía se dedica a las realizaciones prácticas de problemas. Combinaremos las consideraciones teóricas sobre diferentes modelos y ecuaciones con las técnicas. Al principio presentamos modelos alternativos para problemas de interacción. En el capítulo $3$ estudiamos la formulación variacional. Este modelo debe ser considerado como la técnica más avanzada. Damos detalles en la construcción de . Segundo, la formulación es introducida en el capítulo $4$. Este nuevo enfoque alternativo es adecuado para problemas con. Nuevamente, presentamos las herramientas necesarias de discretización y simulación. El capítulo $5$ se ocupa de las herramientas para la solución de los problemas algebraicos que surgen de la discretización. En ambos casos, tenemos que lidiar con problemas muy grandes, no lineales. Finalmente, el capítulo $6$ introduce el concepto de tiempo de escala para la reducción de la dimensión de los esquemas discretos que nos permitirá reducir significativamente la complejidad de los sistemas.
\end{partbacktext}
%\chapauthor{Autor}
%\chapsubtitle{Subtítulo}
\chapter{Método de Euler}\index{Método de Euler}
\abstract{En este capítulo introducimos un tipo de funciones llamadas que pueden ser usados para aproximar otras funciones más generales}
%\chaptermark{xd}
\section{Ecuación diferencial ordinaria lineal}