\documentclass[a4,paper]{article}
\usepackage[utf8]{inputenc}
\usepackage[spanish]{babel}
\usepackage{amsmath}
\usepackage{amssymb}
\usepackage{amsthm}

\begin{document}
\section{Recurrencias Lineales con coeficientes constantes}
Una relación de recurrencia lineal de orden $ r $ con coeficientes constantes es una recurrencia del tipo:
\begin{align}\label{1}
c_{0}x_{n}+c_{1}x_{n-1}+\ldots+c_{r}x_{n-r}=h_{n}, \forall n \geq r,
\end{align}
donde $ c_{0},c_{1},\ldots,c_{r} $ son constantes reales o complejas,con $ c_{0} $ y $ c_{r} $ ambos diferentes de cero y $ (h_{n})_{n \geq r} $ es una sucesión de números reales o complejos llamado suecesión de términos no 
homogéneos de la recurrencia.La recurrencia es llamada homogénea si la sucesión de términos no homogéneos es una sucesión nula,no homogénea si $ h\neg 0 $ para algún $ n $.
La relación de recurrencia :
\begin{align}\label{2}
c_{0}x_{n}+c_{1}x_{n-1}+\ldots+c_{r}x_{n-r}=0, \forall n \geq r,
\end{align}
es llamada la recurrencia homogénea asociada, o la parte homogénea de la recurrencia \ref{1}
Como nosotros ya hemos notado ,la recurrencia:
$$
c_{0}x_{n}+c_{1}x_{n-1}+\ldots+c_{r}x_{n-r}=h_{n}, \forall n\geq r,
$$
puede ser escriot equivalentemente como:
$$
c_{0}x_{n+r}+c_{1}x_{n+(r-1)}+\ldots+c_{r}x_{n}=h_{n+r}, \forall n\geq 0.
$$
Se peude utilizar cualquiera de las formas presentadas.
\subsection*{Observación}%%%%%%ESTO ES UNA OBSERVACIÓN%%%%%%%%%%%%%%%%%%%%%%%%%%%%%%%%%%%%

Cada r-secuencia de valores asignados a las r incógnitas desconocidas de la
relación de recurrencia 
$$
c_{0}x_{n}+c_{1}x_{n-1}+\ldots+c_{r}x_{n-r}=h_{n}, \forall n\geq r,
$$
determina de forma única una solución.
Al resolver una relación de recurrencia lineal, el siguiente principio es fundamental
importancia.
\subsection*{Proposición.(Principio de superposición)}
Sea $ (u_{n})_n,(V_{n})_{n} $ serán respectivamente las soluciones de las relaciones de recurrencia lineal.\\x
\begin{tabular}{ccc}
	
	$c_{0}x_{n}+c_{1}x_{n-1}+\ldots+c_{r}x_{n-r}=h_{n}$&$ n\geq r  $  & y \\ 
	
	$c_{0}x_{n}+c_{1}x_{n-1}+\ldots+c_{r}x_{n-r}=k_{n}$& $ n\geq r $,  &  \\ 
\end{tabular}\\
con partes homogéneas iguales y secuencias de términos no homogéneos $ (h_{n})_{n} $ y $ (k_{n})_{n} $ Para cualquier par de constantes A y B, la secuencia $ (A v_{n}+ B v_{n})_{n} $  es una solución de la relación de recurrencia.
$$
c_{0}x_{n}+c_{1}x_{n-1}+\ldots+c_{r}x_{n-r}=A h_{n}+ B k_{n}
$$
La solución general de la relación de recurrencia
\begin{equation}\label{3}
c_{0}x_{n}+c_{1}x_{n-1}+\ldots+c_{r}x_{n-r}=h_{n}, \quad n\geq r
\end{equation}


\begin{proof}\\
\begin{enumerate}
\item Uno tiene fácilmente\\ $c_{0}(Au_{n}+Bv_{n})+c_{1}(Au_{n-1}+Bv_{n-1})+\ldots+c_{r}(Au_{n-r}+Bv_{n-r})=
A(c_{0}u_{n}+c_{1}u_{n-1}+\ldots+c_{r}u_{n-r})+B(c_{0}v_{n}+c_{1}v_{n-i}+\ldots+c_{r}v_{n-r})
=Ah_{n}+Bk_{n}
$
\item Sea $ (u_{n})_{n} $ una solución particular de (\ref{3}).Por el punto previo nosotros conocemos que $ (v_{n})_{n}=(u_{n})_{n}+(v_{n}-u_{n})_{n} $ es una solución de \ref{3} si y solo si $ v_{n}-u_{n} $ es una solución de la recurrecncia homogenea asociada..Por lo tanto cada solución de \ref{3} es obtenida añadiendo una solución de la  recurrencia homogenea asociada para $ (u_{n})_{n} $
\end{enumerate}
\section{Relación de recurrencia lineal con homogénea con coeficientes constantes}
\end{proof}


	
\end{document}



