\documentclass[a4,paper]{article}
\usepackage[utf8]{inputenc}
\usepackage{amsmath}
\usepackage{amssymb}
\usepackage{amsthm}
\begin{document}
\section{Recurrencias Lineales con coeficientes constantes}
Una relación de recurrencia lineal de orden $ r $ con coeficientes constantes es una recurrencia del tipo:
\begin{align}\label{1}
c_{0}x_{n}+c_{1}x_{n-1}+\ldots+c_{r}x_{n-r}=h_{n}, \forall n \geq r,
\end{align}
donde $ c_{0},c_{1},\ldots,c_{r} $ son constantes reales o complejas,con $ c_{0} $ y $ c_{r} $ ambos diferentes de cero y $ (h_{n})_{n \geq r} $ es una sucesión de números reales o complejos llamado suecesión de términos no 
homogéneos de la recurrencia.La recurrencia es llamada homogénea si la sucesión de términos no homogéneos es una sucesión nula,no homogénea si $ h\neg 0 $ para algún $ n $.
La relación de recurrencia :
\begin{align}\label{2}
c_{0}x_{n}+c_{1}x_{n-1}+\ldots+c_{r}x_{n-r}=0, \forall n \geq r,
\end{align}
es llamada la recurrencia homogénea asociada, o la parte homogénea de la recurrencia \ref{1}
Como nosotros ya hemos notado ,la recurrencia:
$$
c_{0}x_{n}+c_{1}x_{n-1}+\ldots+c_{r}x_{n-r}=h_{n}, \forall n\geq r,
$$
puede ser escriot equivalentemente como:
$$
c_{0}x_{n+r}+c_{1}x_{n+(r-1)}+\ldots+c_{r}x_{n}=h_{n+r}, \forall n\geq 0.
$$
Se peude utilizar cualquiera de las formas presentadas.
\subsection*{Observación}%%%%%%ESTO ES UNA OBSERVACIÓN%%%%%%%%%%%%%%%%%%%%%%%%%%%%%%%%%%%%

Cada r-secuencia de valores asignados a las r incógnitas desconocidas de la
relación de recurrencia 
$$
c_{0}x_{n}+c_{1}x_{n-1}+\ldots+c_{r}x_{n-r}=h_{n}, \forall n\geq r,
$$
determina de forma única una solución.
Al resolver una relación de recurrencia lineal, el siguiente principio es fundamental
importancia.
\subsection*{Proposición.(Principio de superposición)}
Sea $ (u_{n})_n,(V_{n})_{n} $ serán respectivamente las soluciones de las relaciones de recurrencia lineal.
\begin{tabular}{|c|c|c|}
	\hline 
	$c_{0}x_{n}+c_{1}x_{n-1}+\ldots+c_{r}x_{n-r}=h_{n}$&$ n\geq r  $  & y \\ 
	\hline 
	$c_{0}x_{n}+c_{1}x_{n-1}+\ldots+c_{r}x_{n-r}=k_{n}$& $  $  &  \\ 
	\hline 
\end{tabular} 
\end{document}

