\section{Conceptos Previos}

\subsection{Relación de Recurrencia}

Una relación de recurrencia es una ecuación que expresa cada término de una sucesión en función de los términos precedentes. Una relación de recurrencia presenta la siguiente forma:
\begin{align*}
	u_{n}&=\varphi\left(n,u_{n-1}\right),\forall n>0,\\
	\intertext{donde}
	\varphi&\colon\mathds{N}\times X\rightarrow x
\end{align*}
es una función donde $X$ es un conjunto al que deben pertenecer los elementos de una sucesión. Para cualquier $u_{0}\in X$, esto define una sucesión única con $u_{0}$ como su primer elemento, llamado el valor inicial.

Es fácil modificar la definición para obtener sucesiones a partir del término del índice $1$ o superior. Esto define la relación de recurrencia de primer orden. Una relación de recurrencia de orden $k$ tiene la forma:
\begin{align*}
	u_{n}&=\varphi\left(n,u_{n-1},u_{n-2},\ldots,u_{n-k}\right),\forall n\geq k,\\
	\intertext{donde}
	\varphi&\colon\mathds{N}\times X^{k}\rightarrow X
\end{align*}
Es una función que involucra $k$ elementos consecutivos de la sucesión. En este caso, se necesitan $k$ valores iniciales para definir una sucesión.

\subsection{Ecuaciones en Diferencias}

Una ecuación en diferencias es una expresión de la forma:
\begin{align*}
	G\left(n,f(n),f(n+1),\ldots,f(n+k)\right)&=0,\forall n\in\mathds{Z}\\
	\intertext{donde}
\end{align*}
$f$ es una función definida en $\mathds{Z}$.

Si después de simplificar esta expresión quedan los términos $f\left(n+k_{1}\right)$ y $f\left(n+k_{2}\right)$ como el mayor y el menor, respectivamente. Se dice que la ecuación es de orden $k=k_{1}-k_{2}$.

\begin{example}{}
	La ecuación:
	\begin{equation}
		5f(n+4)-4f(n+2)+f(n+1)+(n-2)^{3}=0
	\end{equation}
	es de orden $4-1=3$.
\end{example}

Una ecuación en diferencias de orden $k$ se dice que es \emph{lineal} si puede expresarse de la forma:
\begin{equation*}
	p_{0}(n)f(n+k)+p_{1}(n)f(0+k-1)+\cdots+p_{k}(n)f(n)=g(n),
\end{equation*}
donde los coeficientes $p_{i}$ son funciones definidas en $\mathds{Z}$.

El caso más sencillo es cuando los coeficientes son constantes $p_{i}(n)=a_{i}$:
\begin{equation*}
	a_{0}f(n+k)+a_{1}f(n+k-1)+\cdots+a_{k}f(n)=g(n).
\end{equation*}
La ecuación en diferencias se dice que es \emph{homogénea} en el caso que $g(n)=0$, y completa en el caso contrario.