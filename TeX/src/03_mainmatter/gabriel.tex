\section{Ejercicios}
%%pag 373
\begin{enumerate}
	\item Supongamos que $E_n$ es definido recursivamente en $\mathds{Z}^+$ por \[ E_0=0,E_1=2,\text{ y },E_{n+1}=2n\{E_n+E_{n-1}\} \text{ para }n\geq 1. \] Determine el valor de $E_{10}$.
	\item Supongamos que la función $f$ es definida recursivamente en $\mathds{Z}^+$ por \[ f(n)=\begin{cases}1 & \text{si }n=2^k\text{ para algún }k \in \mathds{N}\\ f(n/2) & \text{si }n\text{ es par pero no una potencia de 2} \\ f(3n+1) & \text{si }n\text{ es impar.}\end{cases}. \] Entonces 
		\begin{align*}
		f(3)=&f(10)\text{ porque }3\text{ es impar}\\
				=&f(5)\text{ porque } 10=2\times 5\\
				=&f(16)\text{ porque }5\text{ es impar}\\
				=&1\text{ porque } 16=2^4.
		\end{align*}
		\begin{enumerate}
	    \item Mostrar que $f(11)$ también es igual a $1$.
	    \item Mostrar que $f(9)$, $f(14)$, y $f(25)$ son todos iguales a $f(11)$ y, por lo tanto, todos iguales a $1$.
	    \item Escriba un programa para hallar $f(27)$.
		\end{enumerate}
	¿Crees que esta función siempre dará el valor de $1$, sin importar con qué $n$ comiences? Busque la ``Conjetura de Collatz'' o el ``Problema del granizo''.
	\item Podríamos definir una degeneración como una $n$--permutación $S$ de $\left\{1.\ldots,n\right\}$ donde cada $S_j\neq j$ y luego definir $\bm{D_n}$ como el número de degeneraciones de $\left\{1,\ldots,n\right\}$. Entonces $\bm{D_n}$ es la única sucesión que satisface la ecuación de recurrencia
		\begin{equation}\label{eq1}
			\bm{D_n}=\left(n-1\right)\left\{\bm{D_{n-1}}+\bm{D_{n-2}}\right\}\text{ para }n=3,4,5,\ldots
    \end{equation}
    con $\bm{D_1}=0$ y $\bm{D_2}=1$.
			\begin{enumerate}
				\item Mostrar que $\bm{D_2}=(2)\left(\bm{D_1}\right)+(-1)^2$.
				\item Use la inducción matemática para probar que para todo entero $n\geq 2$, \[ \bm{D_n}=(n)(\bm{D_{n-1}})+(-1)^n. \]
    \end{enumerate}
    \item Use la inducción matemática y la ecuación \eqref{eq1} para probar que \[ \forall n\in\mathds{Z}^{+}\colon\bm{D_n}=n!\sum_{j=0}^n\frac{(-1)^j}{j!}. \]
    \item Supongamos que (o busque estos dos resultados de cálculo)
    \begin{enumerate}
    	\item \[ \forall x\in\mathds{R}\colon e^x=\sum_{j=0}^\infty\frac{x^j}{j!},\text{ y también }e^{-1}=\sum_{j=0}^\infty\frac{(-1)^j}{j!}, \]
    	\item \[ \exists n\in\mathds{Z}^{+}\ni e^{-1}=\sum_{j=0}^n\frac{(-1)^j}{j!}+E_n \text{ donde }|E_n|<\left|\frac{(-1)^{n+1}}{(n+1)!}\right|=\frac{1}{(n+1)!}. \]
    \end{enumerate}
  
    \begin{enumerate}
        \item Use el resultado de la pregunta anterior para mostrar \[ \frac{n!}{e}=\bm{D_n}+n!E_n\text{ donde }|n!E_n|<\frac{n!}{(n+1)!}=\frac{1}{n+1}\leq 1/2. \]
        \item Explique por qué $\bm{D_n}-\tfrac{1}{2}\leq n!/e\leq\bm{D_n}+\tfrac{1}{2}$.
        \item ¿Es $\lceil n!/e\rfloor=\bm{D_n}$?% TODO:
    \end{enumerate}
    \item La \emph{función de Ackermann} a veces es definida recursivamente en una forma ligeramente diferente
    \item Supongamos que $\bm{A}$ es un conjunto de $2n$ objetos. Sea $\bm{P_n}$ el número de diferentes maneras que los objetos en $\bm{A}$ pueden ser ``emparejados'' (el número de diferentes particiones de $\bm{A}$ en $2$--subconjuntos). Supongamos que $n\in\mathds{Z}^{+}$. Si $n=2$, entonces $\bm{A}$ tiene cuatro elementos, $\bm{A}=\left\{x_1,x_2,x_3,x_4\right\}$.
    
    Los tres posibles emparejamientos son:
    \begin{enumerate}
    	\item $x_1$ con $x_2$ y $x_3$ con $x_4$,
    	\item $x_1$ con $x_3$ y $x_2$ con $x_4$,
    	\item $x_1$ con $x_4$ y $x_2$ con $x_3$.
    \end{enumerate}
		Así $\bm{P_2}=3$.
    \begin{enumerate}
        \item Mostrar que si $n=3$ y $\bm{A}=\left\{x_1,x_2,x_3,x_4,x_5,x_6\right\}$, hay $15$ posibles emparejamientos enumerándolos todos:
        \begin{enumerate}
        	\item $x_1$ con $x_2$ y $x_3$ con $x_4$ y $x_5$ con $x_6$
        	\item Así $\bm{P_3}=15$.
        \end{enumerate}
        \item Mostrar que $\bm{P_n}$ debe satisfacer la ER $\bm{P_n}=(2n-1)\bm{P_{n-1}}$ para $\forall n\geq2$.
        \item Use la ecuación de recurrencia y la inducción matemática para probar \[ \bm{P_n}=(2n)!/[2^n\times n!]\text{ para }\forall n\geq 1. \]
    \end{enumerate}
    \item Mostrar que $y_n=\frac{n(n-1)}{2}+c$ para $n>0$ es una solución de la relación de recurrencia \[ y_{n+1}=y_{n}+n. \]
    \item Supongamos que una sucesión es definida por: \[ f(0)=5\text{ y }f(n+1)=2\times f(n)+1\text{ para } n=0,1,2,\ldots \]
    \begin{enumerate}
        \item Halle el valor de $f(10)$ .
        \item Probar que la sucesión no es una sucesión aritmética ni una sucesión geométrica.
    \end{enumerate}
    \begin{enumerate}
        \item Encuentre la solución general de la ecuación de recurrencia \[ S_n=3S_{n-1}-10\text{ para }n=1,2,\ldots \]
        \item\label{b} Determine la solución particular donde $S_{0}=15$.
        \item Use la fórmula en (\ref{b}) para evaluar $S_6$ y verifique su respuesta usando la ecuación de recurrencia en sí.
    \end{enumerate}
    \item Suponga $s_{0}=60$ y $s_{n+1}=(1/5)s_n-8$ para $n=0,1,\ldots$
    \begin{enumerate}
        \item Halle $s_{1}$, $s_{2}$, y $s_{3}$.
        \item Resuelva la relación de recurrencia para dar una fórmula para $s_{n}$.
        \item ¿Es esa sucesión convergente? Si es así, ¿cuál es el límite?
        \item ¿La serie correspondiente converge? Si es así, ¿cuál es límite?
    \end{enumerate}
    \item Supongamos $s_{0}=75$ y $s_{n+1}=(1/3)s_{n}-6$ para $n=0,1,\ldots$
    \begin{enumerate}
        \item Halle $s_{1}$, $s_{2}$, y $s_{3}$.
         \item Resuelva la relación de recurrencia para dar una fórmula para $s_{n}$.
        \item ¿Es esa sucesión convergente? Si es así, ¿cuál es el límite?
        \item ¿La serie correspondiente converge? Si es así, ¿cuál es límite?
    \end{enumerate}
    \begin{enumerate}
        \item Mostrar que $f_{n}=A\times3^{n}+B\times2^{n}$ satisface la ecuación de recurrencia \[ f_{n}=5f_{n-1}-6f_{n-2}\text{ para }n\geq 2. \]
        \item Encuentre la solución particular (valores para $A$ y $B$) para que \[ f_{0}=4\text{ y }f_{1}=17. \]
    \end{enumerate}
\end{enumerate}