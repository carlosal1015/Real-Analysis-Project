\section{Recurrencias Lineales con coeficientes constantes}

Una relación de recurrencia lineal de orden $r$ con coeficientes constantes es una recurrencia del tipo:
\begin{align}\label{1}
	c_{0}x_{n}+c_{1}x_{n-1}+\cdots+c_{r}x_{n-r}=h_{n},\forall n\geq r,
\end{align}
donde $c_{0},c_{1},\ldots,c_{r}$ son constantes reales o complejas,con $c_{0}$ y $c_{r}$ ambos diferentes de cero y $(h_{n})_{n\geq r}$ es una sucesión de números reales o complejos llamado sucesión de términos no homogéneos de la recurrencia. La recurrencia es llamada homogénea si la sucesión de términos no homogéneos es una sucesión nula, no homogénea si $h\neg 0 $ para algún $n$. La relación de recurrencia:
\begin{align}\label{2}
	c_{0}x_{n}+c_{1}x_{n-1}+\cdots+c_{r}x_{n-r}=0,\forall n\geq r,
\end{align}
es llamada la recurrencia homogénea asociada, o la parte homogénea de la recurrencia \eqref{1}. Como nosotros ya hemos notado, la recurrencia:
\begin{equation*}
	c_{0}x_{n}+c_{1}x_{n-1}+\cdots+c_{r}x_{n-r}=h_{n},\forall n\geq r,
\end{equation*}
puede ser escrito equivalentemente como
\begin{equation*}
	c_{0}x_{n+r}+c_{1}x_{n+(r-1)}+\cdots+c_{r}x_{n}=h_{n+r},\forall n\geq 0.
\end{equation*}
Se puede utilizar cualquiera de las formas presentadas.

\begin{remark}
	Cada $r$-secuencia de valores asignados a las $r$ incógnitas desconocidas de la relación de recurrencia
	\begin{equation*}
		c_{0}x_{n}+c_{1}x_{n-1}+\cdots+c_{r}x_{n-r}=h_{n},\forall n\geq r,
	\end{equation*}
	determina de forma única una solución. Al resolver una relación de recurrencia lineal, el siguiente principio es fundamental importancia.
\end{remark}

\begin{proposition}{Principio de superposición}
	Sean $(u_{n})_n$, $(V_{n})_{n}$ respectivamente las soluciones de las relaciones de recurrencia lineal. x
\end{proposition}

\begin{tabular}{ccc}
	$c_{0}x_{n}+c_{1}x_{n-1}+\cdots+c_{r}x_{n-r}=h_{n}$	&	$n\geq r$	&	y\\
	$c_{0}x_{n}+c_{1}x_{n-1}+\cdots+c_{r}x_{n-r}=k_{n}$	&	$n\geq r$	&,\\
\end{tabular}

con partes homogéneas iguales y secuencias de términos no homogéneos $(h_{n})_{n}$ y $(k_{n})_{n}$. Para cualquier par de constantes $A$ y $B$, la secuencia $(Av_{n}+Bv_{n})_{n}$ es una solución de la relación de recurrencia.
\begin{equation*}
	c_{0}x_{n}+c_{1}x_{n-1}+\cdots+c_{r}x_{n-r}=Ah_{n}+Bk_{n}.
\end{equation*}
La solución general de la relación de recurrencia
\begin{equation}\label{3}
	c_{0}x_{n}+c_{1}x_{n-1}+\cdots+c_{r}x_{n-r}=h_{n},\quad n\geq r.
\end{equation}

\begin{proof}\leavevmode
	\begin{enumerate}
		\item Uno tiene fácilmente
			\begin{equation*}
				\begin{split}
					&c_{0}(Au_{n}+Bv_{n})+c_{1}(Au_{n-1}+Bv_{n-1})+\cdots+c_{r}(Au_{n-r}+Bv_{n-r})=\\
					&\phantom{c_{0}(Au_n+}=A(c_{0}u_{n}+c_{1}u_{n-1}+\cdots+c_{r}u_{n-r})+B(c_{0}v_{n}+c_{1}v_{n-i}+\cdots+c_{r}v_{n-r})\\
					&\phantom{c_{0}(Au_n+}=Ah_{n}+Bk_{n}.
				\end{split}
			\end{equation*}
		\item Sea $(u_{n})_{n}$ una solución particular de \eqref{3}.Por el punto previo nosotros conocemos que $(v_{n})_{n}=(u_{n})_{n}+(v_{n}-u_{n})_{n}$ es una solución de \eqref{3} si y solo si $v_{n}-u_{n}$ es una solución de la recurrencia homogénea asociada. Por lo tanto cada solución de \eqref{3} es obtenida añadiendo una solución de la  recurrencia homogénea asociada para $(u_{n})_{n}$.
	\end{enumerate}
\end{proof}

\section{Relación de recurrencia lineal con homogénea con coeficientes constantes}

La secuencia nula es una solución de cualquier relación de recurrencia lineal. La estructura de la solución general de una relación de recurrencia lineal homogénea corresponde a la estructura de la solución general de un sistema de ecuaciones lineales homogéneas.
\begin{proposition}{}
	Considere la relación de recurrencia lineal homogénea de orden $r$:
		\begin{equation}\label{4}
			c_{0}x_{n}+c_{1}x_{n-1}+\cdots+c_{r}x_{n-r}=0,\quad n\geq r\quad\left(c_{0}c_{r}\neq0\right)
		\end{equation}
		\begin{enumerate}
			\item Cualquier combinación lineal de soluciones de \eqref{4} es de nuevo una solución de \eqref{4}.
			\item Existe $r$ soluciones de \eqref{4} tal que cualquier otra solución de \eqref{4} puede ser expresado únicamente como su combinación lineal.
	\end{enumerate}
\end{proposition}

\begin{proof}
	\begin{enumerate}
		\item Esto sigue inmediatamente por el ``Principio de Superposición''.
		\item Para todo $i\in\left\{0,\ldots,r-1 \right\}$ sea $\left(u^{i}_{n}\right)_{n}$ la solución de \eqref{4} con $r$--sucesión de datos iniciales iguales a $0$ para lugares $j\neq i$, iguales a $1$ en lugares $i$, es decir: \[ u^{i}_{j}=0\text{ si }j\neq i,\quad u^{i}_{i}=1\quad j\in\left\{0,\ldots,r-1 \right\}. \]
		Consideramos ahora alguna solución $(a_{n})_{n}$ de \eqref{4}; la combinación lineal \[ a_{0}{\left(u^{0}_{n}\right)}_{n}+a_{1}{\left(u^{1}_{n}\right)}_{n}+\cdots+a_{r-1}(u^{r-1}_{n})_{n}, \]	es una solución de \eqref{4} con secuencia de datos iniciales $\left(a_{0},\ldots,a_{r-1}\right)$. Ya que la secuencia de datos iniciales determinan la solución de una relación de recurrencia, uno tiene \[ {\left(a_{n}\right)}_{n}=a_{0}\left(u^{0}_{n}\right)_{n}+a_{1}\left(u^{1}_{n}\right)_{n}+\cdots+a_{r-1}\left(u^{r-1}_{n}\right)_{n}. \]
	\end{enumerate}
\end{proof}

\begin{definition}
	Nosotros definimos el polinomio característico de una relación de recurrencia con coeficientes constantes de orden $r$ de la siguiente manera: \[ c_{0}x_{n}+c_{1}x_{n-1}+\cdots+c_{r}x_{n-r}=h_{n},\quad n\geq r\left(c_{0}c_{r}\neq0\right), \] para sel el polinomio de grado $r$: \[ P(X)\coloneqq c_{0}X^{r}+c_{1}X^{r-1}+\cdots+c_{r}. \] Cada polinomio de grado $r$ tiene exactamente $r$ raíces complejas contando con su multiplicidad. Nosotros vemos ahora que la sucesión de las potencias naturales de una determinada raíz del polinomio característico de una relación de recurrencia lineal es una solución de la correspondiente relación homogénea.
\end{definition}

\begin{proposition}{}
	Sea $\lambda\in\mathds{C}$. La sucesión $\left(\lambda^{n}\right)_{n}$ de las potencias de $\lambda$ es una solución de la relación de recurrencia lineal homogénea
		\begin{align}\label{5}
			c_{0}x_{n}+c_{1}x_{n-1}+\ldots+c_{r}x_{n-r}=0,\quad n\leq r \quad (c_{0}c_{r}\neq 0),
		\end{align}
	sii $\lambda$ es una raíz de este polinomio característico.
\end{proposition}

\begin{proof}
	Dado que $c_{r}\neq0$, las raíces del polinomio característico deben ser necesariamente no nulas. Sustituyendo los valores de la sucesión ${\left(\lambda^{n}\right)}_{n}$ en la recurrencia, uno tiene \[ c_{0}x_{n}+c_{1}x_{n-1}+\cdots+c_{r}x_{n-r}=0, \] y dividiendo por $\lambda^{n-r}\neq0$ \[ c_{0}\lambda^{r}+c_{1}\lambda^{r-1}+\cdots+c_{r}=0. \]	Por lo tanto, la sucesión ${\left(\lambda^{n}\right)}_{n}$ es una solución de \eqref{5} si y solo si $\lambda$ es una raíz del polinomio $c_{0}X^{r}+c_{1}X^{r-1}+\cdots+c_{r}$.
\end{proof}

En general, no es fácil encontrar las raíces de un polinomio de grado mayor que dos, aunque uno puede siempre usar un adecuado CAS. El siguiente criterio simple; sin embargo, muestra cómo encontrar las raíces racionales de un polinomio con coeficientes enteros.

\begin{proposition}(Las raíces racionales de un polinomio con coeficientes enteros)
	Sea $P(X)=c_{0}X^{r}+c_{1}X^{r-1}+\cdots+c_{r}$ un polinomio con coeficientes enteros $c_{0}\ldots c_{r}\in\mathds{Z}$, con $c_{0}\neq 0$. Si la fracción $\tfrac{a}{b}$ con $a,b\in\mathds{Z}$ con $\operatorname{mcd}=1$ es una raíz de $P(X)$, luego $a\divides c_{r}$ y $b\divides c_{0}$. En particular, si $c_{0}=\pm1$ las raíces racionales del polinomio $P(X)$ son enteros que dividen a $c_{r}$.
\end{proposition}

\begin{proof}
	Dado $c_{0}\left(\frac{a}{b}\right)^{r}+c_{1}{\left(\frac{a}{b}\right)}^{r-1}+\cdots+c_{r-1}\left(\frac{a}{b}\right)+c_{r}=0$, multiplicado por $b^{r}$ obtenemos: \[ 	c_{0}a^{r}+c_{1}a^{r-1}b+\cdots+c_{r-1}ab^{r-1}+c_{r}b^{r}=0. \] Como $a\divides c_{0}a^{r}+c_{1}a^{r-1}b+\cdots+c_{r-1}ab^{r-1}$, luego tiene que dividir también $c_{r}b^{r}$, y por lo tanto, al no tener $a$ y $b$ factores comunes, $a\divides c_{r}$; análogamente $b\divides c_{0}a^{r}$ y por lo tanto divide a $c_{0}$.
\end{proof}

\begin{example}{}
	La recurrencia homogénea de segundo orden: \[ x_{n}=2x_{n-1}-2x_{n-2},\quad n\geq2, \] tiene polinomio característico $X^{2}-2X+2$ cuyas raíces son $\lambda_{1}=1-i$ y $\lambda_{2}=1+i$. Las sucesiones ${\left((1-i)^{n}\right)}_{n}$ y ${\left((1+i)^{n}\right)}_{n} $ son las soluciones bases de la recurrencia. La solución general compleja de la recurrencia es: \[ x_{n}=A_{1}{\left(1-i\right)}^{n}+A_{2}{\left(1+i\right)}^{n},\quad n\geq 0, \] con la variante de $A_{1}$ y $A_{2}$ entre los números complejos. Veamos la solución real general. Uno tiene: \[ \lambda_{1}=1-i=\sqrt{2}\left(\frac{\sqrt{2}}{2}-\frac{\sqrt{2}}{2}i\right)=\sqrt{2}\left(\cos\left(\frac{\pi}{4}\right)-i\sen\left(\frac{\pi}{4}\right)\right) \] y \[ \lambda_{2}=1+i=\overline{\lambda_{1}}=\sqrt{2}\left(\cos\left(\frac{\pi}{4}\right)-i\sen\left(\frac{\pi}{4}\right)\right). \] Luego, las sucesiones ${\left(2^{n/2}\cos\left( \frac{n\pi}{4}\right)\right)}_{n}$ y ${\left(2^{n/2}\sen\left(\frac{n\pi}{4}\right)\right)}_{n}$ son las soluciones base reales de la recurrencia. Por lo tanto, la solución general real de la recurrencia es: \[ x_{n}=A_{1}2^{n/2}\cos\left(\frac{n\pi}{4}\right)+A_{2}2^{n/2}\sen\left(\frac{n\pi}{4}\right),\quad n\geq 0, \] con la variación de $A_{1}$ y $A_{2}$ entre los números reales.
\end{example}