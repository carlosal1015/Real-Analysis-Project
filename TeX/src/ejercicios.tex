\documentclass[12pt, a4paper]{book}
\usepackage[utf8]{inputenc}
%\usepackage[spanish]{babel}
\usepackage[spanish,es-lcroman]{babel}%Permite la enumeración con números romanos en minúsculas("Que no es usada en babel-spanish").
\usepackage{amsmath}
\usepackage{amssymb}
\usepackage{amsfonts}
\usepackage{amsthm}
\usepackage[margin=2cm]{geometry}
\usepackage{graphicx}
\usepackage{fancybox}
\usepackage{enumitem}
\usepackage{IEEEtrantools}

\begin{document}

\section{Ejercicios} %%pag 373
\begin{enumerate}
    \item Supongamos que $E_n$ es definido recursivamente en $\mathbb{Z}^+$ por $$E_0=0,\, E_1=2\,\, \text{y} \,\,E_{n+1}=2n\{E_n+E_{n-1}\}\,\, \text{para} \,\, n\geq 1.$$ Determine el valor de $E_{10}$.
    %%%%
    \item Supongamos que la función $f$ es definida recursivamente en $\mathbb{Z}^+$ por \[
f(n)=\begin{cases} 
1 & \text{si }n=2^k\text{ para algún }k \in \mathbb{N}\\ f(n/2) & \text{si }n\text{ es par pero no una potencia de 2} \\ f(3n+1) & \text{si }n\text{ es impar.}
\end{cases}
\]
Entonces 
\begin{align*}
  f(3)=&f(10)\quad \text{porque 3 es impar} \\
  =&f(5)\quad \text{  porque 10 = 2} \times 5\\
  =&f(16)\quad \text{porque 5 es impar}\\
  =&1\quad\quad\text{ porque 16} = 2^4.
\end{align*}

\begin{enumerate}
    \item Mostrar que $f(11)$ también es igual a 1.
    \item Mostrar que $f(9)$, $f(14)$, Y $f(25)$ son todos iguales a $f(11)$ y, por lo tanto, todos iguales a 1.
    \item Escriba un programa para hallar $f(27)$.
\end{enumerate}

// ¿Crees que esta función siempre dará el valor de 1, sin importar con qué $n$ comiences?\\// Busque la ''Conjetura de Collatz'' o el "Problema del granizo".
    %%%%
    \item Podríamos definir una degeneración como una $n-$permutación $S$ de $\{1..n\}$ donde cada $S_j\neq j$ y luego definir $\mathbf{D_n}$ como el número de degeneraciones de $\{1..n\}$. Entonces $\mathbf{D_n}$ es la única sucesión  que satisface la ecuación de recurrencia \begin{equation}\label{eq1}
    \mathbf{D_n} =(n-1)\{\mathbf{D_{n-1}}+\mathbf{D_{n-2}}\}\quad\text{para }n=3,4,5,...    
    \end{equation} con $\mathbf{D_1}=0$ y $\mathbf{D_2}=1$.
    \begin{enumerate}
        \item Mostar que $\mathbf{D_2}=(2)(\mathbf{D_1})+(-1)^2$.
        \item Use la inducción matemática para probar que para todo entero $n\geq 2$, $$\mathbf{D_n}=(n)(\mathbf{D_{n-1}})+(-1)^n.$$
    \end{enumerate}
    %%%%
    \item Use la inducción matemática y la ecuación (\ref{eq1}) para probar que $$\text{para todo entero positivo }n,\,\, \mathbf{D_n}=n!\sum_{j=0}^n\frac{(-1)^j}{j!}.$$
    %%%%
    \item Supongamos que (o busque estos dos resultados de cálculo) $$\text{A. para todo número real }x,\,e^x=\sum_{j=0}^\infty\frac{x^j}{j!},\text{ y también }e^{-1}=\sum_{j=0}^\infty\frac{(-1)^j}{j!},$$ y B. para algún $n$ entero positivo, $$e^{-1}=\sum_{j=0}^n\frac{(-1)^j}{j!}+E_n \text{  donde }|E_n|<\left|\frac{(-1)^{n+1}}{(n+1)!}\right|=\frac{1}{(n+1)!}.$$
    \begin{enumerate}
        \item Use el resultado de la pregunta anterior para mostrar $$\frac{n!}{e}=\mathbf{D_n}+n!E_n\text{  donde }|n!E_n|<\frac{n!}{(n+1)!}=\frac{1}{n+1}\leq 1/2.$$
        
        \item Explique por qué $\mathbf{D_n}-\frac{1}{2}\leq n!/e\leq \mathbf{D_n}+\frac{1}{2}$.
        \item ¿Es $\lceil n!/e\rfloor =\mathbf{D_n}$?
    \end{enumerate}
    %%%%
    \item La \textbf{función de Ackermann} a veces es definida recursivamente en una forma ligeramente diferente
    %%%%
    \item Supongamos que \textbf{A} es un conjunto de $2n$ objetos. Sea $\mathbf{P_n}$ el número de diferentes maneras que los objetos en \textbf{A} pueden ser 'emparejados'(el número de diferentes particiones de \textbf{A} en 2-subconjuntos).\hfill // Supongamos que $n$ es un entero positivo.\\[0.2cm]
    Si $n=2$, entonces \text{A} tiene cuatro elementos, $\mathbf{A}=\{x_1,x_2,x_3,x_4\}$.\\[0.1cm]
    Los tres posibles emparejamientos son\\
    1. $x_1$ con $x_2$ y $x_3$ con $x_4$,\\
    2. $x_1$ con $x_3$ y $x_2$ con $x_4$,\\
    3. $x_1$ con $x_4$ y $x_2$ con $x_3$,\hspace{3cm} // Así $\mathbf{P_2}=3$
    \begin{enumerate}
        \item Mostrar que si $n=3$ y $\mathbf{A}=\{x_1,x_2,x_3,x_4,x_5,x_6\}$, hay 15 posibles emparejamientos enumerándolos todos:\\
        1.$x_1$ con $x_2$ y $x_3$ con $x_4$ y $x_5$ con $x_6$\\
        2. ... \hspace{5.3cm} // Así $\mathbf{P_3}=15$.
        \item Mostrar que $\mathbf{P_n}$ debe satisfacer la ER $\mathbf{P_n}=(2n-1)\mathbf{P_{n-1}}$ para $\forall n\geq2$.
        \item Use la ecuación de recurrencia y la inducción matemática para probar $$\mathbf{P_n}=(2n)!/[2^n\times n!]\text{ para }\forall n\geq 1.$$
    \end{enumerate}
    %%%%
    \item Mostrar que $y_n=\frac{n(n-1)}{2}+c$ para $n>0$ es una solución de la relación de recurrencia $$y_{n+1}=y_n+n.$$
    %%%%
    \item Supongamos que una sucesión es definida por:$$f(0)=5\text{ y}$$ $$f(n+1)=2\times f(n)+1\text{ para } n=0,1,2,...$$
    \begin{enumerate}
        \item Halle el valor de $f(10).$ 
        \item Probar que la sucesión no es una sucesión arimética ni una sucesión geométrica.
    \end{enumerate}
    %%%%
    \item \begin{enumerate}
        \item Encuentre la Solución General de la ecuación de recurrencia $$S_n=3S_{n-1}-10\text{  para }n=1,2,....$$
        \item\label{b} Determine la solución particular donde $S_0=15$.
        \item Use la fórmula en (\ref{b}) para evaluar $S_6$ y verifique su respuesta usando la ecuación de recurrencia en sí.
    \end{enumerate}
    %%%%
    \item Suponga $s_0=60$ y $s_{n+1}=(1/5)s_n-8$ para $n=0,1,...$
    \begin{enumerate}
        \item Halle $s_1$, $s_2$, y $s_3$.
        \item Resuelva la relación de recurrencia para dar una fórmula para $s_n$.
        \item ¿Es esa suceción convergente? Si es así, ¿Cuál es el límite?
        \item ¿La serie correspondiente converge? Si es así, ¿Cuál es límite?
    \end{enumerate}
    %%%%
    \item Supongamos $s_0=75$ y $s_{n+1}=(1/3)s_n - 6$ para $n=0,1, ....$
    \begin{enumerate}
        \item Halle $s_1$, $s_2$, y $s_3$.
         \item Resuelva la relación de recurrencia para dar una fórmula para $s_n$.
        \item ¿Es esa suceción convergente? Si es así, ¿Cuál es el límite?
        \item ¿La serie correspondiente converge? Si es así, ¿Cuál es límite?
    \end{enumerate}
    %%%%
    \item \begin{enumerate}
        \item Mostrar que $f_n=A\times 3^n + B\times 2^n$ satisface la ecuación de recurrencia $$f_n=5f_{n-1}-6f_{n-2}\text{ para } n\geq 2.$$
        \item Encuentre la solución particular (valores para $A$ y $B$) para que $$f_0=4\text{ y }f_1=17.$$
    \end{enumerate}
    %%%%
    \item 
    %%%%
    \item 
    %%%%
    \item 
    %%%%
    \item 
    %%%%
    \item 
    %%%%
    \item 
    %%%%
    \item 
    %%%%
    \item 
    %%%%
    \item 
    %%%%
    \item 
    %%%%
    \item 
    %%%%
    \item 
    %%%%
    \item 
    %%%%
    \item 
    %%%%
    \item 
    %%%%
    \item 
    
\end{enumerate}

\end{document}
