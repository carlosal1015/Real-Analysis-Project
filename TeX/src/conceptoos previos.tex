\documentclass[a4,paper]{article}
\usepackage[utf8]{inputenc}
\usepackage{amsmath}
\usepackage{amssymb}
\usepackage{amsthm}
\newtheorem{ejem}{Ejemplo}[subsection]



\begin{document}
\section{Conceptos Previos}
\subsection{Relación de Recurrencia}
Una relación de recurrencia es una ecuación que expresa cada elemento de una sucesión  en función de los anteriores.Una relación de recurrencia presenta la siguiente forma:
$$
u_{n} = \varphi (n,u_{n-1}) ,\forall n>0,
$$
donde 
$$
\varphi:\mathbb{N}\times X\rightarrow x
$$
es una función donde $ X $ es un conjunto al que deben pertenecer los elementos de una sucesión.Para cualquier $ u_{0} \in X $,esto define una sucesión única con $ u_{0} $ como su primer elemento,llamado el valor inicial.\\
Es fácil modificar la definición para obtener sucesiones a partir del término del índice 1 o superior.\\
Esto define la relación de recurrencia de primer orden.Una recurrencia de orden $ k $ tiene la forma:
$$
u_{n}=\varphi(n,u_{n-1},u_{n-2},\ldots,u_{n-k}),\forall n \geq k,
$$
donde $ \varphi:\mathbb{N}\times X^{k} \rightarrow X$.Es una función que involucra $ k $ elementos consecutivos de la sucesión .En este caso ,se necesitan $ k $ valores iniciales para definir una sucesión.
\subsection{Ecuaciones en Diferencias}
Una ecuación en diferencias es una expresión de la forma:
$$
G(n,f(n),f(n+1),\ldots,f(n+k))=0,\forall n \in \mathbb{Z}
$$
donde $ f $ es una función definida en $ \mathbb{Z} $.\\
Si después de simplificar esta expresión quedan los términos $ f(n+k_{1}) $ y $ f(n+k_{2}) $ como el mayor y el menor,respectivamente, se dice que la ecuación es de orden $ k=k_{1}-k_{2} .$
\begin{ejem}
La ecuación :
$$
5f(n+4)-4f(n+2)+f(n+1)+(n-2)^{3}=0
$$
es de orden $ 4-1=3$.
\end{ejem}
Una ecuación en diferencias de orden $ k $ se dice lineal si puede expresarse de la forma:
$$
p_{0}(n)f(n+k)+p_{1}(n)f(0+k-1)+\ldots+p_{k}(n)f(n)=g(n),
$$
donde los coeficientes $ p_{i} $ son funciones definidas en $ \mathbb{Z} $.\\
El caso más sencillo es cuando los coeficientes son constantes $ p_{i}(n)=a_{i}$:
$$
a_{0}f(n+k)+a_{1}f(n+k-1)+\ldots+a_{k}f(n)=g(n).
$$
La ecuación en diferencias se dice homogénea en el caso de que $ g(n)=0 $, y completa en el caso contrario.



\end{document}
