\foreword
El problema \textsc{isoperimétrico}, un problema clásico visto en la antigüedad, lo que no necesita mucho de conocimiento matemático, en el que la proposición consiste: Dado
un número real positivo $L$, se trata de estudiar la cuestión siguiente: de todas las curvas cerradas del plano de longitud dada $L>0$, ¿cuál es la que encierra mayor área?
Esta cuestión puede ser respondida inclusive por personas que cursan el menor grado, en el que su intuición les indica que es la \emph{circunferencia}, la curva que encierra el mayor área y se sabe que la respuesta es $\tfrac{L^2}{4\pi}$. Pues ahí todo bien, la respuesta no era nada complicada. Pero el problema consiste en cómo se demostraría con matemáticas que la circunferencia maximiza el área, pues si se puede formar infinitas curvas con perímetro $L$. La materia de estudio del \textsc{cálculo variacional} consiste en buscar máximos y mínimos (o más generalmente extremos relativos) de funcionales continuos definidos sobre algún espacio funcional. También constituyen una generalización del cálculo elemental de máximos y mínimos de funciones reales de una variable.\par
\vspace{\baselineskip}
\begin{flushright}\noindent
Rímac, abril 2019\hfill {\it El profesor del curso}
\end{flushright}