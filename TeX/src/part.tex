\begin{partbacktext}
\part{Primera parte}
\end{partbacktext}
%
%\chapauthor{Autor}
%\chapsubtitle{Subtítulo}
%\chapter{Título}
%\chaptermark{xd}
%\sectionmark{xdd}
%
%\section{Holaa}
%\motto{hola}
%aaaaaaaa
%\runinhead{xd}
%aaaaaa
%\subruninhead{xd}
%
%\begin{petit}
%A
%\end{petit}
%
%\[ x = \ccases{ x & 1,\\x & 2,\\x & \text{otherwise} } \]
%$ \epsilon $ is a small positive quantity.
%
%\[ \lim_{n\to +\infty} \SQRT{
%	\frac{\displaystyle \int_{-\epsilon}^\epsilon \cos^n x}
%	{\displaystyle \int_{-\pi/2}^{\pi/2} \cos^n x }}
%= 1 \]
%
%\[ \PARENS{ \begin{matrix}
%	a & b \\
%	c & d \\
%	e & f \\
%	g & h
%	\end{matrix} } \]
%
%\begin{claim}
%	Afirmo que el Lema de Zorn es cierto.
%\end{claim}
%
%\begin{proof}
%$\smartqed$
%AAAAAAAAAAAAAAAAAAAAAAAAAAAAAAAAAAAAAAAAAAAAA
%$\qed$
%\end{proof}
%
%\begin{case}
%	content
%\end{case}
%
%\begin{conjecture}
%	content
%\end{conjecture}
%
%\begin{corollary}
%	content
%\end{corollary}
%
%\begin{definition}
%	content
%\end{definition}
%
%\begin{example}{Quispe}
%	content
%\end{example}
%
%\begin{exercise}
%	content
%\end{exercise}
%
%\begin{lemma}
%	content
%\end{lemma}
%
%\begin{note}
%	content
%\end{note}
%
%\begin{problem}
%	content
%\end{problem}
%
%\begin{property}
%	content
%\end{property}
%
%\begin{proposition}
%	content
%\end{proposition}
%
%\begin{question}{Bryan}
%	content
%\end{question}
%
%\begin{remark}
%	content
%\end{remark}
%
%\begin{solution}
%	content
%\end{solution}
%
%\begin{theorem}
%	content
%\end{theorem}

\begin{prob}
\label{1}
Supongamos que $E_n$ es definido recursivamente en $\mathds{Z}^+$ por $E_0=0$, $E_1=2$,\ldots, $E_{n+1}=2n\{E_n+E_{n-1}\}$ para $n\geq1$. Determine el valor de $E_{10}$.
\end{prob}
\begin{prob}
Supongamos que la función $f$ es definida recursivamente en $\mathds{Z}^+$ por
\[
	f(n)=\begin{cases} 
	1 & \text{ si }n=2^k\text{ para algún }k\in\mathds{N},\\
	f(n/2) & \text{ si }n\text{ es par pero no una potencia de }2,
	\\ f(3n+1) & \text{ si }n\text{ es impar}.
	\end{cases}
\]
Entonces 
\begin{align*}
	f(3)=
	&f(10)\quad \text{porque 3 es impar} \\
	=&f(5)\quad \text{  porque 10 = 2} \times 5\\
	=&f(16)\quad \text{porque 5 es impar}\\
	=&1\quad\quad\text{ porque 16} = 2^4.
\end{align*}
\end{prob}

%\begin{trailer}{Enfatizar párrafos}
%\end{trailer}
%
%\begin{question}{?`Qué hora es?}
%\end{question}
%
%\begin{important}{Importante}
%	A
%\end{important}
%
%\begin{warning}{Atención}
%	content
%\end{warning}
%
%\begin{tips}{Consejos}
%	content
%\end{tips}
%
%\begin{overview}{Enfatizar párrafos completos}
%	content
%\end{overview}
%
%\begin{backgroundinformation}{Información de fondo}
%	A
%\end{backgroundinformation}
%
%\begin{legaltext}{Texto legal}
%	
%\end{legaltext}