\documentclass[a4,paper]{article}
\usepackage[spanish]{babel}
\usepackage[utf8]{inputenc}
\usepackage{multicol}
\usepackage{amsmath,amssymb}
\setlength{\columnsep}{4.5cm}
\setlength{\columnseprule}{0mm}
\newcommand{\centrado}[1]{
	\ \\[.1\baselineskip]
	{\Large\bfseries\centering #1\par}
	\hrule
	\ \\[.1\baselineskip]
}

\begin{document}
%\centrado{Glosario de Símbolos}
\centrado{Alfabeto griego}
\begin{multicols}{2}
\begin{tabular}{ccc}
	$\alpha$	&	$A$	&	alpha		\\
	$\beta$		&	$B$	&	beta		\\
	$\gamma$	& $\Gamma$	&	gamma		\\
	$\delta$	& $\Delta$&	delta		\\
	$\epsilon$ & $E$ & épsilon \\
	$\zeta$		& $Z$	&	zeta		\\
	$\eta$		& $H$ &	eta			\\
	$\theta$	& $\Theta$ &	theta		\\
	$\iota$		& $I$&	iota		\\
	$\kappa$	& $K$ &	kappa		\\
	$\lambda$ & $\Lambda$ & lambda \\
	$\mu$			& $M$ &	mu			\\
\end{tabular}
\columnbreak
\begin{tabular}{ccc}
	$\nu$			& $N$ &	nu			\\
	$\xi$			& $\Xi$ &xi			\\
	$o$				& $O$ &omicron	\\
	$\pi$			& $\Pi$ &pi			\\
	$\rho$		& $P$ &rho			\\
	$\sigma$	& $\Sigma$ &sigma		\\
	$\tau$		& $T$ &tau			\\
	$\upsilon$& $\Upsilon$ &upsilon	\\
	$\chi$ & $X$ & chi \\
	$\phi,\varphi$		& $\Phi$ &phi			\\
	$\psi$ & $\Psi$ & psi \\
	$\omega$	& $\Omega$ &omega		\\
\end{tabular}
\end{multicols}

\centrado{General}
	\begin{tabular}{ccc}
		Símbolo & Significado & Definido en la página \\
		\textasteriskcentered &  Indica material opcional en curso de 1 término &\\
		$\blacksquare$	& fin de la prueba &\\
		$\square$		& fin del ejemplo u observación &\\
		$\left[\cdots\right]$ & referencia al ítem en bibliografía &\\
	\end{tabular}
\centrado{Lógica}
\begin{tabular}{ccc}
		Símbolo & Significado & Definido en la página \\
		$\wedge,\vee$ & y, o & \\
		$\implies$ & implica & \\
		$\impliedby$ & la recíproca de $\implies$ & \\
		$\iff$ & si y solo si (sii) & \\
		$\sim$ & no & \\
		$\equiv$ & es lógicamente equivalente a & \\
		$\forall x$ & para todo $x$ & \\
		$\exists x\ni$ & existe un $x$ tal que & \\
\end{tabular}
\centrado{Conjuntos}
\begin{tabular}{ccc}
		Símbolo & Significado & Definido en la página \\
		$\in$  & pertenece a (es miembro de) & \\
		$\left\{a,b,c,\ldots\right\}$ & conjunto que contiene $a,b,c\ldots$ &\\
		$\left\{x:P(x)\right\}$ & conjunto de todos los $x$ tal que $P(x)$ & \\
		$\cup,\cap$ & unión, intersección & \\
		$A^{\complement}$ & complemento de $A$ & \\
		$B\setminus A$ & complemento de $A$ en $B$ & \\
		$\mathcal{U}$ & el conjunto universal & \\
		$\emptyset$ & el conjunto vacío & \\
		$\subseteq$ & es un subconjunto de & \\
		$\left\{A_{\lambda}:\lambda\in\Lambda\right\}$ & familia de conjuntos $A_{\lambda}$, indexados por $\lambda\in\Lambda$ \\
		$\bigcup_{\lambda\in\Lambda}A_{\lambda}$ & unión de conjuntos $A_{\lambda},\lambda\in\Lambda$ \\
		$\bigcap_{\lambda\in\Lambda}A_{\lambda}$ & intersección de conjuntos $A_{\lambda},\lambda\in\Lambda$ \\
		$A\simeq B$ & $A$ y $B$ son conjuntos equivalentes \\
		$x+A$ & $\left\{x+a:a\in A\right\}$ & \\
		$xA$ & $\left\{xa:a\in A\right\}$ & \\
		$-A$ & $\left\{-a:a\in A\right\}$ & \\
		$A+B$ & $\left\{a+b:a\in A, b\in B\right\}$ & \\
\end{tabular}

\centrado{Funciones}
\begin{tabular}{ccc}
		Símbolo & Significado & Definido en la página \\
		$f\colon A\rightarrow B$ & f es una función de $A$ a $B$ & \\
		$\mathcal{D}(f),\mathcal{R}(f)$ & dominio de $f$, rango de $f$ \\
		$f(C)$ & $\left\{f(x):x\in C\right\}$ & \\
		$\mathcal{F}\left(\mathcal{S},\mathbb{R}\right)$ & $\left\{\text{todas las funciones }f\colon\mathcal{S}\rightarrow\mathbb{R}\right\}$ & \\
		$f\pm g,rf,fg,f/g$ & combinaciones algebraicas de $f$ y $g$ & \\
		$|f|$ & valor absoluto de una función & \\
		$\min\left\{f,g\right\}\max\left\{f,g\right\}$ & mínimo (máximo) de $f$ y $g$ & \\
		$g\circ f$ & compuesta de $f$ y $g$ & \\
		$i_{A}$ & función identidad en $A$ & \\
		$f^{-1}$ & función inversa de $f$ & \\
\end{tabular}
\centrado{El sistema de los números reales}
\begin{tabular}{ccc}
	Símbolo & Significado & Definido en la página \\
	$x^{-1}$ & inverso multiplicativo de $x$ & \\
	$\mathcal{P}$ & conjunto de todos los elementos positivos de un cuerpo ordenado & \\
	$<,>,\leq,\geq$ & menor que, mayor que, etc. & \\
	$|x|$ & valor absoluto de $x$ & \\
	$\left[a,b\right],\left(a,b\right)$, etc. & intervalos (acotados) & \\
	$\left(-\infty,a\right),\left[b,+\infty\right)$, etc. & intervalos (no acotados) & \\
	$\mathbb{N}_{F}$ & conjunto de los números naturales de un cuerpo ordenado  $F$ & \\
	$n!$ & $n$ factorial & \\
	$\binom{n}{k}$ & coeficiente binomial, para $0\leq k\leq n\in\mathbb{N}$ & \\
	$\mathbb{Z}_{F}$ & conjunto de los números enteros de un cuerpo ordenado  $F$ & \\
	$\mathbb{Q}_{F}$ & conjunto de los números racionales de un cuerpo ordenado  $F$ & \\
	$\mathbb{N},\mathbb{Z},\mathbb{Q}$ & números naturales, enteros, racionales  $F$ & \\
	$\min A, \max A$ & elementos mínimo y máximo de $A$ & \\
	$\sup A$ & menor cota superior de $A$ &\\
	$\inf A$ & mayor cota inferior de $A$ &\\
	$\mathbb{R}$ & conjunto de todos los números reales & \\
	$+\infty,+\infty$ & supremo o ínfimo de conjuntos no acotados \\
	$e$ & $\lim\limits_{n\to\infty}{\left(1+1/n\right)}^{n}$, frecuentemente llamado número de Euler & \\
	$\pi$ & $2\sin^{-1}1$ & \\
	$\gamma$ & Constante de Euler & \\
\end{tabular}
\centrado{Sucesiones}
\begin{tabular}{ccc}
	Símbolo & Significado & Definido en la página \\
	$\left\{x_{n}\right\}$ & una sucesión de números reales & \\
	$\lim\limits_{n\to\infty}x_{n}=L$ & La sucesión $\left\{x_{n}\right\}$ tiene límite $L$. & \\
	$x_{n}\rightarrow L$ & La sucesión $\left\{x_{n}\right\}$ converge a $L$. &\\
	$T_{m}$ & la $m$--cola de una sucesión $\left\{x_{n}\right\}$ & \\
	$\lim\limits_{n\to\infty}x_{n}\pm\infty$ & La sucesión $\left\{x_{n}\right\}$ tiene límite $+\infty$ o $-\infty$. &\\
	$x_{n}\rightarrow\pm\infty$ & La sucesión $\left\{x_{n}\right\}$ diverge a $+\infty$ o $-\infty$. &\\
	$\underline{\lim\limits}_{n\to\infty}x_{n},\overline{\lim\limits}_{n\to\infty}x_{n}$ & límite inferior o superior de $\left\{x_{n}\right\}$. & \\
\end{tabular}
\centrado{Topología de $\mathbb{R}$}
\begin{tabular}{ccc}
	Símbolo & Significado & Definido en la página \\
	$N_{\varepsilon}(x)$ & $\varepsilon$--vecindad de $x$ & \\
	$A^{\circ}, A^{\text{ext}},A^{\text{b}}$ & interior, exterior, y frontera de $A$ & \\
	$\overline{A},A^{\text{cl}}$ & clausura de $A$ & \\
	$A^{\prime}$ & conjunto de todos los puntos de acumulación de $A$ & \\
	$d(A)$ & diámetro de $A$ & \\
	$\mu(A)$ & medida de $A$ & \\
	$\mathcal{M}$ & clase de todos los conjuntos $\mu$--medibles & \\
\end{tabular}
\centrado{Límite de funciones}
\begin{tabular}{ccc}
	Símbolo & Significado & Definido en la página \\
	$\lim\limits_{x\to x_{0}}f(x)=L$ & $f$ tiene límite $L$ a medida que $x$ se acerca a $x_{0}$. & \\
	$N^{\prime}_{\varepsilon}(x_{0})$ & $\varepsilon$--vecindad aniquilada de $x_{0}$ & \\
	$\lim\limits_{x\to x^{+}_{0}}f(x)$ o $f\left(x^{+}_{0}\right)$ & límite de $f$ a medida que $x$ se acerca a $x_{0}$ por la derecha. & \\
	$\lim\limits_{x\to x^{-}_{0}}f(x)$ o $f\left(x^{-}_{0}\right)$ & límite de $f$ a medida que $x$ se acerca a $x_{0}$ por la izquierda. & \\
	$\lim\limits_{x\to x_{0}}f(x)=+\infty$ & $f$ tiene límite $+\infty$ a medida que $x$ se acerca a $x_{0}$. & \\
	$\lim\limits_{x\to x_{0}}f(x)=-\infty$ & $f$ tiene límite $-\infty$ a medida que $x$ se acerca a $x_{0}$. & \\
	$\lim\limits_{x\to+\infty}f(x)=L$ & $f$ tiene límite $L$ a medida que $x$ se acerca a $+\in{A}$. & \\
	$\lim\limits_{x\to-\infty}f(x)=L$ & $f$ tiene límite $L$ a medida que $x$ se acerca a $-\infty$. & \\
\end{tabular}
\centrado{Funciones continuas}
\begin{tabular}{ccc}
	Símbolo & Significado & Definido en la página \\
	$\operatorname{sgn}(x)$ & función signo & \\
	$T(x)$ & función de Tom\ae & \\
	$\lfloor x\rfloor$ & función máximo entero (piso) & \\
	$\xi_{A}(x)$ & función característica de (el conjunto) $A$ & \\
	$f{\left.\right|}_{A}$ & $f$ restringido al conjunto $A$ & \\
	$\sqrt[n]{x}$ & única raíz $n$--ésima no negativa de $x\geq0$& \\
	$\varphi$ & función de Cantor & \\
	$a^{x}$ & $a^{x}$ para $a>1$ y $x\in\mathbb{R}$ & \\
	$x^{t}$ & $x^{t}$ para $x\in\mathbb{R}$ y $t>0$ & \\
	$\log_{1}x$ & $\log_{a}x$ para $a,x>0$ & \\
	$\Psi_{f}(A)$ & oscilación de $f$ en el conjunto $A$ & \\
	$\Psi_{f}(x)$ & oscilación de $f$ en $x$ & \\
	$F_{\sigma}$--set & una unión numerable de conjuntos cerrados & \\
\end{tabular}

\centrado{Funciones diferenciables}
\begin{tabular}{ccc}
	Símbolo & Significado & Definido en la página \\
	$f^{\prime}(x_{0})$ & derivada de $f$ en $x_{0}$ & \\
	$f^{\prime}_{-}(x_{0}),f^{\prime}_{+}(x_{0})$ & derivada de $f$ por la derecha (izquierda) de $x_{0}$ & \\
	$D_{x}f(x),\frac{df(x)}{dx},\frac{d}{dx}f(x)$ & notación alternativa para la derivada de $f$ & \\
	$y^{\prime},\frac{dy}{dx},\frac{d}{dx}y$ & notación alternativa para la derivada de $f$ & \\
	$f^{(k)}(x)$ & $n$--ésima derivada de $f$ en $x$ & \\
	$T_{n}((x))$ & $n$--ésimo polinomio de Taylor para $f$ & \\
	$R_{n}((x))$ & $n$--ésimo resto de Taylor para $f$ & \\
\end{tabular}

\centrado{La integral de Riemann}
\begin{tabular}{ccc}
	Símbolo & Significado & Definido en la página \\
	$\mathcal{P}$ & partición de $\left[a,b\right]$ & \\
	$m_{i}$ & $\inf\left\{f(x):x\in\left[x_{i-1},x_{i}\right]\right\}$ & \\
	$M_{i}$ & $\sup\left\{f(x):x\in\left[x_{i-1},x_{i}\right]\right\}$ & \\
	$\underline{S}\left(f,\mathcal{P}\right)=\sum_{i=1}^{n}m_{i}\triangle{i}$ & suma de Darboux inferior de $f$ sobre $\mathcal{P}$& \\
	$\overline{S}\left(f,\mathcal{P}\right)=\sum_{i=1}^{n}M_{i}\triangle{i}$ & suma de Darboux superior de $f$ sobre $\mathcal{P}$& \\
	$\underline{\int}_{a}^{b}f,\overline{\int}_{a}^{b}f$ & integrales de Darboux inferior (superior) de $f$ sobre $\left[a,b\right]$ & \\
	$\int_{a}^{b}f$ & integral de Riemann de $f$ sobre $\left[a,b\right]$ & \\
	$\|\mathcal{P}\|$ & malla de la partición $\mathcal{P}$ & \\
	$\mathcal{P}^{\ast}$ & partición etiquetada de $\left[a,b\right]$ & \\
	$R\left(f,\mathcal{P}^{\ast}\right)=\sum_{i=1}^{n}f(x^{\ast}_{i})\triangle_{i}$ & suma de Riemman de $f$ sobre la partición etiquetada $\mathcal{P}^{\ast}$ & \\
	$\mathcal{Q}_{n}$ & partición regular de $\left[a,b\right]$ dentro de $n$ subintervalos & \\
	$j\left(f,x_{0}\right)$ & salto de $f$ en $x_{0}$ & \\
	$\int_{a}^{+\infty}f,\int_{-\infty}^{b}f,\int_{-\infty}^{+\infty}f$ & integrales (impropias) de $f$ sobre intervalos infinitos & \\
\end{tabular}

\centrado{Series de números reales}
\begin{tabular}{ccc}
	Símbolo & Significado & Definido en la página \\
	$\sum_{k=1}^{\infty}a_{k} (=S)$ & una serie infinito de números con suma $S$ & \\
	$S_{n}=\sum_{k=1}^{n}a_{k}$ & $n$--ésima suma parcial de $\sum_{k=1}^{n}a_{k}$ & \\
	$a^{+}_{n},a^{-}_{n}$ & $\max\left\{a_{n},0\right\},\max\left\{-a_{n},0\right\}$ & \\
	$\vec{x}=\left(x_{1},x_{2},\ldots,x_{n}\right)$ & un $n$--vector & \\
	$\mathbb{R}^{n}$ & $n$--espacio euclidiano & \\
	$\vec{x}\cdot\vec{y}$ & producto punto de $\vec{x}$ e $\vec{y}$ & \\
	$\sum_{k=1}^{\infty}a_{k}{\left(x-c\right)}^{k}$ & una serie de potencia en $\left(x-c\right)$ & \\
	$\rho$ & radio de convergencia de una serie de potencia & \\
	$\binom{\alpha}{k}$ coeficiente binomial para un arbitrario $\alpha$, $n\in\mathbb{N}$ & \\
	$\sum_{i,j=1}^{\infty}$ & una serie doble & \\
\end{tabular}

\centrado{Sucesiones y serie de funciones}
\begin{tabular}{ccc}
	Símbolo & Significado & Definido en la página \\
	$\mathcal{F}\left(\mathcal{S},\mathbb{R}\right)$ & conjunto de todas las funciones $f\colon\mathcal{S}\rightarrow\mathbb{R}$ & \\
	$B\left(\mathcal{S}\right)$ & conjunto de todas las funcione acotadas en $\left[a,b\right]$ & \\
	$C\left(\mathcal{S}\right)$ & conjunto de todas las funciones continuas en $\left[a,b\right]$ & \\
	$D\left(\mathcal{S}\right)$ & conjunto de todas las funciones diferenciables en $\left[a,b\right]$ & \\
	$C^{k}\left(\mathcal{S}\right)$ & conjunto de todas las $f$ para cual $f^{(k)}$ es continua en $\left[a,b\right]$ & \\
	$C^{\infty}\left(\mathcal{S}\right)$ & conjunto de todas las $f\ni\forall k\in\mathbb{N},f^{(k)}$ es continua en $\left[a,b\right]$ & \\
	$R\left[a,b\right]$ & conjunto de todas las $f$ que son Riemann integrables en $\left[a,b\right]$ & \\
	$\left\{f_{n}\right\}$ & una sucesión de funciones & \\
	$\lim\limits_{n\to\infty}f_{n}=f$ & $\left\{f_{n}\right\}$ converge puntualmente a $f$ & \\
	$f_{n}\rightarrow f$ & $\left\{f_{n}\right\}$ converge puntualmente a $f$ & \\
	$\|f\|$ & normal del supremo de $f$ & \\
	$d\left(f,g\right)$ & $\|f-g\|$, la distancia entre $f$ y $g$ & \\
	$\zeta(x)$ & función zeta de Riemann & \\
	$P\left[a,b\right]$ & conjunto de todos los polinomios en $\left[a,b\right]$ & \\
	$CAP\left[a,b\right]$ & todas las $f$ continuas aproximable por polinomios en $\left[a,b\right]$ & \\
	${\left(x-c\right)}^{+}$ & $\max\left\{0,x-c\right\}$ & \\
\end{tabular}
\end{document}