\documentclass[11pt,a4paper]{article}
\usepackage[utf8]{inputenc}
\usepackage[spanish]{babel}
\usepackage{amsmath}
\usepackage{amsfonts}
\usepackage{amssymb}
\usepackage{multicol}
\usepackage[left=2.00cm, right=2.00cm, top=2.00cm, bottom=2.00cm]{geometry}
\usepackage{graphicx}
\usepackage[pdftex]{color}
\newcommand{\ds}{\displaystyle}
\newtheorem{teo}{Teorema}{}
\begin{document}
\pagestyle{empty}
\begin{minipage}[c]{10cm}
\end{minipage}
\begin{center}
\textbf{Sucesiones y Series de Recurrencia.}
\end{center}
\vspace*{-0.1cm}
\hrule
\vspace*{0.1cm}
\rule{17cm}{1mm}
\vspace{0.2cm}

\textit{Resolver una ecuación de recurrencia} significa encontrar una secuencia que satisfaga las ecuación de recurrencia. Encontrar una "solución general" significa encontrar una fórmula que describe todas las soluciones posibles (todas las secuencias posibles que satisfacen la ecuación).\\
Veamos el siguiente ejemplo:\\
Considere $T_{n}$ que satisface la siguiente ecuación para todo $n \in \mathbb{N}$; $n>1$ : 
\begin{center}
$T_{n} = 2T_{n-1} + 1 $ 
\end{center}

La ecuación de recurrencia $T_{n}$ indica cómo continúa la sucesión pero no nos dice como empieza tal.
\newline
si $T_{1} = 1$,\hspace*{1cm} se tiene $T = (1,7,3,15,31,...)$.\\
si $T_{2} = 1$,\hspace*{1cm} se tiene $T = (2,5,11,23,47,...)$.\\
si $T_{4} = 1$,\hspace*{1cm} se tiene $T = (4,9,19,39,79,...)$.\\
si $T_{-1}= 1$,\hspace*{1cm} se tiene $T = (-1,-1,-1,1-1,-1,...)$.\\

 ¿Hay formula para cada una de estas sucesiones? ¿Existe una fórmula(tal vez con n y el valor de $T_{1}$) que describa todos los terminos de la sucesión?¿existe una posible solución para $T_{n}$?\\Para poder responder este tipo de problemas, veamos un poco más de ecuaciónes con recurrencia.

\section{Ejemplos definidos por ecuaciones de recurrencia.}
\subsection{ejemplo 1.[Desajustes]}
Imagina una fiesta donde las parejas llegan juntas, pero al final de la noche, cada persona se va con una nueva pareja. Para cada $n \in P$, digamos que $D_{n}$ es el número de diferentes formas en que las parejas pueden ser "trastornadas", es decir, reorganizadas en parejas, por lo que ni uno está emparejado con la persona con la que llegaron.\\

Para los valores :\newline
$D_{1} = 0$ \hspace*{0.5cm} // una pareja no puede ser transtornada.\\
$D_{2} = 1$ \hspace*{0.5cm} // $\exists$ una y solo una manera de transtornar una pareja.\\ 
$D_{3} = 2$ \hspace*{0.5cm} // si las parejas llegan como $Aa$, $Bb$, $Cc$, entonces $A$ estaria emparejado con $b$ o $c$. Si $A$ esta emparejado con $b$, $C$ debe estar emparejado con $a$(y no $c$) y $B$ con $c$. Si $A$ esta emparejado con $c$, $B$ no debe estar emparejado con $a$(y no $b$) y $C$ con $b$.\\ \newline
¿Qué tan grandes son $D_{4}$, $D_{5}$ y $D_{10}$? ¿Cómo podemos calcularlos? ¿Hay una fórmula?\\

Vamos a desarrollar una estrategia para contar los desajustes cuando $n \leq 4$. Supongamos que hay $n$ mujeres $A_{1}, A_{2}, A_{3},...,A_{n}$, y cada $A_{j}$ llega con el hombre $a_{j}$.\\

La mujer $A_{1}$ puede ser "re-emparejada" con cualquiera de los $n-1$ hombre restantes $a_{2}$ o $a_{3}$ o ...o $a_{n}$; digamos que esta emparejada con $a_{k}$, donde $2\leq k \leq n.$ y ahora consideremos $a_{k}'$s pareja original de la mujer $A_{k}$: ella podria tomar $a_{1}$ o ella podria rechazar $a_{1}$ y tomar a alhuien más.\\

si $A_{1}$ es pareja con $a_{k}$ y $A_{k}$ es pareja con $a_{1}$, entonces $n-2$ parejas dejaron para transtornar, y eso puede hacerce exactamente de $D_{n-2}$ maneras diferente.

Ahora para cada uno de los $n-1$ hombres que $A_{1}$ podria elegir, hay $\{D_{n-2}+ D_{n-1}\}$ diferentes formas de completar el trastorno. Por lo tanto, cuando $n \geq 4$ tenemos :


$$ D_{n} = (n-1)\{ D_{n-2} + D_{n-1}\} \hspace*{2cm} (1_1) $$ 
Usando la ecuación $(1_1)$ las evaluaciones para 1 y 2 verifican la igualdad, ahora evaluemos $D_{n}$ para cualquier valor de $n$, con $n \in \mathbb{N}$\\ \newline
$D_{3} = (3 - 1)\{D_{2} + D_{1}\} = 2(1 + 0) = 2$\\
$D_{4} = (4 - 1)\{D_{3} + D_{2}\} = 3(2 + 1) = 9$\\
$D_{5} = (5 - 1)\{D_{4} + D_{3}\} = 4(9 + 2) = 44$\\
$D_{6} = (6 - 1)\{D_{5} + D_{4}\} = 5(44 + 9) = 265$\\
$D_{7} = (7 - 1)\{D_{6} + D_{5}\} = 6(265 + 44) = 1854$\\
$D_{8} = (8 - 1)\{D_{7} + D_{6}\} = 7(1854 + 265) = 14833$\\
$D_{9} = (9 - 1)\{D_{8} + D_{7}\} = 8(14833 + 1854) = 133496$\\
$D_{10}= (10 - 1)\{D_{9} + D_{8}\} = 9(133496 + 14833) = 1334961$\\

La suceción en $P$ definido por $S_{n} = Axn!$ donde $A$ es un número real satisface la ecuación de recurrencia $1_1$. Si $n \geq 3$ se tiene :
\begin{align*}
 (n-1)\{S_{n-2}+S_{n-1}\} &= (n-1)\{A(n-2)! + A(n-1)!\} \\
 						  &= (n-1)A(n-2)!\{1+(n-1)\} \\
 						  &= A(n-1)(n-2)!\{n\}\\
 						  &= A x n!\\
 						  &= S_{n}.
\end{align*}
¿Pero se aplica esta formula cuando $n=1$ o $n=2$?\\
¿Existe algun número real tal que $D_{n} = A(n!)$ cuando $n=1$ o $n=2$?\\
No, porque si $0 = D_{1} =A(1!),$ entonces $A$ debe ser igual a $0$,\\
y si $1 = D_{2} = A(2!),$ se tiene que $A$ deberia tomar el valor de $\frac{1}{2}$\\ \newline
Sin embargo, podemos usar esta fórmula para probar que $D_{n}$ es acotado.
\begin{teo}{}
para todo $n \geq 2,$ $\ds(\frac{1}{3})n! \leq D_{n} \leq (\frac{1}{2})n!$
\end{teo}

\subsection{Ejemplo 2.[Números de ackermann]}
Por los 1920s, un lógico y matemático alemán, Wilhelm Ackermann
(1896–1962), inventó una función muy curiosa, $\ds A:P X P \longrightarrow P,$ que define recursivamente usando "tres reglas":\newline
Regla 1.\hspace*{0.2cm} $\ds A(1,n)=2$ \hspace*{0.3cm} para $n = 1,2,...,$\\
Regla 2.\hspace*{0.2cm} $\ds A(m,1)=2m$ \hspace*{0.3cm} para $m =2,3,...,$\\
Regla 3.\hspace*{0.2cm} cuando $m>1$ y $n>1$ se tiene: $\ds A(m,n) = A(A(m-1,n),n-1).$

\begin{align*}
Entonces \hspace*{0.3cm} A(2,2) &= A(A(2-1,2),2-1) \hspace*{0.3cm} //regla~3\\
							    &= A(A(1,2),1)\\
							    &= A(2,1) \hspace*{0.3cm} //regla~ 1\\
							    &= 2(2) \hspace*{0.3cm} //regla~2\\
							    &= 4.
\intertext{ademas}
A(2,3) &= A(A(2-1,3),3-1) \hspace*{0.3cm} //regla~3\\
							    &= A(A(1,3),2)\\
							    &= A(2,2) \hspace*{0.3cm} //regla 1\\
							    &= 4. \hspace*{0.3cm}
\intertext{De hecho}
si~A(2,k) &= 4, \hspace*{3cm} // para~algun~ k \geq 2\\
entonces~ A(2,k+1) &= A(A(2-1,k+1), (k+1)-1) \hspace*{3cm} // regla~3 \\
				   &=A(A(1,k+1),k)\\
				   &=A(2,k)	\hspace*{2,3cm} //regla~1 \\
				   &=4.	\hspace*{3cm} // nuestro~supuesto
\intertext{Así, tenemos por Inducción Matemática:}
A(2,n) &= 4, ~para~todo ~n\geq 1.        
\end{align*}
Hasta ahora la tabla de los números de Ackermann se ve así:
\begin{equation*}
\begin{tabular}{ c| c| c| c| c| c| c| c| c| r }
     A & n=1 & n=2 & 3 & 4 & 5 & 6 & 7 & 8 & 9... \\
   \hline
   m=1 & 2 & 2 & 2 & 2 & 2 & 2 & 2 & 2 & 2 \\
   \hline
   m=2 & 4 & 4 & 4 & 4 & 4 & 4 & 4 & 4 & 4 \\
   \hline
   3   & 6 &  &  &  &  &  &  &  &  \\
   \hline
   4   & 8 &  &  &  &  &  &  &  &  \\
   \hline
   5  & 10 &  &  &  &  &  &  &  &  \\
\end{tabular}
\end{equation*}
Observamos que la segunda fila es de puro 4s. ¿pero como es la segunda columna?\\
\begin{align*}
Si \hspace*{0.3cm} A(k,2) &= 2^{k} \hspace*{0.3cm} //~para~algunos~k \geq 2 \\
se~tiene			      &= A(A([k+1],2-1) \hspace*{0.3cm}//~regla~ 3\\
						  &= A(A(k,2),1) \hspace*{0.3cm}\\
				          &= A(2^{k},1)\hspace*{0.3cm} //~nuestro~supuesto \\
				   	      &= 2(2^{k}) \hspace*{0.3cm} //~regla~2\\
				   	      &= 2^{k+1}.
\intertext{ademas}
A(2,3) &= A(A(2-1,3),3-1) \hspace*{0.3cm} //regla~3\\
							    &= A(A(1,3),2)\\
							    &= A(2,2) \hspace*{0.3cm} //regla 1\\
							    &= 4. \hspace*{0.3cm}\\
Asi~se~ tiene:~ A(m,2)=2^{m} ~para~todo~ m ~\geq ~1. \\
\intertext{Ahora,~¿como~so~los~otros~valores?}
            A(3,3) &= A(A(3-1),3),3-1) \hspace*{1cm} // Regla~3\\
                   &= A(A(2,3), 2) \hspace*{2cm} \\
				   &=A(4,2)	\hspace*{2,3cm} // Segunda~fila \\
				   &=2^{4} \hspace*{3cm} // segunda~columna\\
				   &=16.\\
 A(4,3) &= A(A(4-1),3),3-1) \hspace*{1cm} // Regla~3\\
                   &= A(A(3,3), 2) \hspace*{2cm} \\
				   &=A(16,2)	\hspace*{2,3cm} // Encima\\
				   &=2^{16} \hspace*{3cm} // Segunda~columna\\
				   &=65536.\\
 A(3,4) &= A(A(3-1),3),4-1) \hspace*{1cm} // Regla~3\\
                   &= A(A(2,4), 3) \hspace*{2cm} \\
				   &=A(4,3)	\hspace*{2,3cm} // Segunda~fila\\
				   &=65536.\\				   			   
\intertext{¿Cual es el valor de A(4,4)?  
¿Podría ejecutar un programa recursivo simple para evaluar A (4,4)?}
 A(5,3) &= A(A(5-1),3),3-1) \hspace*{1cm} // Regla~3\\
                   &= A(A(4,3), 2) \hspace*{2cm} \\
				   &=A(65536,2)	\hspace*{2,3cm}\\
				   &=2^{65536}.\hspace*{3cm}// Segunda~columna\\
				   &=n~(n~ grande~aprox~20000~digitos~en~base~10.)\\
\intertext{hasta ahora tenemos:}
\end{align*}
\begin{equation*}
\begin{tabular}{ c| c| c| c| c| c| c| c| c| r }
     A & n=1 & n=2 & 3 & 4 & 5 & 6 & 7 & 8 & 9... \\
   \hline
   m=1 & 2 & 2 & 2 & 2 & 2 & 2 & 2 & 2 & 2 \\
   \hline
   m=2 & 4 & 4 & 4 & 4 & 4 & 4 & 4 & 4 & 4 \\
   \hline
   3   & 6 & 8 & 16 & 65536 & ? &  &  &  &  \\
   \hline
   4   & 8 & 16 & 65536 & ? &  &  &  &  &  \\
   \hline
   5  & 10 & 32 & $2^{65536}$ &  &  &  &  &  &  \\
\end{tabular}
\end{equation*}
¿Cómo continúa la tercera columna?\\
sea $2\uparrow$ denota el valor de "Torre" de k 2's, definida recursivamente por 
$$2\uparrow 1= 2;~ y~ para~ k ~\geq 1,~ 2\uparrow [k+1]= 2^{2\uparrow k}.$$
Pero este es un número tan grande que nunca podría escribirse en dígitos decimales, incluso utilizando todo el papel del mundo, Su valor nunca podria ser calculado. Ahora nos preguntamos ¿Los números Ackermann son “computables”? Por otro lado, supongamos que las secuencias que encontramos, incluso aquellas definidas por ecuaciones de recurrencia, serán fáciles para entender y tratar.
\vfill \textbf{Alumno} Micha Velasque Junior \footnote{Hecho en \LaTeX}.
\hfill 27 de abril
\end{document}
