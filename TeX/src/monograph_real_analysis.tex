% arara: clean: {
% arara: --> extensions:
% arara: --> ['log','idx','ilg','ind','out','bbl','blg','thm','toc','aux','synctex.gz','bcf','run.xml','contb',
% arara: --> 'contg','contn','diifb','diifg','diffn','funb','fung','funn','genb','geng','genn','intb','intg',
% arara: --> 'intn','limb','limg','limn','logb','logg','logn','realb','realg','realn','seqb','seqg','seqn',
% arara: --> 'serb','serg','sern','setb','setg','setn','ssfunb','ssfung','ssfunn','topb','topg','topn','pdf']
% arara: --> }
% arara: lualatex: { draft: yes }
% arara: biber
% arara: lualatex: { draft: yes }
% arara: makeindex
% arara: lualatex
% arara: pythontex
% arara: lualatex: { draft: yes }
% arara: makeglossaries
% arara: lualatex: {
% arara: --> shell: yes,
% arara: --> synctex: yes,
% arara: --> draft: yes,
% arara: --> interaction: batchpmode
% arara: --> }
% arara: clean: {
% arara: --> extensions:
% arara: --> ['log','idx','ilg','ind','out','bbl','blg','thm','toc','aux','synctex.gz','bcf','run.xml','contb',
% arara: --> 'contg','contn','diifb','diifg','diffn','funb','fung','funn','genb','geng','genn','intb','intg',
% arara: --> 'intn','limb','limg','limn','logb','logg','logn','realb','realg','realn','seqb','seqg','seqn',
% arara: --> 'serb','serg','sern','setb','setg','setn','ssfunb','ssfung','ssfunn','topb','topg','topn','pdf']
% arara: --> }
% arara: --> if missing('pdf') || changed('tex')
\documentclass[graybox,envcountchap,sectrefs]{svmono}
% choose options for [] as required from the list
% in the Reference Guide

%\usepackage{mathptmx}
%\usepackage{helvet}
%\usepackage{courier}
\usepackage{type1cm}

\usepackage{makeidx}% allows index generation
															% when including figure files
\usepackage{multicol}        % used for the two-column index
\usepackage[bottom]{footmisc}% places footnotes at page bottom
\usepackage{amssymb,mathtools}
\usepackage{newtxtext}
%\usepackage{newtxmath}       % selects Times Roman as basic font
\usepackage[lite]{mtpro2}

\usepackage{bm,dsfont}
\usepackage{mathrsfs}
\usepackage{lipsum}
\usepackage[shortlabels]{enumitem}
\usepackage{graphicx}        % standard LaTeX graphics tool
\graphicspath{ {img} }
%\usepackage{subfiles}
%\subfile{sub/sub.tex}
\usepackage{morewrites}
\usepackage[toc,
nonumberlist,
stylemods={longbooktabs},
nomain
]{glossaries-extra}
\newglossary[geng]{general}{genb}{genn}{General}
\newglossary[logg]{logic}{logb}{logn}{Lógica}
\newglossary[setg]{sets}{setb}{setn}{Conjuntos}
\newglossary[fung]{functions}{funb}{funn}{Funciones}
\newglossary[realg]{realnumbers}{realb}{realn}{El sistema de los números reales}
\newglossary[seqg]{sequences}{seqb}{seqn}{Sucesiones}
\newglossary[topg]{topology}{topb}{topn}{Topología de $\mathds{R}$}
\newglossary[limg]{limits}{limb}{limn}{Límite de funciones}
\newglossary[contg]{continuous}{contb}{contn}{Funciones continuas}
\newglossary[diffg]{differentiable}{diffb}{diffn}{Funciones diferenciables}
\newglossary[intg]{riemannintegral}{intb}{intn}{La integral de Riemann}
\newglossary[serg]{series}{serb}{sern}{Series de números reales}
\newglossary[ssfung]{sequencesfunctions}{ssfunb}{ssfunn}{Sucesiones y series de funciones}
\loadglsentries{abbreviations}

\newglossarystyle{symbunitlong}{
	\setglossarystyle{long3col-booktabs}
	\renewenvironment{theglossary}
	{\begin{longtable}{lp{\glsdescwidth}c>{\centering}rp{\glspagelistwidth}}}%
		{\end{longtable}}
	\renewcommand*{\glossaryheader}{
		\bfseries Símbolo & %\descriptionname
		\bfseries Significado &
		\bfseries Definido en la página
		\tabularnewline\midrule\endhead}
}

\usepackage[citestyle=numeric,style=numeric,backend=biber]{biblatex}
\addbibresource{bib.bib}

\usepackage{pythontex}
\usepackage{etoolbox,ifluatex}
\ifluatex
\makeatletter
\patchcmd{\PEX@}{\dp\Pbox@>\dp\z@}{\ht\Pbox@>\dp\z@}{}{}
\patchcmd{\SQEX@}{\dp\Sbox@>\dp0}{\ht\Sbox@>\dp0}{}{}
\makeatother
\fi

\DeclareMathOperator{\sen}{sen}
\newcommand{\divides}{\mid}
\newcommand{\notdivides}{\nmid}

\date{4 de junio del 2019}

\makeindex             % used for the subject index
                       % please use the style svind.ist with
                       % your makeindex program
%\let\claim\relax
%\renewcommand\claimname{Aa}

\makeatletter
\let\addtocontents\newfloat@addtocontents@ORI
\makeatother

\makeglossaries
\begin{document}

\author{
	Carlos A. Aznarán Laos\\
	Franss Cruz Ordoñez\\
	Junior Micha Velasque\\
	Gabriel Quiróz Gómez\\
	Davis S. García Fernández
}
\title{Relación de recurrencia}
\subtitle{Ecuaciones en diferencias y análisis en escalas de tiempo}
\maketitle

\frontmatter
%\extrachap{A}
%\Extrachap{B}
\begin{dedication}
La presente monografía está dedicada a mis profesores y estudiantes de la Facultad de Ciencias.
\end{dedication}
\foreword
El problema \textsc{isoperimétrico}, un problema clásico visto en la antigüedad, lo que no necesita mucho de conocimiento matemático, en el que la proposición consiste: Dado
un número real positivo $L$, se trata de estudiar la cuestión siguiente: de todas las curvas cerradas del plano de longitud dada $L>0$, ¿cuál es la que encierra mayor área?
Esta cuestión puede ser respondida inclusive por personas que cursan el menor grado, en el que su intuición les indica que es la \emph{circunferencia}, la curva que encierra el mayor área y se sabe que la respuesta es $\tfrac{L^2}{4\pi}$. Pues ahí todo bien, la respuesta no era nada complicada. Pero el problema consiste en cómo se demostraría con matemáticas que la circunferencia maximiza el área, pues si se puede formar infinitas curvas con perímetro $L$. La materia de estudio del \textsc{cálculo variacional} consiste en buscar máximos y mínimos (o más generalmente extremos relativos) de funcionales continuos definidos sobre algún espacio funcional. También constituyen una generalización del cálculo elemental de máximos y mínimos de funciones reales de una variable.\par
\vspace{\baselineskip}
\begin{flushright}\noindent
Rímac, abril 2019\hfill {\it El profesor del curso}
\end{flushright}
\preface
Uno de los temas más importantes dentro del \emph{Análisis Matemático} son las sucesiones, es decir, funciones cuyo dominio y contradominio es el conjunto de los números naturales $\mathds{N}$ y el de los números reales $\mathds{R}$, respectivamente. En el presente trabajo nos enfocaremos en nada menos que las ``relaciones de recurrencia'', donde cualquier  término se determina en función de al menos uno de los términos precedentes, un ejemplo famoso es la \emph{sucesión de Fibonacci}. Esta sucesión fue descrita por \emph{Leonardo de Pisa}\footnote{Fibonacci} como la solución a un problema de cría de conejos:
%Paragraph or quotation
``Cierta persona cría una pareja de conejos juntos en un lugar cerrado y desea saber cuántos son creados a partir de este par en un año cuando, de acuerdo a su naturaleza, cada pareja necesita un mes para envejecer y cada mes posterior procrea otra pareja''.\\[\baselineskip]

Viendo esto, hemos concebido un modelo matemático basado en sucesiones recursivas, dando su definición, algunos otros ejemplos, su relación con las ecuaciones en diferencias y otras aplicaciones como resolver sistemas de ecuaciones lineales empleando nuestros conocimientos adquiridos en el curso de Análisis Real de la carrera de Matemática en la Universidad
Nacional de Ingeniería.
\vspace{\baselineskip}
\begin{flushright}\noindent
Rímac,\hfill {\it Carlos Aznarán Laos}\\
mayo 2019\hfill {\it Franss Cruz Ordoñez}\\
\end{flushright}
\extrachap{Agradecimientos}
Nos gustaría expresar el agradecimiento especial al maestro Manuel Toribio Cangana, así como a nuestro profesor Benito Ostos, que nos brindó la excelente oportunidad de hacer este maravilloso proyecto sobre el tema de cálculo de variaciones, quien también me ayudó a hacer mucha investigación y llegué a conocer a muchos cosas nuevas. Estoy muy agradecido con ellos. En segundo lugar, también me gustaría agradecer a mis padres y amigos que me ayudaron mucho a terminar este proyecto en un tiempo limitado.\par

\

Estoy haciendo este proyecto no solo para las notas sino también para aumentar nuestro conocimiento.

\tableofcontents

\extrachap{Acrónimos}

\begin{description}[CABR]
	\item[CAS]{Sistema Computarizado Algebraico}
\end{description}

\mainmatter
%%%%%%%%%%%%%%%%%%%%%%%%%%%%%%%%%%%%%%%% Primera parte %%%%%%%%%%%%%%%%%%%%%%%%%%%%%%%%%%%%%%%%
\begin{partbacktext}
\part{Fundamentos}
En la primera parte de la monografía presentaremos los conceptos fundamentales para modelar y simular los problemas de las ecuaciones en diferencias, a veces, mal llamado \emph{relaciones de recurrencias}. En el capítulo $1$ presentaremos los modelos fundamentales y las ecuaciones en diferencias. Discutiremos la relación de \emph{Ackermann}
\end{partbacktext}
%\chapauthor{Autor}
%\chapsubtitle{Subtítulo}
\chapter{Introducción}
\abstract{En este capítulo introducimos un tipo de funciones llamadas que pueden ser usados para aproximar otras funciones más generales}
%\chaptermark{xd}
\section{Relación de Recurrencia}
En esta sección presentamos a nuestros lectores las nociones básicas subyacentes de las relaciones de recurrencia, así como varios ejemplos de tales relaciones.

Una relación de recurrencia es una familia numerable de ecuaciones que definen sucesiones en modo recursivo. Aquellas sucesiones que así surgen se llaman \emph{soluciones de la recurrencia}, dependiendo de uno o más valores iniciales: cada término que sigue al valor inicial en tales sucesiones es definida como una función de los términos anteriores.

\begin{definition}[Relación de recurrencia]\index{Relación de recurrencia!definición}
	Una \textbf{relación de recurrencia} en las incógnitas $x_{i}$, $i\in\mathds{N}$, es una familia de ecuaciones
	\begin{equation*}
	x_{n}=f_{n}\left(x_{0},\ldots,x_{n-1}\right),\quad n\geq r,
	\end{equation*}
	donde $r\in\mathds{N}_{\geq1}$, y ${\left(f_{n}\right)}_{n\geq r}$ son funciones
	\begin{equation*}
	f_{n}\colon D_{n}\rightarrow\mathds{R},\quad D_{n}\subseteq\mathds{R}^{n},\qquad\text{o}\qquad f_{n}\colon D_{n}\rightarrow\mathds{C},\quad D_{n}\subseteq\mathds{C}^{n}.
	\end{equation*}
	Dependiendo del caso, las llamaremos \textbf{recurrencias reales}\index{Relación de recurrencia!real} o \textbf{recurrencias complejas}\index{Relación de recurrencia!compleja}. Las incógnitas $x_{0},\ldots,x_{r-1}$ son llamadas \textbf{libres}. El número $r$ es el \emph{orden de la relación}\index{Relación de recurrencia!orden}.
	
	Al reemplazar $n$ por $n+r$, la relación de recurrencia de orden $r$
	\begin{equation*}
	x_{n}=f_{n}\left(x_{0},\ldots,x_{n-1}\right),\quad n\geq r,
	\end{equation*}
	puede también escribirse como
	\begin{equation*}
	x_{n+r}=f_{n+r}\left(x_{0},\ldots,x_{n+r-1}\right),\quad n\geq0.
	\end{equation*}
\end{definition}

\begin{definition}[Solución de una recurrencia]\index{Relación de recurrencia!solución}
	Una sucesión ${\left(a_{n}\right)}_{n\in\mathds{N}}$ es una \textbf{solución} de la relación de recurrencia de orden $r$
	\begin{equation}
	x_{n}=f_{n}\left(x_{0},\ldots,x_{n-1}\right),\quad n\geq r,
	\end{equation}
	con $f_{n}\colon D_{n}\rightarrow\mathds{R}$, $D_{n}\in\mathds{R}^{n}$, si
	\begin{equation*}
	\left(a_{0},\ldots,a_{n-1}\right)\in D_{n},\quad a_{n}=f_{n}\left(a_{0},a_{1},\ldots,a_{n-1}\right)\quad\forall\,n\geq r.
	\end{equation*}
\end{definition}

La sucesión $\left(a_{0},\ldots,a_{n-1}\right)$ de valores asignados para las $r$ incógnitas libres es llamado la $r$--sucesión de \textbf{valor inicial} o de las \textbf{condiciones iniciales} de la solución. Definimos la \textbf{solución general real} (respectivamente \textbf{compleja}) de la sucesión como la familia de todas las soluciones con elementos que están en $\mathds{R}$ (respectivamente en $\mathds{C}$).

\begin{example}{Relación de recurrencia de orden $1$}
	Considere la relación de recurrencia de primer orden definida por \[x_{n}=\frac{1}{x_{n-1}-1},\quad n\geq1.\]
\end{example}

La $1$--sucesión (2) no es una sucesión de valor inicial de una solución, en efecto, $2$ pertenece al dominio de $f_{0}\left(x\right)=\frac{1}{x-1}$, pero $\left(2,f_{0}(2)\right)=\left(2,1\right)$ no pertenece al dominio de $f_{1}\left(x_{0},x_{1}\right)=\frac{1}{x-1}$. Por otra parte, la $1$--sucesión (3) es en efecto la sucesión de valor inicial de la solución (sucesión) \[\left(a_{n}\right)_{n}\coloneqq\left(3,1/2,-2,-1/3,-3/4,-4/7,-7/11,\ldots\right).\]
Note que para $n\geq2$ uno tiene $a_{n}<0$ y así $a_{n+1}=\frac{1}{a_{n}-1}<0$ es distinto de $1$.

\begin{example}{}
	En muchas ocasiones una relación de recurrencia de orden $r$ involucra solo los últimos $r$ términos y es de la forma
	\begin{equation*}
	x_{n}=g_{n}\left(x_{n-r},\ldots,x_{n-1}\right),\quad n\geq r,
	\end{equation*}
	donde ${\left(g_{n}\right)}_{n\geq r}$ son las funciones definidas en un subconjunto $E_{n}$ de $\mathds{R}^{r}$ o $\mathds{C}^{r}$. Este último es de hecho una relación de recurrencia: es suficiente para establecer $f_{n}\left(x_{0},\ldots,x_{n-1}\right)\coloneqq g_{n}\left(x_{n-r},\ldots,x_{n-1}\right)$ para $\left(x_{0},\ldots,x_{n-1}\right)\in D_{n}\coloneqq\mathds{R}^{n-r}\times E_{n}$ (o $\mathds{C}^{n-r}\times E_{n}$) a fin de cumplir los requerimientos de la definición %9.2
\end{example}

Una relación de recurrencia es una ecuación que expresa cada término de una sucesión en función de los términos precedentes. Una relación de recurrencia presenta la siguiente forma:
\begin{align*}
u_{n}&=\varphi\left(n,u_{n-1}\right),\forall n>0,\\
\intertext{donde}
\varphi&\colon\mathds{N}\times X\rightarrow x
\end{align*}
es una función donde $X$ es un conjunto al que deben pertenecer los elementos de una sucesión. Para cualquier $u_{0}\in X$, esto define una sucesión única con $u_{0}$ como su primer elemento, llamado el valor inicial.

Es fácil modificar la definición para obtener sucesiones a partir del término del índice $1$ o superior. Esto define la relación de recurrencia de primer orden. Una relación de recurrencia de orden $k$ tiene la forma:
\begin{align*}
u_{n}&=\varphi\left(n,u_{n-1},u_{n-2},\ldots,u_{n-k}\right),\forall n\geq k,\\
\intertext{donde}
\varphi&\colon\mathds{N}\times X^{k}\rightarrow X
\end{align*}
Es una función que involucra $k$ elementos consecutivos de la sucesión. En este caso, se necesitan $k$ valores iniciales para definir una sucesión.

\subsection{Ecuaciones en diferencias}\index{Ecuación en diferencias!definición}
Una ecuación en diferencias es una expresión de la forma: \[ G\left(n,f(n),f(n+1),\ldots,f(n+k)\right)=0,\forall n\in\mathds{Z} \] donde $f$ es una función definida en $\mathds{Z}$.

Si después de simplificar esta expresión quedan los términos $f\left(n+k_{1}\right)$ y $f\left(n+k_{2}\right)$ como el mayor y el menor, respectivamente. Se dice que la ecuación es de orden $k=k_{1}-k_{2}$.
\begin{example}{Ecuación en diferencias de orden $3$}
	La ecuación:
	\begin{equation}
	5f(n+4)-4f(n+2)+f(n+1)+(n-2)^{3}=0
	\end{equation}
	es de orden $4-1=3$.
\end{example}
Una ecuación en diferencias de orden $k$ se dice que es \emph{lineal}\index{Ecuación en diferencias!lineal} si puede expresarse de la forma:
\begin{equation*}
p_{0}(n)f(n+k)+p_{1}(n)f(0+k-1)+\cdots+p_{k}(n)f(n)=g(n),
\end{equation*}
donde los coeficientes $p_{i}$ son funciones definidas en $\mathds{Z}$.

El caso más sencillo es cuando los coeficientes son constantes $p_{i}(n)=a_{i}$:
\begin{equation*}
a_{0}f(n+k)+a_{1}f(n+k-1)+\cdots+a_{k}f(n)=g(n).
\end{equation*}
La ecuación en diferencias se dice que es \emph{homogénea}\index{Ecuación en diferencias!homogénea} en el caso que $g(n)=0$, y completa en el caso contrario.

\subsection{Algunos modelos de recurrencias lineales}

Ahora damos una serie de ejemplos que ilustran cómo reducir la solución de un problema en el que la búsqueda de las soluciones de una relación de recurrencia apropiada.

\begin{example}{La escalera}
	Un niño decide escalar una escalera con $n\geq 1$ de tal manera que cada paso que él despeja uno o dos de los pasos de la escalera %(vea)
	Encuentre la relación de recurrencia que sirva para calcular el número de diferentes maneras posibles de escalar la escalera.
\end{example}
Usamos la variable desconocida $x_{n}$ para denotar el número de maneras en las cuales el niño puede escalar la escalera de $n\geq1$ pasos. Es fácil de observar que $x_{1}=1$ y $x_{2}=2$ (dos pasos cada uno de longitud uno, o un paso de longitud dos escalones). Ahora sea $n\geq3$: si con el primer paso el niño mueve solo el primer escalón; existen claramente $x_{n-1}$ posibles maneras de escalar los que quedan. Si en cambio con el primer lugar, se suben dos peldaños de escalera.
%https://rajsain.files.wordpress.com/2013/11/randomized-algorithms-motwani-and-raghavan.pdf
%
%https://www.csie.ntu.edu.tw/~r97002/temp/Concrete%20Mathematics%202e.pdf
%
%https://link.springer.com/chapter/10.1007/978-3-642-61544-3_9
%
%https://link.springer.com/chapter/10.1007/978-94-011-1814-9_9
%
%https://link.springer.com/chapter/10.1007/978-3-319-15579-1_39
%
%https://link.springer.com/chapter/10.1007/978-94-011-2058-6_14
%
%https://link.springer.com/chapter/10.1007/BFb0120904
%
%https://link.springer.com/chapter/10.1007%2FBFb0120904
%
%https://link.springer.com/article/10.1007/BF00874886
%
%https://link.springer.com/search?date-facet-mode=between&showAll=true&query=recurrence+AND+relation&facet-discipline=%22Mathematics%22

%\motto{hola}
%\runinhead{xd}
%\subruninhead{xd}
%\begin{petit}
%A
%\end{petit}
\subsection{Problemas}

\begin{exercise}
Supongamos que $E_n$ es definido recursivamente en $\mathds{Z}^+$ por \[ E_0=0,E_1=2,\text{ y },E_{n+1}=2n\{E_n+E_{n-1}\} \text{ para }n\geq 1. \] Determine el valor de $E_{10}$.
\end{exercise}

\begin{solution}

\end{solution}

\begin{exercise}
Supongamos que la función $f$ es definida recursivamente en $\mathds{Z}^+$ por \[ f(n)\coloneq \ccases{1 & \text{si }n=2^k\text{ para algún }k \in \mathds{N}.\\ f(n/2) & \text{si }n\text{ es par pero no una potencia de 2} \\ f(3n+1) & \text{si }n\text{ es impar.}}. \] Entonces
	\begin{alignat*}{2}
f(3)	=&f(10)	&&\qquad\text{ porque }3\text{ es impar}\\
			=&f(5)	&&\qquad\text{ porque } 10=2\times 5\\
			=&f(16)	&&\qquad\text{ porque }5\text{ es impar}\\
			=&1&&\qquad\text{ porque } 16=2^4.
\end{alignat*}
\begin{enumerate}[(a)]
	\item Mostrar que $f(11)$ también es igual a $1$.
	\item Mostrar que $f(9)$, $f(14)$, y $f(25)$ son todos iguales a $f(11)$ y, por lo tanto, todos iguales a $1$.
	\item Escriba un programa para hallar $f(27)$.
\end{enumerate}
?`Crees que esta función siempre dará el valor de $1$, sin importar con qué $n$ comiences? Busque la ``Conjetura de Collatz'' o el ``Problema del granizo''.
\end{exercise}

\begin{solution}
	
\end{solution}

\begin{exercise}
Podríamos definir una \emph{desajuste} como una $n$--permutación $S$ de $\left\{1,\ldots,n\right\}$ donde cada $S_{j}\neq j$ y luego definir $\bm{D_n}$ como el número de desajustes de $\left\{1,\ldots,n\right\}$. Entonces $D_{n}$ es la única sucesión que satisface la ecuación de recurrencia
\begin{equation}\label{ex:1.3}
D_{n}=\left(n-1\right)\left\{D_{n-1}+D_{n-2}\right\}\text{ para }n=3,4,5,\ldots
\end{equation}
con las condiciones iniciales $D_{1}=0$ y $D_{2}=1$.
\begin{enumerate}[(a)]
	\item Mostrar que $D_{2}=(2)\left(D_{1}\right)+(-1)^2$.
	\item Use la inducción matemática para probar que para todo entero $n\geq2$, \[ D_{n}=(n)(D_{n-1})+{(-1)}^n. \]
\end{enumerate}
\end{exercise}

\begin{solution}

\end{solution}

\begin{exercise}
Use la inducción matemática y la ecuación \eqref{ex:1.3} para probar que \[ \forall n\in\mathds{Z}^{+}\colon\bm{D_n}=n!\sum_{j=0}^n\frac{(-1)^j}{j!}. \]
\end{exercise}

\begin{solution}
	
\end{solution}

\begin{exercise}
Supongamos que (o busque estos dos resultados de cálculo)
\begin{enumerate}[A.]
	\item $\forall x\in\mathds{R}\colon e^x=\sum_{j=0}^\infty\frac{x^j}{j!}$, y entonces $e^{-1}=\sum_{j=0}^\infty\frac{(-1)^j}{j!},$
	\item $\forall n\in\mathds{Z}^{+}\colon e^{-1}=\sum_{j=0}^n\frac{(-1)^j}{j!}+E_n$ donde $|E_n|<\left|\frac{(-1)^{n+1}}{(n+1)!}\right|=\frac{1}{(n+1)!}$.
\end{enumerate}

\begin{enumerate}[(a)]
	\item Use el resultado de la pregunta anterior para mostrar \[ \frac{n!}{e}=D_{n}+n!E_{n}\text{ donde }|n!E_n|<\frac{n!}{(n+1)!}=\frac{1}{n+1}\leq\frac{1}{2}. \]
	\item Explique por qué $D_{n}-\frac{1}{2}\leq\frac{n!}{e}\leq D_{n}+\frac{1}{2}$.
	\item ?`Es $\left\lceil\frac{n!}{e}\right\rfloor=D_{n}$?
\end{enumerate}

\end{exercise}

\begin{solution}
	
\end{solution}

\begin{exercise}
La \emph{función de Ackermann}\index{Ackermann!función} a veces es definida recursivamente en una forma ligeramente diferente
\begin{enumerate}[Regla (1)]
	\item $B\left(0,n\right)=n+1$ para $n=0,1,2,\ldots$,
	\item $B\left(m,0\right)=B\left(m-1,1\right)$ para $m=1,2,3,\ldots$, y
	\item $B\left(m,n\right)=B\left(m-1,B\left(m,n-1\right)\right)$ cuando ambos $m$ y $n$ son positivos.
\end{enumerate}

\begin{enumerate}
	\item Use inducción matemática para probar $\forall n\in\mathds{N}\colon B\left(1,n\right)=n+2$.
	\item Use inducción matemática para probar $\forall n\in\mathds{N}\colon B\left(2,n\right)=3+2n$.
	\item Use inducción matemática para probar $\forall n\in\mathds{N}\colon B\left(3,n\right)=2^{3+n}-3$.
	\item Use inducción matemática para probar $\forall n\in\mathds{N}\colon B\left(4,n\right)=$.
\end{enumerate}
\end{exercise}

\begin{solution}
	
\end{solution}

\begin{exercise}
Supongamos que $A$ es un conjunto de $2n$ objetos. Sea $P_{n}$ el número de diferentes maneras que los objetos en $A$ pueden ser ``emparejados'' (el número de diferentes particiones de $A$ en $2$--subconjuntos). Supongamos que $n\in\mathds{Z}^{+}$. Si $n=2$, entonces $A$ tiene cuatro elementos, $A=\left\{x_1,x_2,x_3,x_4\right\}$. Los tres posibles emparejamientos son:
\begin{enumerate}
	\item $x_{1}$ con $x_{2}$ y $x_{3}$ con $x_{4}$,
	\item $x_{2}$ con $x_{3}$ y $x_{2}$ con $x_{4}$,
	\item $x_{3}$ con $x_{4}$ y $x_{2}$ con $x_{3}$.
\end{enumerate}
Así $P_{2}=3$.
\end{exercise}
\begin{enumerate}[(a)]% TODO: Crear el programa en Python
	\item Mostrar que si $n=3$ y $A=\left\{x_{1},x_{2},x_{3},x_{4},x_{5},x_{6}\right\}$, existen $15$ posibles emparejamientos enumerándolos a todos:
	\begin{enumerate}
		\item $x_{1}$ con $x_{2}$ y $x_{3}$ con $x_{4}$ y $x_{5}$ con $x_{6}$.
		\item \ldots
	\end{enumerate}
	Así $\bm{P_3}=15$.
	\item Mostrar que $P_{n}$ debe satisfacer la RE $P_{n}=(2n-1)P_{n-1}$ para $\forall n\geq2$.
	\item Use esta ecuación de recurrencia y la inducción matemática para probar \[ P_{n}=\frac{(2n)!}{2^n\times n!}=\text{ para }\forall n\geq 1. \]
	\item Mostrar que $y_{n}=\frac{n(n-1)}{2}+c$ para $n>0$ es una solución de la relación de recurrencia \[ y_{n+1}=y_{n}+n. \]
\end{enumerate}

\begin{solution}
	
\end{solution}

\begin{exercise}
Suponga que una sucesión es definida por: \[ f(0)=5\text{ y }f(n+1)=2\times f(n)+1\text{ para } n=0,1,2,\ldots. \]
\begin{enumerate}[(a)]
	\item Encuentre el valor de $f(10)$.
	\item Probar que la sucesión ni es una sucesión aritmética ni es una sucesión geométrica.
\end{enumerate}
\end{exercise}

\begin{solution}

\end{solution}

\begin{exercise}
\begin{enumerate}[(a)]
	\item Encuentre la solución general de la ecuación de recurrencia \[ S_{n}=3S_{n-1}-10\text{ para }n=1,2,\ldots \]\label{ex:1.10a}
	\item Determine la solución particular donde $S_{0}=15$.
	\item Use la fórmula en (\eqref{ex:1.10a}) para evaluar $S_6$ y verifique su respuesta usando la ecuación de recurrencia en sí.
\end{enumerate}
\end{exercise}

\begin{solution}
	
\end{solution}

\begin{exercise}
Suponga que $s_{0}=60$ y $s_{n+1}=(1/5)s_n-8$ para $n=0,1,\ldots$.
\begin{enumerate}[(a)]
	\item Encuentre $s_{1}$, $s_{2}$, y $s_{3}$.
	\item Resuelva la relación de recurrencia para dar una fórmula para $s_{n}$.
	\item ?`Es esa sucesión convergente? Si es así, ?`cuál es el límite?
	\item ?`La serie correspondiente converge? Si es así, ?`cuál es el límite?
\end{enumerate}
\end{exercise}

\begin{solution}
	
\end{solution}

\begin{exercise}
Suponga que $s_{0}=75$ y $s_{n+1}=(1/3)s_{n}-6$ para $n=0,1,\ldots$.
\begin{enumerate}[(a)]
	\item Encuentre $s_{1}$, $s_{2}$, y $s_{3}$.
	\item Resuelva la relación de recurrencia para dar una fórmula para $s_{n}$.
	\item ?`Es esa sucesión convergente? Si es así, ?`cuál es el límite?
	\item ?`La serie correspondiente converge? Si es así, ?`cuál es límite?
\end{enumerate}
\end{exercise}

\begin{solution}
	
\end{solution}

\begin{exercise}
\begin{enumerate}[(a)]
	\item Mostrar que $f_{n}=A\times3^{n}+B\times2^{n}$ satisface la ecuación de recurrencia \[ f_{n}=5f_{n-1}-6f_{n-2}\text{ para }n\geq 2. \]
	\item Encuentre la solución particular (valores para $A$ y $B$) para que \[ f_{0}=4\text{ y }f_{1}=17. \]
\end{enumerate}

\end{exercise}

\begin{solution}
	
\end{solution}

\section{Recurrencias Lineales con coeficientes constantes}

Una relación de recurrencia lineal de orden $r$ con coeficientes constantes es una recurrencia del tipo:
\begin{align}\label{1}
c_{0}x_{n}+c_{1}x_{n-1}+\cdots+c_{r}x_{n-r}=h_{n},\forall n\geq r,
\end{align}
donde $c_{0},c_{1},\ldots,c_{r}$ son constantes reales o complejas, con $c_{0}$ y $c_{r}$ ambos diferentes de cero y $(h_{n})_{n\geq r}$ es una sucesión de números reales o complejos llamado sucesión de términos no homogéneos de la recurrencia. La recurrencia es llamada homogénea si la sucesión de términos no homogéneos es una sucesión nula, no homogénea si $h\neq0 $ para algún $n$. La relación de recurrencia:
\begin{align}\label{2}
c_{0}x_{n}+c_{1}x_{n-1}+\cdots+c_{r}x_{n-r}=0,\forall n\geq r,
\end{align}
es llamada la recurrencia homogénea asociada, o la parte homogénea de la recurrencia \eqref{1}. Como nosotros ya hemos notado, la recurrencia:
\begin{equation*}
c_{0}x_{n}+c_{1}x_{n-1}+\cdots+c_{r}x_{n-r}=h_{n},\forall n\geq r,
\end{equation*}
puede ser escrito equivalentemente como
\begin{equation*}
c_{0}x_{n+r}+c_{1}x_{n+(r-1)}+\cdots+c_{r}x_{n}=h_{n+r},\forall n\geq 0.
\end{equation*}
Se puede utilizar cualquiera de las formas presentadas.

\begin{remark}
	Cada $r$-secuencia de valores asignados a las $r$ incógnitas desconocidas de la relación de recurrencia
	\begin{equation*}
	c_{0}x_{n}+c_{1}x_{n-1}+\cdots+c_{r}x_{n-r}=h_{n},\forall n\geq r,
	\end{equation*}
	determina de forma única una solución. Al resolver una relación de recurrencia lineal, el siguiente principio es fundamental importancia.
\end{remark}

\begin{proposition}[Principio de superposición]\index{Principio de superposición}
	Sean ${(u_{n})}_{n}$, ${(V_{n})}_{n}$ respectivamente las soluciones de las relaciones de recurrencia lineal.
	\begin{align*}
	c_{0}x_{n}+c_{1}x_{n-1}+\cdots+c_{r}x_{n-r}&=h_{n},\quad n\geq r
	\intertext{y}
	c_{0}x_{n}+c_{1}x_{n-1}+\cdots+c_{r}x_{n-r}&=k_{n},\quad n\geq r,
	\end{align*}
	con partes homogéneas iguales y secuencias de términos no homogéneos $(h_{n})_{n}$ y $(k_{n})_{n}$. Para cualquier par de constantes $A$ y $B$, la sucesión $(Av_{n}+Bv_{n})_{n}$ es una solución de la relación de recurrencia. \[ c_{0}x_{n}+c_{1}x_{n-1}+\cdots+c_{r}x_{n-r}=Ah_{n}+Bk_{n}. \] La solución general de la relación de recurrencia
	\begin{equation}\label{eq:super}
	c_{0}x_{n}+c_{1}x_{n-1}+\cdots+c_{r}x_{n-r}=h_{n},\quad n\geq r.
	\end{equation}
\end{proposition}

\begin{proof}\leavevmode
	\begin{enumerate}
		\item Uno tiene fácilmente
		\begin{equation*}
		\begin{split}
		&c_{0}(Au_{n}+Bv_{n})+c_{1}(Au_{n-1}+Bv_{n-1})+\cdots+c_{r}(Au_{n-r}+Bv_{n-r})=\\
		&\phantom{c_{0}(Au_n+}=A(c_{0}u_{n}+c_{1}u_{n-1}+\cdots+c_{r}u_{n-r})+B(c_{0}v_{n}+c_{1}v_{n-i}+\cdots+c_{r}v_{n-r})\\
		&\phantom{c_{0}(Au_n+}=Ah_{n}+Bk_{n}.
		\end{split}
		\end{equation*}
		\item Sea $(u_{n})_{n}$ una solución particular de \eqref{eq:super}. Por el punto previo nosotros conocemos que $(v_{n})_{n}=(u_{n})_{n}+(v_{n}-u_{n})_{n}$ es una solución de \eqref{eq:super} si y solo si $v_{n}-u_{n}$ es una solución de la recurrencia homogénea asociada. Por lo tanto cada solución de \eqref{eq:super} es obtenida añadiendo una solución de la  recurrencia homogénea asociada para $(u_{n})_{n}$.
	\end{enumerate}
\end{proof}

\section{Relación de recurrencia lineal con homogénea con coeficientes constantes}

La sucesión nula es una solución de cualquier relación de recurrencia lineal. La estructura de la solución general de una relación de recurrencia lineal homogénea corresponde a la estructura de la solución general de un sistema de ecuaciones lineales homogéneas.
\begin{proposition}[Teorema principal]
	Considere la relación de recurrencia lineal homogénea de orden $r$:
	\begin{equation}\label{eq:homo}
	c_{0}x_{n}+c_{1}x_{n-1}+\cdots+c_{r}x_{n-r}=0,\quad n\geq r\quad\left(c_{0}c_{r}\neq0\right)
	\end{equation}
	\begin{enumerate}
		\item Cualquier combinación lineal de soluciones de \eqref{eq:homo} es de nuevo una solución de \eqref{eq:homo}.
		\item Existe $r$ soluciones de \eqref{eq:homo} tal que cualquier otra solución de \eqref{eq:homo} puede ser expresado únicamente como su combinación lineal.
	\end{enumerate}
\end{proposition}

\begin{proof}\leavevmode
	\begin{enumerate}
		\item Esto sigue inmediatamente por el ``Principio de Superposición''.
		\item Para todo $i\in\left\{0,\ldots,r-1 \right\}$ sea $\left(u^{i}_{n}\right)_{n}$ la solución de \eqref{eq:homo} con $r$--sucesión de valores iniciales iguales a $0$ para índices $j\neq i$, iguales a $1$ en índices $i$, es decir: \[ u^{i}_{j}=0\text{ si }j\neq i,\quad u^{i}_{i}=1\quad j\in\left\{0,\ldots,r-1 \right\}. \]
		Consideramos ahora alguna solución $(a_{n})_{n}$ de \eqref{eq:homo}; la combinación lineal \[ a_{0}{\left(u^{0}_{n}\right)}_{n}+a_{1}{\left(u^{1}_{n}\right)}_{n}+\cdots+a_{r-1}(u^{r-1}_{n})_{n}, \]	es una solución de \eqref{eq:homo} con secuencia de datos iniciales $\left(a_{0},\ldots,a_{r-1}\right)$. Ya que la sucesión de valores iniciales determinan la solución de una relación de recurrencia, uno tiene \[ {\left(a_{n}\right)}_{n}=a_{0}\left(u^{0}_{n}\right)_{n}+a_{1}\left(u^{1}_{n}\right)_{n}+\cdots+a_{r-1}\left(u^{r-1}_{n}\right)_{n}. \]
	\end{enumerate}
\end{proof}

\begin{definition}[Polinomio característico]\index{Relación de recurrencia!polinomio característico}
	Definimos el \emph{polinomio característico} de una relación de recurrencia con coeficientes constantes de orden $r$ de la siguiente manera: \[ c_{0}x_{n}+c_{1}x_{n-1}+\cdots+c_{r}x_{n-r}=h_{n},\quad n\geq r\left(c_{0}c_{r}\neq0\right), \] para el polinomio de grado $r$: \[ P(X)\coloneqq c_{0}X^{r}+c_{1}X^{r-1}+\cdots+c_{r}. \] Cada polinomio de grado $r$ tiene exactamente $r$ raíces complejas contando con su multiplicidad. Vemos ahora que la sucesión de las potencias naturales de una determinada raíz del polinomio característico de una relación de recurrencia lineal es una solución de la correspondiente relación homogénea.
\end{definition}

\begin{proposition}[Raíz del polinomio característico]\index{Relación de recurrencia!polinomio característico!raíz}
	Sea $\lambda\in\mathds{C}$. La sucesión $\left(\lambda^{n}\right)_{n}$ de las potencias de $\lambda$ es una solución de la relación de recurrencia lineal homogénea
	\begin{align}\label{5}
	c_{0}x_{n}+c_{1}x_{n-1}+\cdots+c_{r}x_{n-r}=0,\quad n\leq r \quad (c_{0}c_{r}\neq 0),
	\end{align}
	sii $\lambda$ es una raíz de este polinomio característico.
\end{proposition}

\begin{proof}
	Dado que $c_{r}\neq0$, las raíces del polinomio característico deben ser necesariamente no nulas. Sustituyendo los valores de la sucesión ${\left(\lambda^{n}\right)}_{n}$ en la recurrencia, uno tiene \[ c_{0}x_{n}+c_{1}x_{n-1}+\cdots+c_{r}x_{n-r}=0, \] y dividiendo por $\lambda^{n-r}\neq0$ \[ c_{0}\lambda^{r}+c_{1}\lambda^{r-1}+\cdots+c_{r}=0. \]	Por lo tanto, la sucesión ${\left(\lambda^{n}\right)}_{n}$ es una solución de \eqref{5} sii $\lambda$ es una raíz del polinomio $c_{0}X^{r}+c_{1}X^{r-1}+\cdots+c_{r}$.
\end{proof}

En general, no es fácil encontrar las raíces de un polinomio de grado mayor que dos, aunque uno puede siempre usar un adecuado CAS. El siguiente criterio simple, sin embargo, muestra cómo encontrar las raíces racionales de un polinomio con coeficientes enteros.

\begin{proposition}[Las raíces racionales de un polinomio con coeficientes enteros]
	Sea $P(X)=c_{0}X^{r}+c_{1}X^{r-1}+\cdots+c_{r}$ un polinomio con coeficientes enteros $c_{0}\ldots c_{r}\in\mathds{Z}$, con $c_{0}\neq 0$. Si la fracción $\tfrac{a}{b}$ con $a,b\in\mathds{Z}$ con $\operatorname{mcd}=1$ es una raíz de $P(X)$, luego $a\divides c_{r}$ y $b\divides c_{0}$. En particular, si $c_{0}=\pm1$ las raíces racionales del polinomio $P(X)$ son enteros que dividen a $c_{r}$.
\end{proposition}

\begin{proof}
	Dado $c_{0}\left(\frac{a}{b}\right)^{r}+c_{1}{\left(\frac{a}{b}\right)}^{r-1}+\cdots+c_{r-1}\left(\frac{a}{b}\right)+c_{r}=0$, multiplicado por $b^{r}$ obtenemos: \[ 	c_{0}a^{r}+c_{1}a^{r-1}b+\cdots+c_{r-1}ab^{r-1}+c_{r}b^{r}=0. \] Como $a\divides c_{0}a^{r}+c_{1}a^{r-1}b+\cdots+c_{r-1}ab^{r-1}$, luego tiene que dividir también $c_{r}b^{r}$, y por lo tanto, al no tener $a$ y $b$ factores comunes, $a\divides c_{r}$. Análogamente $b\divides c_{0}a^{r}$ y por lo tanto divide a $c_{0}$.
\end{proof}

\begin{example}{Polinomio característico}
	La recurrencia homogénea de segundo orden: \[ x_{n}=2x_{n-1}-2x_{n-2},\quad n\geq2, \] tiene polinomio característico $X^{2}-2X+2$ cuyas raíces son $\lambda_{1}=1-i$ y $\lambda_{2}=1+i$. Las sucesiones ${\left((1-i)^{n}\right)}_{n}$ y ${\left((1+i)^{n}\right)}_{n} $ son las soluciones bases de la recurrencia. La solución general compleja de la recurrencia es: \[ x_{n}=A_{1}{\left(1-i\right)}^{n}+A_{2}{\left(1+i\right)}^{n},\quad n\geq 0, \] con la variante de $A_{1}$ y $A_{2}$ entre los números complejos. Veamos la solución real general. Uno tiene: \[ \lambda_{1}=1-i=\sqrt{2}\left(\frac{\sqrt{2}}{2}-\frac{\sqrt{2}}{2}i\right)=\sqrt{2}\left(\cos\left(\frac{\pi}{4}\right)-i\sen\left(\frac{\pi}{4}\right)\right) \] y \[ \lambda_{2}=1+i=\overline{\lambda_{1}}=\sqrt{2}\left(\cos\left(\frac{\pi}{4}\right)-i\sen\left(\frac{\pi}{4}\right)\right). \] Luego, las sucesiones ${\left(2^{n/2}\cos\left( \frac{n\pi}{4}\right)\right)}_{n}$ y ${\left(2^{n/2}\sen\left(\frac{n\pi}{4}\right)\right)}_{n}$ son las soluciones base reales de la recurrencia. Por lo tanto, la solución general real de la recurrencia es: \[ x_{n}=A_{1}2^{n/2}\cos\left(\frac{n\pi}{4}\right)+A_{2}2^{n/2}\sen\left(\frac{n\pi}{4}\right),\quad n\geq 0, \] con la variación de $A_{1}$ y $A_{2}$ entre los números reales.
\end{example}

\begin{claim}
Afirmo que el Lema de Zorn es cierto.
\end{claim}

\begin{proof}
$\smartqed$

$\qed$
\end{proof}

\begin{case}
	
\end{case}

\begin{conjecture}
	
\end{conjecture}

\begin{corollary}
	
\end{corollary}

\begin{definition}
	
\end{definition}

\begin{example}{Quispe}
	
\end{example}

\begin{lemma}
	
\end{lemma}

\begin{note}
	
\end{note}

\begin{problem}
	
\end{problem}

\begin{property}
	
\end{property}

\begin{proposition}
	
\end{proposition}

\begin{question}{Bryan}
	
\end{question}

\begin{remark}

\end{remark}

\begin{theorem}

\end{theorem}

\begin{trailer}{Enfatizar párrafos}
\end{trailer}

\begin{question}{?`Qué hora es?}
\end{question}

\begin{important}{Importante}
	A
\end{important}

\begin{warning}{Atención}
	
\end{warning}

\begin{tips}{Consejos}
	
\end{tips}

\begin{overview}{Enfatizar párrafos completos}
	
\end{overview}

\begin{backgroundinformation}{Información de fondo}
	
\end{backgroundinformation}

\begin{legaltext}{Texto legal}
	
\end{legaltext}
%\motto{Use the template \emph{chapter.tex} to style the various elements of your chapter content.}
use \chaptermark{Una prueba}
\abstract*{Otra prueba}
% Always give a unique label
% and use \ref{<label>} for cross-references
% and \cite{<label>} for bibliographic references
% use \sectionmark{}
%% to alter or adjust the section heading in the running head
\begin{figure}[!ht]
	\sidecaption[t]
	% Use the relevant command for your figure-insertion program
	\includegraphics[width=0.343\textwidth]{./img/example}
	% If not, use
	%\picplace{5cm}{2cm} % Give the correct figure height and width in cm
	\caption{Da}
	\label{fig:1}       % Give a unique label
\end{figure}

\eject

%\begin{eqnarray}
%\left|\nabla U_{\alpha}^{\mu}(y)\right| &\le&\frac1{d-\alpha}\int
%\left|\nabla\frac1{|\xi-y|^{d-\alpha}}\right|\,d\mu(\xi) =
%\int \frac1{|\xi-y|^{d-\alpha+1}} \,d\mu(\xi)\qquad  \\
%&=&(d-\alpha+1) \int\limits_{d(y)}^\infty
%\frac{\mu(B(y,r))}{r^{d-\alpha+2}}\,dr \le (d-\alpha+1)
%\int\limits_{d(y)}^\infty \frac{r^{d-\alpha}}{r^{d-\alpha+2}}\,dr
%\label{eq:01}
%\end{eqnarray}

\enlargethispage{24pt}

\begin{quotation}
Please do not use quotation marks when quoting texts! Simply use the \verb|quotation| environment -- it will automatically be rendered in the preferred layout.
\end{quotation}

Fig.~\ref{fig:1}\footnote{X}
\paragraph{Paragraph Heading}
For typesetting numbered lists we recommend to use the \verb|enumerate| environment -- it will automatically render Springer's preferred layout.
\begin{figure}[h]
	\centering
	\includegraphics[width=0.6\textwidth]{./img/example}
	\caption{Costo de producción proporcional a la raíz cuadrada de la tasa de producción.}
\end{figure}
\begin{figure}[t]
\sidecaption[t]
% Use the relevant command for your figure-insertion program
% to insert the figure file.
% For example, with the option graphics use
\includegraphics[scale=.65]{./img/example}
% If not, use
%\picplace{5cm}{2cm} % Give the correct figure height and width in cm
\caption{Please write your figure caption here}
\label{fig:2}       % Give a unique label
\end{figure}

\runinhead{Run-in Heading Boldface Version} 

\subruninhead{Run-in Heading Boldface and Italic Version}

\subsubruninhead{Run-in Heading Displayed Version}

\begin{table}[!t]
\caption{Please write your table caption here}
\label{tab:1}       % Give a unique label
%
% For LaTeX tables use
%
\begin{tabular}{p{2cm}p{2.4cm}p{2cm}p{4.9cm}}
\hline\noalign{\smallskip}
Classes & Subclass & Length & Action Mechanism  \\
\noalign{\smallskip}\svhline\noalign{\smallskip}
Translation & mRNA  & 24--26 & Histone and DNA Modification\\
\noalign{\smallskip}\hline\noalign{\smallskip}
\end{tabular}
$^a$ Table foot note (with superscript)
\end{table}
% Always give a unique label
% and use \ref{<label>} for cross-references
% and \cite{<label>} for bibliographic references
% use \sectionmark{}
% to alter or adjust the section heading in the running head
If you want to list definitions or the like we recommend to use the Springer-enhanced \verb|description| environment -- it will automatically render Springer's preferred layout.

\begin{svgraybox}
If you want to emphasize complete paragraphs of texts we recommend to use the newly defined Springer class option \verb|graybox| and the newly defined environment \verb|svgraybox|. This will produce a 15 percent screened box 'behind' your text.

If you want to emphasize complete paragraphs of texts we recommend to use the newly defined Springer class option and environment \verb|svgraybox|. This will produce a 15 percent screened box 'behind' your text.
\end{svgraybox}

\paragraph{Paragraph Heading}

\begin{trailer}{Cabeza de remolque}
If you want to emphasize complete paragraphs of texts in an \verb|Trailer Head| we recommend to
use  \begin{verbatim}\begin{trailer}{Trailer Head}
...
\end{trailer}\end{verbatim}
\end{trailer}
%
\begin{question}{Preguntas}
If you want to emphasize complete paragraphs of texts in an \verb|Questions| we recommend to
use  \begin{verbatim}\begin{question}{Questions}
...
\end{question}\end{verbatim}
\end{question}

\begin{important}{Importante}
If you want to emphasize complete paragraphs of texts in an \verb|Important| we recommend to
use  \begin{verbatim}\begin{important}{Important}
...
\end{important}\end{verbatim}
\end{important}
%
\clearpage
\begin{warning}{Atención}
If you want to emphasize complete paragraphs of texts in an \verb|Attention| we recommend to
use  \begin{verbatim}\begin{warning}{Attention}
...
\end{warning}\end{verbatim}
\end{warning}

\begin{programcode}{Código de programa}
If you want to emphasize complete paragraphs of texts in an \verb|Program Code| we recommend to
use

\verb|\begin{programcode}{Program Code}|

\verb|\begin{verbatim}...\end{verbatim}|

\verb|\end{programcode}|

\end{programcode}
%
\begin{tips}{Consejos}
If you want to emphasize complete paragraphs of texts in an \verb|Tips| we recommend to
use  \begin{verbatim}\begin{tips}{Tips}
...
\end{tips}\end{verbatim}
\end{tips}
%
%
\begin{overview}{Visión general}
If you want to emphasize complete paragraphs of texts in an \verb|Overview| we recommend to
use  \begin{verbatim}\begin{overview}{Overview}
...
\end{overview}\end{verbatim}
\end{overview}
\clearpage
\begin{backgroundinformation}{Background Information}
If you want to emphasize complete paragraphs of texts in an \verb|Background|
\verb|Information| we recommend to
use

\verb|\begin{backgroundinformation}{Background Information}|

\verb|...|

\verb|\end{backgroundinformation}|
\end{backgroundinformation}
\begin{legaltext}{Legal Text}
If you want to emphasize complete paragraphs of texts in an \verb|Legal Text| we recommend to
use  \begin{verbatim}\begin{legaltext}{Legal Text}
...
\end{legaltext}\end{verbatim}
\end{legaltext}

\section*{Apéndice}
\addcontentsline{toc}{section}{Apéndice}

\section*{Problemas}
\addcontentsline{toc}{section}{Problems}
% Use the following environment.
% Don't forget to label each problem;
% the label is needed for the solutions' environment
\begin{prob}
\label{prob1}
A given problem or Excercise is described here. The
problem is described here. The problem is described here.
\end{prob}
\textit{Resolver una ecuación de recurrencia} significa encontrar una sucesión que satisfaga las ecuaciones de recurrencias. Encontrar una ``solución general'' significa hallar una fórmula que describa todas las soluciones posibles (todas las sucesiones posibles que satisfacen la ecuación). Veamos el siguiente ejemplo:

\begin{example}{}
	Considere que $T_{n}$ satisface la siguiente ecuación para todo $n\in\mathds{N}$, $n>1$:
	\begin{equation*}
	T_{n}=2T_{n-1}+1.
	\end{equation*}
	La ecuación de recurrencia $T_{n}$ indica cómo continúa la sucesión pero no nos dice como empieza tal. % TODO: Crear una tabla de valores.
	\begin{itemize}
		\item Si $T_{1}=1$, se tiene $T=\left(1,7,3,15,31,\ldots\right)$.
		\item Si $T_{2}=1$, se tiene $T=\left(2,5,11,23,47,\ldots\right)$.
		\item Si $T_{4}=1$, se tiene $T=\left(4,9,19,39,79,\ldots\right)$.
		\item Si $T_{-1}=1$, se tiene $T=\left(-1,-1,-1,1-1,-1,\ldots\right)$.
	\end{itemize}
	?`Existe alguna fórmula para cada una de estas sucesiones? ?`Existe una fórmula en términos de $n$ y $T_{1}$ que describa todos los términos de la sucesión? ?`Existe una posible solución para $T_{n}$? Para poder responder este tipo de problemas, veamos un poco más de ecuaciones con recurrencia.
\end{example}

\section{Ejemplos definidos por ecuaciones de recurrencia}
% TODO: Checkear transtornada por permutada.
\begin{example}{}
	Imagina una fiesta donde las parejas llegan juntas, pero al final de la noche, cada persona se va con una nueva pareja. Para cada $n\in P$, digamos que $D_{n}$ es el número de diferentes formas en que las parejas pueden ser ``trastornadas'', es decir, reorganizadas en parejas, por lo que ni uno está emparejado con la persona con la que llegaron.

	Para los valores:

	$D_{1} = 0$  // una pareja no puede ser transtornada.

	$D_{2} = 1$  // $\exists$ una y solo una manera de transtornar una pareja.

	$D_{3} = 2$  // si las parejas llegan como $Aa$, $Bb$, $Cc$, entonces $A$ estaría emparejado con $b$ o $c$. Si $A$ esta emparejado con $b$, $C$ debe estar emparejado con $a$(y no $c$) y $B$ con $c$. Si $A$ esta emparejado con $c$, $B$ no debe estar emparejado con $a$(y no $b$) y $C$ con $b$.

	?`Qué tan grandes son $D_{4}$, $D_{5}$ y $D_{10}$? ?`Cómo podemos calcularlos? ?`Existe alguna expresión cerrada para obtener todos los términos de la? % TODO: Sucesión.

	Vamos a desarrollar una estrategia para contar los desajustes cuando $n\leq4$. Supongamos que hay $n$ mujeres $A_{1},A_{2},A_{3},\ldots,A_{n}$, y cada $A_{j}$ llega con el hombre $a_{j}$.

	La mujer $A_{1}$ puede ser ``re-emparejada'' con cualquiera de los $n-1$ hombres restantes $a_{2}$ o $a_{3}$ o \ldots o $a_{n}$. Digamos que está emparejada con $a_{k}$, donde $2\leq k\leq n$ y ahora consideremos $a_{k}^{\prime}$ pareja original de la mujer $A_{k}$: ella podría tomar $a_{1}$ o ella podría rechazar $a_{1}$ y tomar a alguien más.
	% TODO: Checkearlo.
	
	Si $A_{1}$ es pareja con $a_{k}$ y $A_{k}$ es pareja con $a_{1}$, entonces $n-2$ parejas dejaron para transtornar, y eso puede hacerse exactamente de $D_{n-2}$ maneras diferente.
	% TODO: Cambiar hacerse.

	Ahora para cada uno de los $n-1$ hombres que $A_{1}$ podría elegir, hay $\{D_{n-2}+ D_{n-1}\}$ diferentes formas de completar el trastorno. Por lo tanto, cuando $n\geq 4$ tenemos:
\begin{equation*}\label{eq:1_1}
D_{n}=\left(n-1\right)\left\{D_{n-2}+D_{n-1}\right\}%\tag{1_1}
\end{equation*}
Usando la ecuación \eqref{eq:1_1} las evaluaciones para 1 y 2 verifican la igualdad, ahora evaluemos $D_{n}$ para cualquier valor de $n$, con $n\in\mathds{N}$

$D_{3}=\left(3-1\right)\left\{D_{2}+D_{1}\right\}=2\left(1+8\right)=2$

$D_{4}=\left(4 - 1\right)\left\{D_{3}+D_{2}\right\}=3\left(2+1\right)=9$

$D_{5}=\left(5-1\right)\left\{D_{4}+D_{3}\right\}=4\left(9+2\right)=44$

$D_{6}=(6-1)\left\{D_{5}+D_{4}\right\}=5\left(44 + 9\right)=265$

$D_{7}=\left(7-1\right)\left\{D_{6}+D_{5}\right\}=6\left(265+44\right)=1854$

$D_{8}=\left(8-1\right)\left\{D_{7}+D_{6}\right\}=7\left(1854+265\right)=14833$

$D_{9}=\left(9-1\right)\left\{D_{8}+D_{7}\right\}=8\left(14833+1854\right)=133496$

$D_{10}=\left(10-1\right)\left\{D_{9}+D_{8}\right\}=9\left(133496+14833\right)=1334961$

La sucesión en $P$ definido por $S_{n}=A\times n!$ donde $A$ es un número real satisface la ecuación de recurrencia \eqref{eq:1_1}. Si $n\geq3$ se tiene:
\begin{align*}
	\left(n-1\right)\left\{S_{n-2}+S_{n-1}\right\}
	&=(n-1)\{A(n-2)!+A(n-1)!\} \\
	&=(n-1)A(n-2)!\{1+(n-1)\} \\
	&=A(n-1)(n-2)!\{n\}\\
	&=A\times n!\\
	&=S_{n}.
\end{align*}
?`Es válida la fórmula para $n=1$ o $n=2$? ?`Existe algún número real tal que $D_{n}=A(n!)$ cuando $n=1$ o $n=2$? No, porque si $0=D_{1}=A(1!)$, entonces $A$ debe ser igual a $0$, y si $1=D_{2}=A(2!)$, se tiene que $A$ debería tomar el valor de $\frac{1}{2}$. Sin embargo, podemos usar esta fórmula para probar que $D_{n}$ es acotado.
\end{example}

\begin{theorem}{}
Para todo $n\geq 2$, $\left(\frac{1}{3}\right)n!\leq D_{n}\leq\left(\frac{1}{2}\right)n!$.
\end{theorem}

\begin{example}{Números de Ackermann}\index{Ackermann!número}
	En la década de 1920's, el lógico y matemático alemán, Wilhelm Ackermann (1896–1962), inventó una función muy curiosa, $A\colon P\times P\rightarrow P$ que define recursivamente usando ``tres reglas'':
	\begin{enumerate}[Regla 1]
		\item $A(1,n)=2$ para $n=1,2,\ldots$.
		\item $A(m,1)=2m$ para $m=2,3,\ldots$.
		\item Cuando $m>1$ y $n>1$ se tiene $A(m,n)=A(A(m-1,n),n-1)$.
	\end{enumerate}
\end{example}

\begin{align*}
	\intertext{Entonces}
	A(2,2)
	&=A(A(2-1,2),2-1) //regla~3\\
	&= A(A(1,2),1)\\
	&= A(2,1)//regla~ 1\\
	&= 2(2)  //regla~2\\
	&= 4.
	\intertext{además}
	A(2,3) &= A(A(2-1,3),3-1) //regla~3\\
	&= A(A(1,3),2)\\
	&= A(2,2)  //regla 1\\
	&= 4.
	\intertext{De hecho}
	si~A(2,k) &= 4, // para algun k\geq2
	\intertext{entonces}
	A(2,k+1) &= A(A(2-1,k+1), (k+1)-1) // regla~3 \\
				   &=A(A(1,k+1),k)\\
				   &=A(2,k) //regla~1 \\
				   &=4.// nuestro~supuesto
	\intertext{Así, tenemos por inducción matemática:}
	A(2,n) &= 4,  \forall n\geq1.
\end{align*}
Hasta ahora la tabla de los números de Ackermann se ve así:
\begin{equation*}
\begin{tabular}{ c| c| c| c| c| c| c| c| c| r }
A & n=1 & n=2 & 3 & 4 & 5 & 6 & 7 & 8 & 9... \\
\hline
m=1 & 2 & 2 & 2 & 2 & 2 & 2 & 2 & 2 & 2 \\
\hline
m=2 & 4 & 4 & 4 & 4 & 4 & 4 & 4 & 4 & 4 \\
\hline
3   & 6 &  &  &  &  &  &  &  &  \\
\hline
4   & 8 &  &  &  &  &  &  &  &  \\
\hline
5  & 10 &  &  &  &  &  &  &  &  \\
\end{tabular}
\end{equation*}
Observamos que la segunda fila es de puro 4s. ?`Pero cómo es la segunda columna?
\begin{align*}
	\intertext{Si}
	A(k,2)
	&= 2^{k} // para algunos k \geq 2 \\
	\intertext{se tiene}
	&= A(A([k+1],2-1) // regla 3\\
	&= A(A(k,2),1) \\
	&= A(2^{k},1) // nuestro supuesto \\
	&= 2(2^{k}) //~regla~2\\
	&= 2^{k+1}.
	\intertext{además}
	A(2,3) &= A(A(2-1,3),3-1) //regla~3\\
	&= A(A(1,3),2)\\
	&= A(2,2)  //regla 1\\
	&= 4.
	\intertext{Asi se tiene:}
	A(m,2)=2^{m} \forall m\geq1. \\
	\intertext{Ahora, ?`como son los otros valores?}
	A(3,3)
	&= A(A(3-1),3),3-1) // Regla~3\\
	&= A(A(2,3), 2)  \\
	&=A(4,2) // Segunda~fila \\
	&=2^{4} // segunda~columna\\
	&=16.\\
	A(4,3) &= A(A(4-1),3),3-1 // Regla~3\\
	&= A(A(3,3), 2) \\
	&=A(16,2) // Encima\\
	&=2^{16} // Segunda columna\\
	&=65536.\\
	A(3,4) &= A(A(3-1),3),4-1 // Regla~3\\
	&= A(A(2,4), 3) \\
	&=A(4,3) // Segunda fila\\
	&=65536.\\
	\intertext{?`Cual es el valor de A(4,4)?
	?`Podría ejecutar un programa recursivo simple para evaluar A (4,4)?}
	A(5,3) &= A(A(5-1),3),3-1 // Regla~3\\
	&= A(A(4,3), 2) \\
	&=A(65536,2) \\
	&=2^{65536}.// Segunda~columna\\
	&=n(n grande aprox 20000 digitos en base 10.)\\
	\intertext{hasta ahora tenemos:}
\end{align*}
\begin{equation*}
\begin{tabular}{c|c|c|c|c|c|c|c|c|r}
A & n=1 & n=2 & 3 & 4 & 5 & 6 & 7 & 8 & 9... \\
\hline
m=1 & 2 & 2 & 2 & 2 & 2 & 2 & 2 & 2 & 2 \\
\hline
m=2 & 4 & 4 & 4 & 4 & 4 & 4 & 4 & 4 & 4 \\
\hline
3   & 6 & 8 & 16 & 65536 & ? &  &  &  &  \\
\hline
4   & 8 & 16 & 65536 & ? &  &  &  &  &  \\
\hline
5  & 10 & 32 & $2^{65536}$ &  &  &  &  &  &  \\
\end{tabular}
\end{equation*}
?`Cómo continúa la tercera columna? Sea $2\uparrow$ denota el valor de ``Torre'' de k 2's, definida recursivamente por \[ 2\uparrow 1=2 \text{ y para } k\geq1,2\uparrow[k+1]=2^{2\uparrow k}. \]
Pero este es un número tan grande que nunca podría escribirse en dígitos decimales, incluso utilizando todo el papel del mundo, Su valor nunca podría ser calculado. Ahora nos preguntamos ?`Los números Ackermann son ``computables''? Por otro lado, supongamos que las sucesiones que encontramos, incluso aquellas definidas por ecuaciones de recurrencia, serán fáciles para entender y tratar.
\section{Resolución de ecuaciones de recurrencia lineal de primer orden}

\subsection{Las torres de Hanoi}

La ecuación de recurrencia para el número de movimientos en las Torres de Hanoi es una ecuación de recurrencia lineal de primer orden:
\begin{equation*}
	T_{n}=2T_{n-1}+1.
\end{equation*}
Sea $a=2$ y $c=1$, entonces $\tfrac{c}{1-a}=\tfrac{1}{1-2}=-1$, y cualquier secuencia $T$ que satisfaga este $RE$ está dado por la fórmula
\begin{align*}
	\bm{T_{n}}&=2^{n}\left[I-(-1)\right]+(-1)\\
	\bm{T_{n}}&=2^{n}\left[I+1\right]-1
\end{align*}
Asumiendo que $T$ tiene el dominio $\mathds{N}$ y que denota $T_0$ por $I$, vimos al principio de este capítulo varias soluciones particulares:

Si $I=0$, entonces $\bm{T}=\left(0,1,3,7,15,31,\ldots\right)$ \ $\bm{T_{n}}=2^{n}[0+1]-1=2^n-1$.

Si $I=2$, entonces $\bm{T}=\left(4,9,19,39,79,159,\ldots\right)$ \ $\bm{T_{n}}=2^{n}[2+1]-1=3\times2^n -1$.

Si $I=4$, entonces $\bm{T}=\left(2,5,11,23,47,95,\ldots\right)$ \ $\bm{T_{n}}=2^{n}[4+1]-1=5\times2^n -1$.

Si $I=-1$, entonces $\bm{T}=\left(-1,-1,-1,-1,-1,\ldots)$ \ $\bm{T_n}=2^{n}\left[-1+1\right]-1=-1$.

\subsection{Los tres piratas naufragados}

Un barco pirata es naufragado en una tormenta en la noche. Tres de los piratas sobreviven y se encuentran en una playa la mañana después de la tormenta. Aceptan cooperar para asegurar su supervivencia. Ellos divisan a un mono en la selva cerca de la playa y pasan todo ese primer día recogiendo una gran pila de cocos y luego se van a dormir exhaustos. Pero ellos son piratas. El primero duerme bien, preocupado por su parte de los cocos; despierta, divide la pila en 3 montones iguales, pero encuentra uno sobrante que arroja en el arbusto para el mono, entierra su tercero en la arena, amontona los otros dos montones, y se va a dormir profundamente. El segundo pirata duerme bien, preocupado por su parte de los cocos; se despierta, divide la pila en 3 montones iguales, pero encuentra uno sobrante que arroja en el arbusto para el mono, entierra su tercero en la arena, amontona los otros dos montones, y se va a dormir profundamente.

El tercero también duerme bien, preocupado por su parte de los cocos; despierta, divide la pila en 3 montones iguales, pero encuentra uno sobrante que arroja en el arbusto para el mono, entierra su tercero en la arena, amontona los otros dos montones juntos, y se va a dormir profundamente.

A la mañana siguiente, todos se despiertan y ven una pila algo más pequeña de cocos que se dividen en 3 montones iguales, pero encontrar uno sobrante que tiran en el arbusto para el mono. ¿Cuántos cocos recolectaron el primer día?

Sea $S_{j}$ el tamaño de la pila después del pirata $j^{4h}$ y sea $S_{0}$ el número que recogieron en el primer día. Entonces
\begin{align*}
	S_{0}&=3x+1\text{ para algún número entero }x\text{ y }S_1=2x,\\
	S_{1}&= 3y+1\text{ para algún número entero }y\text{ y }S_2=2y,\\
	S_{2}&=3z+1\text{ para algún número entero}z\text{ y }S_3=2z,\\
	\intertext{y}
	S_{3}&=3w+1\text{ para algún número entero }w.
\end{align*}
¿Hay una ecuación de recurrencia aquí?
\begin{align*}
	S_{1}&=2x\text{ donde }x=(S_{0}-1)/3\text{, entonces }S_{1}=(2/3)S_0-(2/3)\\
	S_{2}&=2y\text{ donde }y=(S_{1}-1)/3\text{, entonces }S_{2}=(2/3)S_1-(2/3)\\
	S_{3}&=2z\text{ donde }z=(S_{2}-1)/3\text{, entonces }S_{3}=(2/3)S_2-(2/3).
\end{align*}
La ecuación de recurrencia satisfecha por los primeros $S_{j}^{\prime}s$ es
\begin{equation}
	S_{j+1}=(2/3)S_{j}-(2/3).
\end{equation}
Si ahora tenemos $S_{4}=(2/3)S_{3}-(2/3)$, entonces $S_{4}=2[S_{3}-1]/3=2w$ para algún número entero $w$. Queremos saber qué valor (o valores) de $S_0$ producirá un número entero par para $S_4$ cuando aplicamos el RE (1). En (1), $a=2/3$ y $c=-2/3$, entonces $c/(1-a) = -2$, y así la solución general de (1) es
\begin{equation*}
	S_{n}={\left(\frac{2}{3}\right)}^{n}\left[S_{0}+2\right]-2.
\end{equation*}
Por lo tanto, $S_{4}=(2/3)^{4}[S_0+2]-2=(16/81)\left[S_0 + 2\right]-2$.

$S_{4}$ será un número entero

$\Leftrightarrow S_{4}+2$ es (un aún) el número entero

$\Leftrightarrow 81\divides\left[S_0 + 2\right]$

$\Leftrightarrow\left[S_{0}+2\right]= 81k$ para algún número entero $k$

$\Leftrightarrow S_{0}=81k-2$ para algún número entero $k$.

$S_{0}$ debe ser un número entero positivo, pero hay un número infinito de respuestas posibles: \[ 79\vee160\vee241\vee322\vee\cdots \]

Necesitamos más información para determinar $S_0$. Si nos hubieran dicho que el primer día los piratas recolectaron entre $200$ y $300$ cocos, ahora podríamos decir ``el número que recogieron el primer día fue exactamente $241$''.

\subsection{Interés Compuesto}

Supongamos que se le ofrecen dos planes de ahorro para la jubilación. En el plan $A$, empiezas con $\$1,000$, y cada año (en el aniversario del plan), te pagan un $11\%$ de interés simple, y agregas $\$1,000$. En el plan $B$, empiezas con $\$100$, y cada mes, te pagan una-duodécima parte del $10\%$ de interés simple (anual), y agregas $\$100$. ¿Qué plan será más grande después de $40$ años?. ¿Podemos aplicar una ecuación de recurrencia? Considere el plan A y deje que $S_{n}$ denote el número de dólares en el plan después de (exactamente) $n$ años de operación. Entonces $S_{0}=\$1,000$ y
\begin{align*}
	S_{n+1}&= S_{n}+\text{ interés sobre }S_n+\$1000\\
	S_{n+1}&=S_{n}+11\%\text{ de }S_n+\$1000\\
	S_{n+1}&=S_{n}(1+0.11)+\$1000.
\end{align*}
En esta RE, $a=1.11$, $c=1000$, entonces $\ffrac{c}{1-a}=\ffrac{1000}{-0.11}$ y
\begin{align*}
	S_{n}&={\left(1.11\right)}^{n}\left[1000-\frac{1000}{-0.11}\right]+\frac{1000}{+0.11}\\
	S_{n}&={\left(1.11\right)}^{n}\left[\frac{1110}{+0.11}\right]-\frac{1000}{+0.11}
\end{align*}
Por lo tanto,
\begin{align*}
	S_{40}&={\left(1.11\right)}^{40}(10090.090909\ldots)-(-9090.909090\ldots)\\
	S_{40}&=(65.000867\ldots)(10090.090909\ldots)-(9090.909090\ldots)\\
	S_{40}&=655917.842\ldots-(9090.909090\ldots)\\
	S_{40}&\approxeq\$646826.
\end{align*}
¿Puede ser cierto? Pusiste $\$40,000$ y sacaste mayor que $\$600,000$ en intereses. Ahora considere el plan $B$ y sea $T_{n}$ denota el número de dólares en el plan después de (exactamente) $n$ meses de funcionamiento. Entonces $T_{0}=\$100$ y
\begin{align*}
	T_{n+1}&=T_{n}+\text{ interés sobre }T_{n}+\$100\\
	T_{n+1}&= T_{n}+(1/2)\text{ de }10\%\text{ de }T_{n}+\$100\\
	T_{n+1}&=T_{n}\left[1+0.1/12\right]\$100
\end{align*}
En esta RE, $a=12.1/12,c=100$, entonces $\tfrac{c}{1-a}=\tfrac{100}{-0.1/12}=-12000$ y
\begin{equation*}
	T_{n}={\left(12.1/12\right)}^{n}\left[100+12000\right]-12000
\end{equation*}
De ahí, después $40\times12$ meses,
\begin{align*}
	T_{480}&={\left(12.1/12\right)}^{480}(12100)\quad-(12000)\\
	T_{480}&={\left(1.008333\ldots\right)}^{480}(12100)\quad-(12000)\\
	T_{480}&=\left(53.700663\ldots\right)(12100)\quad-(12000)\\
	T_{480}&=649778.0234\ldots \quad-(12000)\\
	T_{480}&\approxeq\$637778.
\end{align*}
Por lo tanto, el plan $A$ tiene un valor ligeramente mayor después de $40$ años.

\section{Resolución de ecuaciones de recurrencia lineal de segundo orden}

Una ecuación de la recurrencia lineal de segundo orden relaciona entradas consecutivas en una secuencia por una ecuación de la forma
\begin{equation}
	S_{n+2}=aS_{n+1}+bS_{n}+c\quad\forall n\text{ en el dominio de }S.
\end{equation}
Pero vamos a asumir que el dominio de $S$ es $\mathds{N}$. Supongamos también que $ab\neq0$, de lo contrario, $S_{n}=c$ para $\forall\,n \in\left\{2\ldots\right\}$, y las soluciones para (2) no son muy interesantes.

¿Qué es de ellos?
El primer orden RE son solo un caso especial de segundo orden RE’s cuando $b=0$.

Cuando $c=0$, se dice que la RE es homogénea (todos los términos se ven igual–una constante veces una entrada de secuencia).

Cuando $c=0$, se dice que la RE es homogénea (todos los términos se ven igual – una constante veces una entrada de secuencia).

El Fibonacci RE es homogénea.

Vamos a restringir también nuestra atención (por el momento) a una ecuación de segundo orden lineal, la recurrencia homogénea
\begin{equation}
	S_{n+2}=aS_{n+1}+bS_{n}\text{ para }\forall n\in\mathds{N}.
\end{equation}
Tal como hicimos para la ecuación de la recurrencia de Fibonacci, supongamos que hay una secuencia geométrica, $S_n=r^n$, que satisface (3)

Si lo hubiera, entonces $r^{n+2}=ar^{n+1}+br^{n}$ para $\forall\,n\in\mathds{N}$.

Cuando $n = 0$, $r^2=ar+b$.

La ``ecuación característica'' de (3) es $x^2-ax-b=0$, que tiene ``raíces'' $r=\tfrac{-(-a)\pm\sqrt{(-a)^2-4(1)(-b)}}{2(1)}=\tfrac{a\pm\sqrt{a^2+4b}}{2}$.

Sea $\Delta=\sqrt{a^2+4b}$, $r_1=\tfrac{a+\Delta}{2}$, y $r_2=\frac{a-\Delta}{2}$.

Entonces $r_{1}+r_{2}=a$, $r_{1}xr_{2}=-b$, y $r_{1}-r_{2}=\Delta$.

¿estos son derechos?
The Greek capital letter delta denotes the “difference” in the roots.
Tanto $r_{1}$ como $r_{2}$ satisfacen la ecuación $x^{2}=ax+b$, y son las únicas soluciones.

\begin{example}{}
Si $S_{n+2}=10S_{n+1}-21S_n$ para $\forall n\in\mathds{N}$, la ecuación característica es $x^{2}-10x+21=0$ o $(x-7)(x-3)=0$ donde, $a=10$, $b=-21$, $a^2+4b=100-84=16$, $\Delta = 4$, entonces $r_{1}=7$ y $r_{2}=3$.
\end{example}

\begin{example}{}
Si $S_{n+2} = 3S_{n+1}-2S_{n}$ para $\forall n\in\mathds{N}$, la ecuación característica es $x^2-3x+2=0$ o $(x-2)(x-1)=0$ donde, $a=3$, $b=-2$, $a^2+4b=9-8=1$, $\Delta = 1$, entonces $r_1=2$ y $r_2=1$.
\end{example}

\begin{example}{}
Si $S_{n+2}=2S_{n+1}-S_{n}$ para $\forall\,n\in\mathds{N}$, la ecuación característica es $x^{2}-2x+1=0$ o $(x-1)(x-1)=0$ donde, $a=2$, $b=-1$, $a^2+4b=4-4=0$, $\Delta = 0$, entonces $r_{1}=1$ y $r_{2}=1$. ¿Pero qué hay de una fórmula que da la solución general?
\end{example}

\begin{theorem}
La solución general de la RE homogénea (3) es
	\begin{align*}
		S_{n}&=A(r_1)^{n}+B(r_2)^{n}\text{, si }r_{1}\neq r_{2}\quad\text{ si}\Delta\neq0\\
		S_{n}&=A(r)^n+Bn(r)^{n}\text{, si }r_{1}=r_{2}=r\quad\text{ si }\Delta=0
	\end{align*}
\end{theorem}
\begin{proof}
Supongamos que $T$ es cualquier solución particular de la RE homogénea. Nos ocupamos de los dos casos por separado.

Caso 1. Si $\Delta\neq0$, entonces las dos raíces son distintas (pero pueden ser números ``complejos'').

Encontraremos valores para $A$ y $B$, luego probaremos que $T_{n}=A(r_1)^{n}+B(r_2)^{n}$ para $\forall\,n\in\mathds{N}$.

Mostraremos que $A(r_1)^{n}+B(r_2)^{n}$ arranca correctamente para valores especialmente elegidos de $A$ y $B$, y luego mostrar $A(r_1)^{n}+B(r_2)^{n}$ continúa correctamente.

Vamos a resolver las ecuaciones (para $A$ y $B$) que garantizaría $T_{n}=A(r_1)^{n}+B(r_{2})^n$, entonces $n=0$ y $n=1$. Si
$T_0 = A(r_1)^0 + B(r_2)^0 = A + B$.................................(1)\\
y $T_1 = A(r_1)^1 + B(r_2)^1 = A(r_1) + B(r_2)$.................................(2)\\

entonces $(r_1)T_0 = A(r_1) + B(r_1)$.......................//multiplicamos (1) por $r_1$\\
y $T_1 = A(r_1) + B(r_2)$.................// (2) otra vez restamos, obtenemos\\

$(r_1)T_0 - T_1 = B(r_1 - r_2) = B\Delta$...............//$r_1 - r_2 = \Delta \neq 0$\\

entonces $B = \frac{(r_1)T_0 - T_1}{\Delta}$\\

Tenemos, $A=T_0 - B =\frac{\Delta T_0}{\Delta} -\frac{(r_1)T_0 - T_1}{\Delta} = \frac{-(r_2)T_0+T_1}{\Delta}$\\
// No importa cómo comience la secuencia T (no importa cuáles sean los valores para $T_0$ y $T_1$)\\
//hay números únicos A y B tales que $T_n = A(r_1)^n + B(r_2)^n$ para $n = 0$ y 1\\
// Continuando la prueba por la inducción matemática que $T_n= A(r_1)^n + B(r_2)^n$ para $\forall \; n \in \mathbb{N}$\\

Paso 1. Si $n=0$ o $1$, entonces $T_{n} = A(r_{1})^{n}+B(r_{2})^{n}$, por nuestra ``opción'' $A$ y $B$.\\
Paso 2. Asuma que $\exists k\geq1$ tal que si $0\leq n\leq k$, entonces $T_{n}=A(r_1)^{n}+B(r_{2})^{n}$.\\
Paso 3. Si $n=k+1$, entonces $n\geq2$ entonces, porque $T$ satisface la RE homogénea (3)\\

\begin{align*}
	T_{k+1}&=aT_{k}+bT_{k-1}\\
	T_{k+1}&=a\left[A(r_1)^k + B(r_2)^k\right]+b\left[A(r_1)^{k-1} + B(r_2)^{k-1}\right] \text{por el paso }2\\
	T_{k+1}&=\left[aA(r_1)^k+bA(r_1)^{k-1}\right]+[aB(r_2)^k+bB(r_2)^{k-1}]\\
	T_{k+1}&=A(r_1)^{k-1}[a(r_1)+b]+B(r_2)^{k-1}[a(r_2)+n]\\
	T_{k+1}&= A(r_1)^{k+1}+B(r_2)^{k+1}
\end{align*}
Así, si $r_{1}\neq r_{2}$, $T_{n}=A(r_1)^n+B(r_2)^n$ para $\forall\,n\in\mathds{N}$.
\end{proof}
% TODO: Checkear bien escrito.
\begin{example}{}
Si $S_{n+2}=10S_{n+1}-21S_{n}$ para $\forall\, n\in\mathds{N}$, entonces $r_{1}=7$ y $r_{2}=3$. Tenemos, la solución general de la RE es $S_{n}=A7^n+B3^{n}$.
\end{example}

\begin{example}{}
Si $S_{n+2}=3S_{n+1}-2S_{n}$ para $\forall n\in\mathds{N}$, entonces $r_{1}=2$ y $r_{2}=1$. Tenemos, la solución general de la RE es $S_{n}= A2^{n}+B1^{n}=A2^{n}+B$.

Caso 2. Si $\Delta=0$, entonces las raíces son (ambos) iguales a $r$ donde $r=a/2$. También, $b=-a^2/4=-r^2$. Si $a$ eran $0$, entonces $b=0$, pero asumimos que no tanto $a$ y $b$ son $0$. De ahí, $r\neq0$. Vamos a resolver las ecuaciones (para $A$ y $B$) que garantizarían $T_{n}= A\left(r)\right^{n}+nB\left(r\right)^{n}$ cuando $n=0$ y $n=1$. Si

$T_{0}=A(r)^{0}+0B(r)^{0}=A$................(1)\\
y $T_{1}= A(r)^{1}+1B(r)^{1}=Ar+Br$, ....................(2)\\

entonces $A=T_{0}$ y $B=(T_{1}-Ar)/r$

No importa cómo comience la sucesión $T$ (no importa cuáles sean los valores para $T_{0}$ y $T_{1}$)
hay números únicos $A$ y $B$ tales que $T_n = A(r)^n + B(r)^n$ para $n = 0$ y 1\\
// Continuando la prueba por la inducción matemática que $T_n= A(r)^n + B(r)^n$ para $\forall \; n \in \mathbb{N}$\\

Paso 1. Si $n=0$ o $n=1$, entonces $T_n=A(r)^n+B(r)^n$, por nuestra ``opción'' $A$ y $B$.

Paso 2. Asuma que $\exists k\geq 1$ tal que si $0\leq n\leq k$, entonces $T_n=A(r)^n+B(r)^n$.

Paso 3. Si $n= k+1$, entonces $n\geq 2$ entonces, porque $T$ satisface la RE homogénea (3).
\end{example}

\begin{align*}
	T_{k+1}&= aT_{k}+bT_{k-1}\\
	T_{k+1}&= a[A(r)^k+kB(r)^k]+b[A(r)^{k-1}+(k-1)B(r)^{k-1}] // \text{ por el paso 2}\\
	T_{k+1}&=[aAr^k+bAr^{k-1}]+[akBr^k+b(k-1)Br^{k-1}]\\
	T_{k+1}&=Ar^{k-1}[ar+b]+Br^{k-1}[akr + b(k-1)]\\
	T_{k+1}&=Ar^{k-1}[r^2]+Br^{k-1}[k(r^{2}) +r^{2}]//r^2 = ar + b\\
	T_{k+1}&=Ar^{k+1}+Br^{k-1}[k(r^2) + r^2]//-b = r^{2}\\
	T_{k+1}&=Ar^{k+1}+(k+1)Br^{k+1}
\end{align*}

Así, si $r_{1}=r_{2}=r$, $T_{n}=A(r)^{n}+nB(r)^{n}$ para $\forall n\in\mathds{N}$.
\begin{prob}
Una sucesión $\left\{x_{n}\right\}$ se dice que es \textbf{sucesión contractiva} si $\exists$ alguna constante $c$, $0<c<1\ni\forall\,n\in\mathbb{N}$, $|x_{n+2}-x_{n+1}|\leq c|x_{n+1}-x_{n}|$. Pruebe que una sucesión contractiva debe ser una sucesión de Cauchy, y por lo tanto converge.
\end{prob}

\begin{prob}
\textbf{Recursiva de la media aritmética} Sea $a\neq b$ números reales arbitrarios, y defina la sucesión $\left\{x_{n}\right\}$ por
\[ x_{1}=a,x_{2}=b,\text{ y }\forall\,n\in\mathbb{N},x_{n+2}=\frac{x_{n+1}+x_{n}}{2}. \] Esto es, cada nuevo término está iniciando con el tercero que es el promedio de los dos términos previos.
\begin{enumerate}
	\item Pruebe que $\left\{x_{n}\right\}$ converge probando que este es una sucesión constructiva.
	\item Pruebe que $\forall\,n\in\mathbb{N}$, $x_{n+1}+\frac{1}{2}x_{n}=b+\frac{1}{2}a$.\label{9b}
	\item Use~\ref{9b} y el álgebra de límites para encontrar que  $\lim\limits_{n\to\infty}x_{n}$. ¿Está sorprendido por la respuesta? Note que si usted intercambia $a$ y $b$ la respuesta podría ser diferente.
\end{enumerate}
\end{prob}

\begin{prob}
\textbf{Recursiva de la media aritmética ponderada} Sean $a\neq b$ dos números reales arbitrarios, sea $0<t<1$, y defina la sucesión $\left\{x_{n}\right\}$ por \[ x_{1}=a, x_{2}=b,\text{ z }\forall\,n\in\mathbb{N}, x_{n+2}=tx_{n}+\left(1-t\right)x_{n+1}. \] Esto es, cada nuevo término que inicia con el tercero que es el promedio ponderado de los términos previos. Geométricamente, $x_{n+2}$ es un punto en el intervalo entre $x_{n}$ y $x_{n+1}$ que corta el intervalo en dos segmentos cuyas longitudes están en la proporción $t$ a $1-t$. Pruebe que $\left\{x_{n}\right\}$ es contractiva, y encuentre su límite.
\end{prob}

\begin{prob}
\textbf{Mapeo contractivo} Sean $a<b$ e $I=\left[a,b\right]$. Una función $f\colon I\rightarrow I$ se dice que es un \textbf{mapeo contractivo} si $\exists c\ni0<c<1$ y $\forall\,x,y\in I$, $|f\left(x\right)-f\left(y\right)|\leq c|x-y|$. Pruebe que un mapeo contractivo debe tener por lo menos un ``punto fijo'', $x\in I\ni f\left(x\right)=x$. También pruebe que $f$ no puede tener más de un punto fijo en $I$.
\end{prob}

\begin{prob}
\textbf{Números de Fibonacci} La sucesión de Fibonacci consiste de los números de Fibonacci, $1,1,2,3,5,8,13,21,\ldots$, y está definido recursivamente por $f_{1}=1$, $f_{2}=1$, y $\forall\,n\geq2$, $f_{n+2}=f_{n+1}+f_{n}$. Cada nuevo término después del segundo es la suma de los dos términos previos. Muchos resultados interesantes han sido probados acerca de los números de Fibonacci--lo suficiente para llenar un libro entero. Deberemos concentrarnos aquí con la sucesión de proporciones de los sucesivos números de Fibonacci. Empezamos definiendo la sucesión por $r_{n}=\frac{f_{n+1}}{f_{n}}$.
\begin{enumerate}
	\item Desarrolle una tabla que muestre los primeros diez términos de $\left\{r_{n}\right\}$. En la basa de esta tabla, conjeture las respuestas a las siguientes preguntas. ¿$\left\{r_{n}\right\}$ es convergente? ¿Es monótona? ¿Eventualmente monótona? ¿Puede encontrar una subsucesión estrictamente creciente? ¿Una subsucesión estrictamente decreciente? (No se requieren demostraciones).
	\item Pruebe que $\forall\,n\in\mathbb{N}$, $r_{n+1}=1+\frac{1}{r_{n}}$.
	\item Pruebe que $\forall\,n\geq2$, $\frac{3}{2}<r_{n}<2$.
	\item Pruebe que $\left\{r_{n}\right\}$ es ``contractiva'', y por lo tanto es una sucesión de Cauchy.
	\item Encuentre $\lim\limits_{n\to\infty}r_{n}$. [Tome nota de este límite; este reaparecerá.]
	\item La ecuación cuadrática $x^{2}-x-1=0$ tiene dos soluciones, $\alpha=\frac{1+\sqrt{5}}{2}$ y $\beta=\frac{1-\sqrt{5}}{2}$. Muestre que $\alpha+\beta=1$, $\alpha^{2}=a+1$, y $\beta^{2}=\beta+1$, y desde estos hechos muestre que $\forall\,n\in\mathbb{N}$, $\alpha^{n+2}=\alpha^{n+1}+\alpha^{n}$ y $\beta^{n+2}=\beta^{n+1}+\beta^{n}$..\label{9f}
	\item $\forall\,n\in\mathbb{N}$, defina $u_{n}=\frac{\alpha^{n}-\beta^{n}}{\alpha-\beta}$, donde $\alpha$ y $\beta$ están definidos en~\ref{9f}. Pruebe que $u_{1}=1$, $u_{2}=1$, y $\forall\,n\geq2$, $u_{n+2}=u_{n+1}+u_{n}$. Por lo tanto, $\left\{u_{n}\right\}$ debe ser la sucesión de Fibonacci. Tenemos encontrado una fórmula para los números de Fibonacci: $f_{n}=u_{n}$.
	\item \textbf{Significado geométrico} de $\alpha$. Considere un rectángulo cuyo ancho $\alpha$ y $largo$ $a+b$ son así proporcionados que cuando un cuadrado de lado $a$ es removido, como se muestra aquí, el rectángulo restante tiene ancho y longitud en la misma proporción. Esto es, $\frac{a+b}{a}=\frac{a}{b}$.
	
	Los matemáticos de la Grecia clásica llamaron esta proporción $R=\frac{a}{b}$ la ``\textbf{Proporción áurea}'' y cualquier rectángulo con lados en la proporción un ``\textbf{rectángulo áureo}''. Ellos consideraron esto como la más estéticamente agradable de todos los rectángulos, y se usó esto frecuentemente en su arte y arquitectura. Pruebe algebraicamente que $R=\alpha$, definida en~\ref{9f} arriba.
	\item Pruebe que $\forall\,n\geq2$, $\forall\,n\geq2$, $f_{n+1}f_{n-1}-{\left(f_{n}\right)}^{2}={\left(-1\right)}^{n}$.
	\item Pruebe que $\forall\,n\in\mathbb{N}$, $r_{n+1}-r_{n}=\frac{{\left(-1\right)}^{n+1}}{f_{n}f_{n+1}}$.\label{9j}
	\item Use~\ref{9j} para probar que $\left\{r_{2n}\right\}$ es estrictamente decreciente y $\left\{r_{2n+1}\right\}$ es estrictamente creciente.
\end{enumerate}
\end{prob}

\begin{prob}
Sea $a\geq1$. Defina la sucesión $\left\{x_{n}\right\}$ por $x_{1}=a$, y $x_{n+1}=a+\frac{1}{x_{m}}$. Pruebe que $\forall\,n\geq2$, $a+\frac{1}{2a}\leq x_{n}\leq 2a$, y use este resultado para probar que $x_{n}$ es contractiva. Encuentre $\lim\limits_{n\to\infty}x_{n}$.
\end{prob}

\begin{prob}
Sea $a>1$. Defina la sucesión $\left\{x_{n}\right\}$ por $x_{1}=a$ y $x_{n}=\frac{1}{a+x_{n}}$. Pruebe que $\forall\,n\in\mathbb{N}$, $\frac{1}{2a}\leq x_{n}\leq a$, y use este resultado para probar que $\left\{x_{n}\right\}$ es contractiva. Encuentre el $\lim\limits_{n\to\infty}x_{n}$. Compare este límite con el Ejercicio anterior.
\end{prob}
%%%%%%%%%%%%%%%%%%%%%%%%%%%%%%%%%%%%%%%% Segunda parte %%%%%%%%%%%%%%%%%%%%%%%%%%%%%%%%%%%%%%%%
\begin{partbacktext}
\part{Realización numérica}
La segunda parte de la monografía se dedica a las realizaciones prácticas de problemas. Combinaremos las consideraciones teóricas sobre diferentes modelos y ecuaciones con las técnicas. Al principio presentamos modelos alternativos para problemas de interacción. En el capítulo $3$ estudiamos la formulación variacional. Este modelo debe ser considerado como la técnica más avanzada. Damos detalles en la construcción de . Segundo, la formulación es introducida en el capítulo $4$. Este nuevo enfoque alternativo es adecuado para problemas con. Nuevamente, presentamos las herramientas necesarias de discretización y simulación. El capítulo $5$ se ocupa de las herramientas para la solución de los problemas algebraicos que surgen de la discretización. En ambos casos, tenemos que lidiar con problemas muy grandes, no lineales. Finalmente, el capítulo $6$ introduce el concepto de tiempo de escala para la reducción de la dimensión de los esquemas discretos que nos permitirá reducir significativamente la complejidad de los sistemas.
\end{partbacktext}
%%%%%%%%%%%%%%%%%%%%%%%%%%%%%%%%%%%%%%%% Tercera parte %%%%%%%%%%%%%%%%%%%%%%%%%%%%%%%%%%%%%%%%
\begin{partbacktext}
\part{Aplicaciones}

\end{partbacktext}
%\chapauthor{Autor}
%\chapsubtitle{Subtítulo}
\chapter{Bifurcación de la ecuación logística}\index{Ecuación logística}
\abstract{Muchas aplicaciones involucran problemas inversos. En esta sección, }
%\chaptermark{xd}
\section{Ecuación diferencial ordinaria lineal}

\begin{pylabcode}[plotsession]
rc('text', usetex=True)
rc('font', **{'family':'serif', 'serif':['Times']})
rc('legend', fontsize=10.0)
x = linspace(0,3*pi)
figure(figsize=(3.25,2))
plot(x, sin(x), label='$\sin x$')
plot(x, sin(x)**2, label='$\sin^{2}(x)$',linestyle='dashed')
xlabel(r'$x$-axis')
ylabel(r'$y$-axis')
xticks(arange(0,4*pi, pi), ('$0$', '$\pi$', '$2\pi$', '$3\pi$'))
axis([0, 3*pi, -1, 1])
legend(loc='lower right')
savefig('img/myplot.pdf', bbox_inches='tight')
\end{pylabcode}

\begin{figure}
	\centering
	\includegraphics{./img/myplot}
	\caption{\label{fig:matlpotlib} A plot created with PythonTeX}
\end{figure}
En este capítulo introducimos algunos conceptos básicos concernientes al cálculo en una escala de tiempo. Una \emph{escala de tiempo} es un subconjunto arbitrario no vacío de los números reales. Así, \[ \mathds{R},\quad\mathds{Z},\quad\mathds{N},\quad\mathds{N}_{0}, \] es decir, los números reales, los enteros, los números naturales, y los enteros no negativos son ejemplos de escala de tiempo, como son \[ \left[0,1\right]\cup\left[2,3\right],\quad\left[0,1\right]\cup\mathds{N},\quad\text{el conjunto de Cantor}, \] mientras que \[ \mathds{Q},\quad\mathds{R}\setminus\mathds{Q},\quad\mathds{C},\quad\left(0,1\right), \] los números racionales, los números irracionales, los números complejos y el intervalo abierto entro $0$ y $1$, \emph{no} son escalas de tiempo. A lo largo de esta monografía denotaremos una escala de tiempo por el símbolo $\mathds{T}$. Asumiremos que una escala de tiempo $\mathds{T}$ tiene la topología que hereda de los números reales con la topología estándar.

El cálculo de escala de tiempo fue iniciado por Stefan Hilger, a fin de crear una teoría que pueda unificar el análisis discreto y continuo. En efecto, abajo en la sección 1.2 introduciremos la derivada delta $f^{\Delta}$ para una función $f$ definida sobre $\mathds{T}$, y resulta que
\begin{enumerate}
	\item $f^{\Delta}=f^{\prime}$ es la derivada usual si $\mathds{T}=\mathds{R}$ y
	\item $f^{\Delta}=\Delta f$ es el operador diferencia posterior usual si $\mathds{T}=\mathds{Z}$.
\end{enumerate}
En esta sección introducimos las nociones básicas conectadas a las escalas de tiempo. Empezamos definiendo los operadores salto posterior y anterior.

\section{Introducción}

\begin{definition}[Escala de tiempo]
Sea $\mathds{T}$ una escala de tiempo. Para $t\in\mathds{T}$ definimos el \emph{operador salto posterior} $\sigma\colon\mathds{T}\rightarrow\mathds{T}$ por \[ \sigma(t)\coloneqq\inf\left\{s\in\mathds{T}:s>t\right\}\quad\text{ para cualquier }\quad t\in\mathds{T}, \] mientras que el \emph{operador salto anterior} $\rho\colon\mathds{T}\rightarrow\mathds{T}$ es definido por \[ \rho(t)\coloneqq\sup\left\{s\in\mathds{T}:s<t\right\}\quad\text{ para cualquier }\quad t\in\mathds{T}. \] En esta definición agregamos el $\inf\emptyset=\sup\mathds{T}$, es decir, $\sigma(M)=M$ si $T$ tiene un máximo $M$ y el $\sup\emptyset=\inf\mathds{T}$, es decir, $\rho(m)=m$ si $\mathds{T}$ tiene un mínimo $m$. Si $\sigma(t)>t$, diremos que $t$ es \emph{dispersa a la derecha}, mientras que si $\rho(t)<t$ diremos que $t$ es \emph{dispersa a la izquierda}. Puntos que son dispersos a la derecha y dispersos a la izquierda en el mismo tiempo son llamados \emph{aislados}. También, si $t<\sup\mathds{T}$ y $\sigma(t)=t$, entonces $t$ es llamado \emph{denso a la derecha}, y si $t>\inf\mathds{T}$ y $\rho=t$, entonces $t$ es llamado \emph{denso a la izquierda}. Los puntos que son denso derecha y denso izquierda se llaman \emph{densos}. Si $T$ tiene un máximo disperso a la derecha $M$, entonces definimos $\mathds{T}^{\kappa}=\mathds{T}\setminus\{M\}$, caso contrario $\mathds{T}^{\kappa}=\mathds{T}$. Finalmente, la función \emph{grano} $\mu\colon\mathds{T}\rightarrow\left[0,\infty\right)$ es definida por \[ \mu(t)\coloneqq\sigma(t)-t\quad\text{para cualquier}\quad t\in\mathds{T}. \]
\end{definition}

\section{Diferenciación}

Ahora consideremos una función $f\colon\mathds{T}\rightarrow\mathds{R}$ y definir el llamado delta derivada (o Hilger) de $f$ en un punto $t\in\mathds{T}^{\kappa}$.

\begin{definition}[Delta diferenciable]
	Asuma que $f\colon\mathds{T}\rightarrow\mathds{R}$ es una función y sea $t\in\mathds{T}^{\kappa}$. Entonces definimos el número $f^{\Delta}(t)$  (siempre que este exista) con la propiedad que dado cualquier $\varepsilon>0$, existe una vecindad $U$ de $t$ (es decir, $U=\left(t-\delta\right)\cap\mathds{T}$ para algún $\delta>0$) tal que \[ |f(\sigma(t))|-f(s)-f^{\Delta}(t)(\sigma(t)-s)|\leq\varepsilon|\sigma(t)-s|\quad\text{para cualquier}\quad s\in U. \] Llamamos $f^{\Delta}(t)$ la derivada delta (o Hilger) de $f$ en $t$. Es más, diremos que $f$ es \emph{delta} (o Hilger) \emph{diferenciable} (o en breve: \emph{diferenciable}) en $\mathds{T}^{\kappa}$ siempre que $f^{\Delta}(t)$ exista para cualquier $t\in\mathds{T}^{\kappa}$. La función $f^{\Delta}\colon\mathds{T}^{\kappa}\rightarrow\mathds{R}$ es entonces llamada la derivada (delta) de $f$ sobre $\mathds{T}^{\kappa}$.

	Algunas relaciones sencillas y útiles en relación con la derivada delta se dan a continuación.
\end{definition}

\begin{theorem}{}
	Asuma que $f\colon\mathds{T}\rightarrow\mathds{R}$ es una función y sea $t\in\mathds{T}^{k}$. Entonces tenemos lo siguiente:
	\begin{enumerate}
		\item Si $f$ es diferenciable en $\mathds{T}$, entonces $f$ es continua en $t$.
		\item Si $f$ es continua en $t$ y $t$ es dispersa a la derecha, entonces $f$ es diferenciable en $t$ con \[ f^{\Delta}(t)=\frac{f(\sigma(t))-f(t)}{\mu(t)}. \]
		\item Si $t$ es densa a la derecha, entonces $f$ es diferenciable en $t$ sii el límite \[ \lim_{s\to t}\frac{f(t)-f(s)}{t-s} \] existe como un número finito. En este caso \[ f^{\Delta}(t)=\lim_{s\to t}\frac{f(t)-f(s)}{t-s}. \]
		\item Si $f$ es diferenciable en $t$, entonces \[ f(\sigma(t))=f(t)+\mu(t)f^{\Delta}(t). \]
	\end{enumerate}
\end{theorem}
\begin{exercise}
	Muestre que si $\mathds{T}=q^{\mathds{N}_{0}}\coloneqq\left\{q^{n}:n\in\mathds{N}_{0}\right\}$, $q>1$, entonces \[ {\left(\log t\right)}^{\Delta}=\frac{\log q}{q-1}\cdot\frac{1}{t}. \]
\end{exercise}
\begin{example}{}
	Nuevamente consideremos los dos casos $\mathds{T}=\mathds{R}$ y $\mathds{T}=\mathds{Z}$.
	\begin{enumerate}
		\item Si $\mathds{T}=\mathds{R}$, entonces el Teorema 1.3 resulta que $f\colon\mathds{R}\rightarrow\mathds{R}$ es delta diferenciable en $t\in\mathds{R}$ sii \[ f^{\prime}(t)=\lim_{s\to t}\frac{f(t)-f(s)}{t-s}\quad\text{existe}, \] es decir, sii $f$ es diferenciable (en el sentido clásico) en $t$. En este caso tenemos entonces \[ f^{\Delta}(t)=\lim_{s\to t}\frac{f(t)-f(s)}{t-s}=f^{\prime}(t) \] por el Teorema 1.3 (iii).
		\item Si $\mathds{T}=\mathds{Z}$, entonces el Teorema 1.3 (ii) resulta que $f\colon\mathds{Z}\rightarrow\mathds{R}$ es delta diferenciable en $t\in\mathds{Z}$ con \[ f^{\Delta}(t)=\frac{f(\sigma(t))-f(t)}{\mu(t)}=\frac{f(t+1)-f(t)}{1}=f(t+1)-f(t)=\Delta f(t), \] donde $\Delta$ es el \emph{operador diferencia posterior} usual definida por la última ecuación de arriba.
	\end{enumerate}
\end{example}

A continuación, nos gustaría poder encontrar las derivadas de sumas, productos, y cocientes de funciones diferenciables. Esto es posible de acuerdo con el siguiente teorema:
\begin{theorem}{}
	Asuma que $f,g\colon\mathds{T}\rightarrow\mathds{R}$ son diferenciables en $t\in\mathds{T}^{\kappa}$. Entonces
	\begin{enumerate}
		\item La suma de $f+g\colon\mathds{T}\rightarrow\mathds{R}$ es diferenciable en $a$ con \[ {\left(f+g\right)}^{\Delta}(t)=f^{\Delta}(t)+g^{\Delta}(t). \]
		\item Para cualquier constante $\alpha$, $\alpha f\colon\mathds{T}\rightarrow\mathds{R}$ es diferenciable en $t$ con \[ {\left(\alpha f\right)}^{\Delta}\left(t\right)=f^{\Delta}\left(t\right)+g^{\Delta}\left(t\right). \]
		\item El producto $fg\colon\mathds{T}\rightarrow\mathds{R}$ es diferenciable en $t$ con \[ {\left(fg\right)}^{\Delta}\left(t\right)=f^{\Delta}\left(t\right)g\left(t\right)+f\left(\sigma\left(t\right)\right)g^{\Delta}\left(t\right)=f\left(t\right)g^{\Delta}\left(t\right)+f^{\Delta}\left(t\right)g\left(\sigma\left(t\right)\right). \]
		\item Si $f\left(t\right)f\left(\sigma\left(t\right)\right)\neq0$, entonces $\frac{1}{f}$ es diferenciable en $t$ con \[ {\left(\frac{1}{f}\right)}^{\Delta}\left(t\right)=-\frac{f^{\Delta}\left(t\right)}{f\left(t\right)f\left(\sigma\left(t\right)\right)}. \]
		\item Si $g\left(t\right)g\left(\sigma\left(t\right)\right)\neq0$, entonces $\frac{f}{g}$ es diferenciable en $t$ y \[ {\left(\frac{f}{g}\right)}^{\Delta}\left(t\right)=\frac{f^{\Delta}\left(t\right)g\left(t\right)-f\left(t\right)g^{\Delta}\left(t\right)}{g\left(t\right)g\left(\sigma\left(t\right)\right)}. \]
	\end{enumerate}
\end{theorem}
\begin{proof}
	Asuma que $f$ y $g$ son delta diferenciables en $t\in\mathds{T}^{\kappa}$.
	\begin{enumerate}
		\item Sea $\varepsilon>0$. Entonces, existen vecindades $U_{1}$ y $U_{2}$ de $t$ con \[ \left|f\left(\sigma\left(t\right)\right)-f\left(s\right)-f^{\Delta}\left(t\right)\left(\sigma\left(t\right)-s\right)\right|\leq\frac{\varepsilon}{2}\left|\sigma\left(t\right)-s\right|\quad\text{para todo}\quad s\in U_{1} \] y \[ \left|g\left(\sigma\left(t\right)\right)-g\left(s\right)-g^{\Delta}\left(t\right)\left(\sigma\left(t\right)-s\right)\right|\leq\frac{\varepsilon}{2}\left|\sigma\left(t\right)-s\right|\quad\text{para todo}\quad s\in U_{2}. \] Sea $U=U_{1}\cap U_{2}$. Entonces tenemos para todo $s\in U$
		\begin{align*}
		=+\\
		\end{align*}
		Por lo tanto, $f+g$ es diferenciable en $t$ y ${\left(\right)}^{\Delta}$
	\end{enumerate}
\end{proof}
%%\motto{Use the template \emph{chapter.tex} to style the various elements of your chapter content.}
%\chapter{}
%\label{intro} % Always give a unique label
% use \chaptermark{}
% to alter or adjust the chapter heading in the running head
%
%\abstract*{}
%
%\abstract{El cálculo de variaciones se desarrolló a partir del problema de la curva braquistócrona, planteado inicialmente por \textbf{Johann Bernoulli} (1696). Inmediatamente este problema captó la atención de Jakob Bernoulli y el marqués de l'Hôpital, aunque fue \textbf{Leonhard Euler} el primero que elaboró una teoría del cálculo variacional. Las contribuciones de Euler se iniciaron en 1733 con su Elementa Calculi Variationum (``Elementos del cálculo de variaciones'') que da nombre a la disciplina. \newline\indent
%\textbf{Lagrange} contribuyó extensamente a la teoría y Legendre (1786) asentó un método, no enteramente satisfactorio para distinguir entre máximos y mínimos. \textbf{Isaac Newton} y \textbf{Gottfried Leibniz} también prestaron atención a este asunto.}
%
%\section{Otras publicaciones}
%\label{sec:1}
%Otros trabajos destacados fueron los de Vincenzo Brunacci (1810), Carl Friedrich Gauss (1829), Siméon Poisson (1831), Mijaíl Ostrogradski (1834) y Carl Jacobi (1837). Un trabajo general particularmente importante es el de Sarrus (1842) que fue resumido por Cauchy (1844). Otros trabajos destacados posteriores son los de Strauch (1849), Jellett (1850), Otto Hesse (1857), Alfred Clebsch (1858) y Carll (1885), aunque quizá el más importante de los trabajos durante el siglo XIX es el de \textbf{Weierstrass}. Este importante trabajo fue una referencia estándar y es el primero que trata el cálculo de variaciones sobre una base firme y rigurosa. Los problema 20 y 23 de Hilbert planteados en 1900 estimularon algunos desarrollos posteriores. Durante el siglo XX, David Hilbert, Emmy Noether, Leonida Tonelli, Henri Lebesgue y Jacques Hadamard, entre otros, hicieron contribuciones notables.\par
%\textbf{Marston Morse} aplicó el cálculo de variaciones a lo que actualmente se conoce como teoría de Morse. Lev Semenovich Pontryagin, Ralph Rockafellar y Clarke desarrollaron nuevas herramientas matemáticas dentro de la teoría del control óptimo, generalizando el cálculo de variaciones.
%
%\section{Fermat, Bernoulli, Newton y Leibniz}
%\label{sec:2}
%% Always give a unique label
%% and use \ref{<label>} for cross-references
%% and \cite{<label>} for bibliographic references
%% use \sectionmark{}
%%% to alter or adjust the section heading in the running head
%%
%%% For figures use
%%%
%\begin{figure}[!ht]
%	\sidecaption[t]
%	% Use the relevant command for your figure-insertion program
%	% to insert the figure file.
%	% For example, with the option graphics use
%	\includegraphics[width=0.343\textwidth]{bola.jpg}
%	%
%	% If not, use
%	%\picplace{5cm}{2cm} % Give the correct figure height and width in cm
%	%
%	\caption{Dados dos puntos $A$ y $B$, con $A$ a una elevación mayor que $B$, existe solo una curva cicloide con la concavidad hacia arriba que pasa por $A$ con pendiente infinita (dirección vertical y sentido de arriba hacia abajo), también pasa por $B$ y no posee puntos máximos entre $A$ y $B$.}
%	\label{fig:1}       % Give a unique label
%\end{figure}
%
%\eject
%
%%\begin{eqnarray}
%%\left|\nabla U_{\alpha}^{\mu}(y)\right| &\le&\frac1{d-\alpha}\int
%%\left|\nabla\frac1{|\xi-y|^{d-\alpha}}\right|\,d\mu(\xi) =
%%\int \frac1{|\xi-y|^{d-\alpha+1}} \,d\mu(\xi)\qquad  \\
%%&=&(d-\alpha+1) \int\limits_{d(y)}^\infty
%%\frac{\mu(B(y,r))}{r^{d-\alpha+2}}\,dr \le (d-\alpha+1)
%%\int\limits_{d(y)}^\infty \frac{r^{d-\alpha}}{r^{d-\alpha+2}}\,dr
%%\label{eq:01}
%%\end{eqnarray}
%
%\enlargethispage{24pt}
%
%\begin{quotation}
%Please do not use quotation marks when quoting texts! Simply use the \verb|quotation| environment -- it will automatically be rendered in the preferred layout.
%\end{quotation}
%\subsection{Algunas observaciones, demostraciones y aplicaciones de Euler, Lagrange y Jacobi en el Cálculo variacional}
%Instead of simply listing headings of different levels we recommend to let every heading be followed by at least a short passage of text. Furtheron please use the \LaTeX\ automatism for all your cross-references and citations as has already been described in Sect.~\ref{subsec:2}, see also Fig.~\ref{fig:1}\footnote{If you copy text passages, figures, or tables from other works, you must obtain \textit{permission} from the copyright holder (usually the original publisher). Please enclose the signed permission with the manucript. The sources\index{permission to print} must be acknowledged either in the captions, as footnotes or in a separate section of the book.}
%\paragraph{Paragraph Heading} %
%Instead of simply listing headings of different levels we recommend to let every heading be followed by at least a short passage of text. Furtheron please use the \LaTeX\ automatism for all your cross-references and citations as has already been described in Sect.~\ref{sec:2}.
%
%Please note that the first line of text that follows a heading is not indented, whereas the first lines of all subsequent paragraphs are.
%
%For typesetting numbered lists we recommend to use the \verb|enumerate| environment -- it will automatically render Springer's preferred layout.
%\begin{figure}[h]
%	\centering
%	\includegraphics[width=0.6\textwidth]{grafica.jpg}
%	\caption{Costo de producción proporcional a la raíz cuadrada de la tasa de producción.}
%\end{figure}
%\begin{figure}[t]
%\sidecaption[t]
%% Use the relevant command for your figure-insertion program
%% to insert the figure file.
%% For example, with the option graphics use
%\includegraphics[scale=.65]{figure}
%%
%% If not, use
%%\picplace{5cm}{2cm} % Give the correct figure height and width in cm
%%
%\caption{Please write your figure caption here}
%\label{fig:2}       % Give a unique label
%\end{figure}
%
%\runinhead{Run-in Heading Boldface Version} Use the \LaTeX\ automatism for all your cross-references and citations as has already been described in Sect.~\ref{sec:2}.
%
%\subruninhead{Run-in Heading Boldface and Italic Version} Use the \LaTeX\ automatism for all your cross-refer\-ences and citations as has already been described in Sect.~\ref{sec:2}\index{paragraph}.
%
%\subsubruninhead{Run-in Heading Displayed Version} Use the \LaTeX\ automatism for all your cross-refer\-ences and citations as has already been described in Sect.~\ref{sec:2}\index{paragraph}.
%% Use the \index{} command to code your index words
%%
%% For tables use
%%
%\begin{table}[!t]
%\caption{Please write your table caption here}
%\label{tab:1}       % Give a unique label
%%
%% For LaTeX tables use
%%
%\begin{tabular}{p{2cm}p{2.4cm}p{2cm}p{4.9cm}}
%\hline\noalign{\smallskip}
%Classes & Subclass & Length & Action Mechanism  \\
%\noalign{\smallskip}\svhline\noalign{\smallskip}
%Translation & mRNA$^a$  & 22 (19--25) & Translation repression, mRNA cleavage\\
%Translation & mRNA cleavage & 21 & mRNA cleavage\\
%Translation & mRNA  & 21--22 & mRNA cleavage\\
%Translation & mRNA  & 24--26 & Histone and DNA Modification\\
%\noalign{\smallskip}\hline\noalign{\smallskip}
%\end{tabular}
%$^a$ Table foot note (with superscript)
%\end{table}
%%
%\section{Section Heading}
%\label{sec:3}
%% Always give a unique label
%% and use \ref{<label>} for cross-references
%% and \cite{<label>} for bibliographic references
%% use \sectionmark{}
%% to alter or adjust the section heading in the running head
%Instead of simply listing headings of different levels we recommend to let every heading be followed by at least a short passage of text. Furtheron please use the \LaTeX\ automatism for all your cross-references and citations as has already been described in Sect.~\ref{sec:2}.
%
%Please note that the first line of text that follows a heading is not indented, whereas the first lines of all subsequent paragraphs are.
%
%If you want to list definitions or the like we recommend to use the Springer-enhanced \verb|description| environment -- it will automatically render Springer's preferred layout.
%
%\begin{description}[Type 1]
%\item[Type 1]{That addresses central themes pertainng to migration, health, and disease. In Sect.~\ref{sec:1}, Wilson discusses the role of human migration in infectious disease distributions and patterns.}
%\item[Type 2]{That addresses central themes pertainng to migration, health, and disease. In Sect.~\ref{subsec:2}, Wilson discusses the role of human migration in infectious disease distributions and patterns.}
%\end{description}
%
%\subsection{Subsection Heading} %
%In order to avoid simply listing headings of different levels we recommend to let every heading be followed by at least a short passage of text. Use the \LaTeX\ automatism for all your cross-references and citations citations as has already been described in Sect.~\ref{sec:2}.
%
%Please note that the first line of text that follows a heading is not indented, whereas the first lines of all subsequent paragraphs are.
%
%\begin{svgraybox}
%If you want to emphasize complete paragraphs of texts we recommend to use the newly defined Springer class option \verb|graybox| and the newly defined environment \verb|svgraybox|. This will produce a 15 percent screened box 'behind' your text.
%
%If you want to emphasize complete paragraphs of texts we recommend to use the newly defined Springer class option and environment \verb|svgraybox|. This will produce a 15 percent screened box 'behind' your text.
%\end{svgraybox}
%
%
%\subsubsection{Subsubsection Heading}
%Instead of simply listing headings of different levels we recommend to let every heading be followed by at least a short passage of text. Furtheron please use the \LaTeX\ automatism for all your cross-references and citations as has already been described in Sect.~\ref{sec:2}.
%
%Please note that the first line of text that follows a heading is not indented, whereas the first lines of all subsequent paragraphs are.
%
%\begin{theorem}
%Theorem text goes here.
%\end{theorem}
%%
%% or
%%
%\begin{definition}
%Definition text goes here.
%\end{definition}
%
%\begin{proof}
%%\smartqed
%Proof text goes here.
%%\qed
%\end{proof}
%
%\paragraph{Paragraph Heading} %
%Instead of simply listing headings of different levels we recommend to let every heading be followed by at least a short passage of text. Furtheron please use the \LaTeX\ automatism for all your cross-references and citations as has already been described in Sect.~\ref{sec:2}.
%
%Note that the first line of text that follows a heading is not indented, whereas the first lines of all subsequent paragraphs are.
%%
%% For built-in environments use
%%
%\begin{theorem}
%Theorem text goes here.
%\end{theorem}
%%
%\begin{definition}
%Definition text goes here.
%\end{definition}
%%
%\begin{proof}
%%\smartqed
%Proof text goes here.
%%\qed
%\end{proof}
%%
%%
%\begin{trailer}{Cabeza de remolque}
%If you want to emphasize complete paragraphs of texts in an \verb|Trailer Head| we recommend to
%use  \begin{verbatim}\begin{trailer}{Trailer Head}
%...
%\end{trailer}\end{verbatim}
%\end{trailer}
%%
%\begin{question}{Preguntas}
%If you want to emphasize complete paragraphs of texts in an \verb|Questions| we recommend to
%use  \begin{verbatim}\begin{question}{Questions}
%...
%\end{question}\end{verbatim}
%\end{question}
%%
%%
%\begin{important}{Importante}
%If you want to emphasize complete paragraphs of texts in an \verb|Important| we recommend to
%use  \begin{verbatim}\begin{important}{Important}
%...
%\end{important}\end{verbatim}
%\end{important}
%%
%\clearpage
%\begin{warning}{Atención}
%If you want to emphasize complete paragraphs of texts in an \verb|Attention| we recommend to
%use  \begin{verbatim}\begin{warning}{Attention}
%...
%\end{warning}\end{verbatim}
%\end{warning}
%
%\begin{programcode}{Código de programa}
%If you want to emphasize complete paragraphs of texts in an \verb|Program Code| we recommend to
%use
%
%\verb|\begin{programcode}{Program Code}|
%
%\verb|\begin{verbatim}...\end{verbatim}|
%
%\verb|\end{programcode}|
%
%\end{programcode}
%%
%\begin{tips}{Consejos}
%If you want to emphasize complete paragraphs of texts in an \verb|Tips| we recommend to
%use  \begin{verbatim}\begin{tips}{Tips}
%...
%\end{tips}\end{verbatim}
%\end{tips}
%%
%%
%\begin{overview}{Visión general}
%If you want to emphasize complete paragraphs of texts in an \verb|Overview| we recommend to
%use  \begin{verbatim}\begin{overview}{Overview}
%...
%\end{overview}\end{verbatim}
%\end{overview}
%\clearpage
%\begin{backgroundinformation}{Background Information}
%If you want to emphasize complete paragraphs of texts in an \verb|Background|
%\verb|Information| we recommend to
%use
%
%\verb|\begin{backgroundinformation}{Background Information}|
%
%\verb|...|
%
%\verb|\end{backgroundinformation}|
%\end{backgroundinformation}
%\begin{legaltext}{Legal Text}
%If you want to emphasize complete paragraphs of texts in an \verb|Legal Text| we recommend to
%use  \begin{verbatim}\begin{legaltext}{Legal Text}
%...
%\end{legaltext}\end{verbatim}
%\end{legaltext}
%%
%\begin{acknowledgement}
%If you want to include acknowledgments of assistance and the like at the end of an individual chapter please use the \verb|acknowledgement| environment -- it will automatically render Springer's preferred layout.
%\end{acknowledgement}
%%
%\section*{Apéndice}
%\addcontentsline{toc}{section}{Apéndice}
%%
%When placed at the end of a chapter or contribution (as opposed to at the end of the book), the numbering of tables, figures, and equations in the appendix section continues on from that in the main text. Hence please \textit{do not} use the \verb|appendix| command when writing an appendix at the end of your chapter or contribution. If there is only one the appendix is designated ``Appendix'', or ``Appendix 1'', or ``Appendix 2'', etc. if there is more than one.
%
%\begin{equation}
%a \times b = c
%\end{equation}
%% Problems or Exercises should be sorted chapterwise
%\section*{Problemas}
%\addcontentsline{toc}{section}{Problems}
%%
%% Use the following environment.
%% Don't forget to label each problem;
%% the label is needed for the solutions' environment
%\begin{prob}
%\label{prob1}
%A given problem or Excercise is described here. The
%problem is described here. The problem is described here.
%\end{prob}
%
%\begin{prob}
%\label{prob2}
%\textbf{Problem Heading}\\
%(a) The first part of the problem is described here.\\
%(b) The second part of the problem is described here.
%\end{prob}
\begin{frame}\transblindsvertical
\frametitle{Referencias}
	%------------------------------------------------------------ 1
	\only<1>{
		\framesubtitle{The first frame subtitle}
		\begin{itemize}
			\item Libros
			\nocite{*}
			\printbibliography[heading=none,keyword=book]
		\end{itemize}
	}
	%------------------------------------------------------------ 2
	\only<2>{
		\framesubtitle{The second frame subtitle}
		\begin{itemize}
			\item Artículos matemáticos
			\printbibliography[heading=none,keyword=paper]
		\end{itemize}
	}
	%------------------------------------------------------------ 3
	\only<2>{
		\framesubtitle{The second frame subtitle}
		\begin{itemize}
			\item Sitios web
			\printbibliography[heading=none,keyword=online]
		\end{itemize}
	}
\end{frame}
\begin{frame}\transblindsvertical
\frametitle{Referencias}
	%------------------------------------------------------------ 1
	\only<1>{
		\framesubtitle{The first frame subtitle}
		\begin{itemize}
			\item Libros
			\nocite{*}
			\printbibliography[heading=none,keyword=book]
		\end{itemize}
	}
	%------------------------------------------------------------ 2
	\only<2>{
		\framesubtitle{The second frame subtitle}
		\begin{itemize}
			\item Artículos matemáticos
			\printbibliography[heading=none,keyword=paper]
		\end{itemize}
	}
	%------------------------------------------------------------ 3
	\only<2>{
		\framesubtitle{The second frame subtitle}
		\begin{itemize}
			\item Sitios web
			\printbibliography[heading=none,keyword=online]
		\end{itemize}
	}
\end{frame}

\backmatter
\include{./04_backmatter/appendix}
\Extrachap{Glosario}

\runinhead{término del glosario}
%\include{./04_backmatter/solutions}
\printindex
\end{document}